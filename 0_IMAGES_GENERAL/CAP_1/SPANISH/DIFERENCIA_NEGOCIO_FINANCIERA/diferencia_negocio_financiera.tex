\documentclass{standalone}

\begin{document}

% \begin{table}
	\begin{tabular}{|cp{6.5cm}|cp{6.5cm}|}
		\hline 
		& \textbf{Avalúo de Negocio en Marcha} & 
		& \textbf{Avalúo de Entidades Financieras} \\
		\hline 
		\hline 
		a. & El dinero es inversión para Capital. & 
		a. & El dinero es la materia prima. \\[3mm] 
		b. & Crecimiento asociado a la reinversión. &
		b. & Crecimiento depende de la inyección de capital que tenga la empresa 
		para intermediar. \\[3mm] 
		c. & El Costo de capital (WACC) se calcula fácilmente ya que es identificable
		la deuda explícita y el capital. &
		c. & El Costo de capital por su estructura altamente apalancada es difícil
		de calcular. \\[3mm] 
		d. & Utiliza modelos de FCFF (Free Cash Flow to Firm). &
		d. & Utiliza modelo de FCFE (Free Cash Flow to Equity). \\[3mm] 
		e. & Se aplica comúnmente el modelo EVA para calcular el crédito mercantil o
		goodwill. &
		e. & Se utilizan modelos como el Exceso de rendimiento (valor presente del
		exceso de rendimientos que la firma espera en un futuro). \\
		\hline 
	\end{tabular}
% \end{table}

\end{document}
