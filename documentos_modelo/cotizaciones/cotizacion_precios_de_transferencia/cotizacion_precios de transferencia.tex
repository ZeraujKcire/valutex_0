\documentclass[5pt,letter]{report}
\usepackage{import}
\import{../../../sistema/}{rutas}
\input{0.preambulo1}
\usepackage{textcomp}
\decimalpoint
%------------Diseño de página --------------------------------
\usepackage[centering, letterpaper,margin=2cm,top=1cm, headsep=24pt, headheight=2cm,includehead, includefoot]{geometry}

%------------Colores----------------------

\definecolor{principal}{RGB}{0, 53, 73}
\definecolor{secundario}{RGB}{43, 82, 96}
\definecolor{terciario}{RGB}{43,82,96}

%--------------Hipervinculos en color, ligas-azul, archivos-magenta, url-azul----------------
\hypersetup{
    colorlinks=true,
    linkcolor=secundario,
    filecolor=magenta,      
    urlcolor=blue}

%-------------Encabezado---------------------
\pagestyle{fancy}
\fancyhf{}

\chead{\includegraphics[width=8cm]{\rutaImagenes/logo_valuami_fondo_blanco}}

%--------------Pie de página------------------
\cfoot{\textbf{\textcolor{principal}{\footnotesize{VALUAMI\tiny\textregistered}}}\\ \scriptsize{\textit{Amargura 50, Interior 7 y 8, Antigua Granada Parques de la Herradura.\\ Huixquilucan, Estado de M\'exico. CP 52786.\\ Tel. 52 94 76 80 / 55 89 96 34\\ \url{www.ami-mexico.com/valuami}}}}

%------------Títulos de seccion y subseccion--------------



%\renewcommand \thechapter {\Roman{chapter}}
\renewcommand \thesection {\Roman{section}}
%\renewcommand \thesubsection {\thesection.\arabic{subsection}}
%\renewcommand \thesubsubsection {\thesection.\arabic{subsection}.\arabic{subsubsection}}

\titleformat{\section}[hang]{\color{gray}\huge\bfseries}{\thesection.}{1em}{} 

%\chapterfont{\color{principal}}
\sectionfont{\color{principal}}
%\subsectionfont{\color{secundario}}
%\subsubsectionfont{\color{terciario}}

%------------------Profundidad del índice---------------------------

\setcounter{tocdepth}{3}
\setcounter{secnumdepth}{3}




%------------------Marca de agua---------------

\backgroundsetup{angle=0, contents={\includegraphics[width=8cm]{\rutaImagenes/logo_valuami_fondo_blanco}},opacity=.3, scale=1}


\newcommand{\fechaCotizacion}{\today}
\newcommand{\lugarCotizacion}{\inserta}
\newcommand{\empresaSolicitante}{\inserta}
\newcommand{\personaSolicitante}{\inserta}
\newcommand{\caracterSolicitante}{\inserta}
\newcommand{\honorarios}{\inserta}
\newcommand{\honorariosLetra}{\inserta}
\newcommand{\anticipo}{\inserta}
\newcommand{\liquidacion}{\inserta}
\newcommand{\totalMasIva}{\inserta}


\begin{document}

\begin{minipage}{12cm}

\textbf{Para:}		A quien corresponda.\\[5pt]

\textbf{De:}		Correduría Pública 14 de N.L.\\[5pt]

\textbf{Fecha:}	8 de Mayo de 2023.\\[5pt]

\textbf{Ref:} 		Propuesta de Servicios de precios de transferencia 2022

\end{minipage}

\begin{center}
\vspace{20pt}
\textcolor{principal}{\textbf{Privilegiado y Confidencial}}
\end{center}

Estimados:\\[5pt]

De acuerdo a su solicitud, a continuación me permito someter a su consideración nuestra propuesta de servicios y honorarios para la determinación de precios de transferencia por las operaciones entre las empresas mencionadas más adelante, considerando que son partes relacionadas, por el periodo comprendido del 1o. de enero al 31 de diciembre de 2022.\\[5pt]

Sin más por el momento me pongo a sus órdenes para cualquier aclaración a la presente.\\[5pt] 

\section{TRANSACCIONES SUJETAS A ANÁLISIS}

Las transacciones con partes relacionadas a evaluarse como parte de la propuesta son las siguientes: \\

\begin{figure}[H]
\centering
\includegraphics[width=\textwidth]{transacciones}
\end{figure}


En caso de que existieran transacciones con partes relacionadas adicionales a las mencionadas anteriormente, éstas se adicionarán al estudio de precios de transferencia y se elaborará una nueva propuesta de servicios incluyendo el total de las transacciones.

\section{OBJETIVOS}

Los objetivos del proyecto propuesto se resumen a continuación: 

\begin{enumerate}[A)]

\item \textit{\underline{Análisis Comprobatorio}}\\ 

Una vez cerrados los libros contables del ejercicio fiscal 2022, se realizará el análisis de las transacciones descritas anteriormente a fin de confirmar que éstas se efectuaron a valores de mercado, conforme a lo requerido por el artículo 76 fracción IX y XII de la Ley del Impuesto Sobre la Renta (en adelante LISR). 

\item \textit{\underline{Preparación de Documentación Comprobatoria}} \\

Los resultados del análisis se documentarán en un reporte de precios de transferencia, el cual también contendrá información sobre las partes relacionadas, las transacciones evaluadas y los métodos aplicados en el análisis. Este reporte será la documentación comprobatoria de precios de transferencia de LA SOLICITANTE; para el ejercicio fiscal 2022, conforme a lo requerido en el artículo 76 fracción IX y XII de la LISR. 

\end{enumerate}

\section{PLAN DE TRABAJO}

Para cumplir con los objetivos del proyecto, el trabajo se realizará en tres fases:

\begin{enumerate}[A)]


\item \textit{\underline{Fase I: Recopilación y documentación de hechos relevantes.}}\\ 

La documentación de estos aspectos relevantes se llevará a cabo a través de entrevistas personales y/o llamadas telefónicas que se realizarán con el personal de LA SOLICITANTE. Dichas entrevistas y conferencias serán enfocadas en los siguientes aspectos de cada una de las transacciones sujetas a análisis.\footnote{Tratándose de personas morales que celebren operaciones con partes relacionadas, estas deberán determinar sus ingresos acumulables y sus deducciones autorizadas, considerando para esas operaciones los precios y montos de las contraprestaciones que hubieran utilizado con o entre partes independientes en operaciones comparables. según se establece en el Artículo 76, Fracción XII de LISR (2020). } 

\begin{enumerate}[1.]

\item Las características específicas de los créditos.

\item Las condiciones particulares de la industria y de la economía que pudieran afectar dichos términos. 

\item Las funciones realizadas, activos utilizados y riesgos asumidos por las partes en las transacciones. 

\item Los términos de los acuerdos (tácitos y/o formalizados) entre las partes y 

\item Las estrategias de negocios de LA SOLICITANTE y/o las empresas del grupo que puedan afectar los resultados en las transacciones por evaluarse. 

\end{enumerate}

\item \textit{\underline{Fase II:  Análisis de Precios de Transferencia}} 

El análisis de precios de transferencia considerará los aspectos relevantes de las transacciones y consistirá en las siguientes actividades: 

\begin{enumerate}[1.]

\item La identificación de transacciones que LA SOLICITANTE realice con terceros independientes bajo circunstancias similares a las transacciones por evaluarse (Comparables Internos). De no existir estas transacciones comparables, se procederá a la búsqueda de empresas independientes que realicen transacciones comparables con terceros (Comparables externos). 


\item La determinación del método más adecuado para cada tipo de transacción. 

\item La aplicación de ajustes para mejorar la comparabilidad. 

\item La determinación de rangos de precios o tasas. 

\item La comprobación de que los precios, tasas o márgenes obtenidos por la compañía son consistentes con los rangos determinados de las transacciones comparables. 

\end{enumerate}


\item \textit{\underline{Fase III: Documentación}} \\[5pt]

El resultado de nuestro trabajo se presentará en un documento, el cual se detalla a continuación: 

\item \textit{\underline{Estudio de precios de transferencia}} 

El contenido de la documentación comprobatoria estará de acuerdo con los requerimientos establecidos en el artículo 76 fracción IX y XII de la LISR, y tendrá como base de su preparación la documentación de aspectos relevantes de las transacciones y el análisis de comparabilidad considerando los estados financieros auditados del ejercicio 2022 y la información financiera segmentada proporcionada por LA SOLICITANTE. Estimamos la entrega de un borrador de la documentación comprobatoria aproximadamente seis semanas después de recibir la información.\\[5pt] 

El estudio final de precios de transferencia contendrá información general relativa a las transacciones analizadas, los hechos relevantes de dichas transacciones, así como los métodos aplicados y las conclusiones derivadas de dicho análisis. Este análisis estará basado en la información financiera relativa a las consideraciones que surjan de nuestro análisis. En el caso de que no existieran elementos suficientes para documentar que una transacción fue valuada bajo el principio de plena competencia, esta transacción no será incluida en este estudio final de precios de transferencia.\\[5pt] 

Se presentará un estudio de precios de transferencia para la empresa SOLICITANTE mencionada anteriormente en la sección transacciones sujetas a análisis. 

\item \textit{\underline{Alcance y uso de la Documentación}}\\[5pt] 

Los reportes, así como cualquier documento resultante de este trabajo, será preparado exclusivamente para asistir a LA SOLICITANTE en el análisis de sus transacciones con partes relacionadas conforme a las secciones IX y XII del artículo 76 de la LISR, vigente para el ejercicio fiscal 2022.\\[5pt] 

Este análisis será realizado considerando los artículos 179 y 180 de la LISR, vigente para el ejercicio fiscal 2022, así como las regulaciones vigentes publicadas en la Miscelánea Fiscal. Los documentos que se obtengan como resultado de este trabajo, no estarán dirigidos a ningún aspecto tributario o comercial como pueden ser Retenciones, Deducibilidad, Acumulación, Clientes o Impuesto al Valor Agregado.\\[5pt] 

Los resultados del análisis presentado en el estudio de precios de transferencia, dependen de la exactitud y fiabilidad de la información proporcionada por LA SOLICITANTE. Las conclusiones del análisis se basarán en la información pública y la información proporcionada por LA SOLICITANTE. Por lo anterior la información se considerará confiable y no será auditada por el equipo de precios de transferencia de la \textcolor{principal}{CP14NL}.\\[5pt]

La \textcolor{principal}{CP14NL} y LA SOLICITANTE aceptan que mantendrán total confidencialidad de los documentos preparados como parte de este trabajo, así como de la información utilizada para este propósito. Los estudios de precios de transferencia entregados como resultado de este trabajo así como todos los documentos relativos, pueden estar sujetos a interpretación de las Autoridades fiscales y no podrán ser utilizados para otros propósitos que no sean documentación de precios de transferencia.\\[5pt] 

En caso de que las transacciones evaluadas en este trabajo sean sujetas a una revisión por parte de las Autoridades fiscales competentes, por petición de LA SOLICITANTE o de las Autoridades Mexicanas, \textcolor{principal}{CP14NL} podría explicar el análisis y las consideraciones técnicas relativas a dicho estudio, limitándose a los contenidos del estudio de precios de transferencia preparado como parte de este trabajo. \\[5pt]

\end{enumerate}
\section{TÉRMINOS Y CONSIDERACIONES}

\begin{enumerate}[A)]
\item \textit{\underline{Honorarios}}\\[5pt]

Nuestros honorarios estimados para el análisis y documentación de las transacciones entre las empresas del grupo realizadas en el ejercicio 2022, son de  \textcolor{principal}{\textbf{\$98,750.00 (Noventa y ocho mil setecientos cincuenta pesos 00/100)}}. Estos honorarios se facturarían de la siguiente manera: 

\begin{enumerate}[1.]
\item 50\% al iniciar el proyecto. 

\item 50\% al emitir los estudios de precios de transferencia. 

\item Estos montos no consideran el Impuesto al Valor Agregado (IVA). Estos impuestos se incluirían en la factura correspondiente. 

\end{enumerate}

Las facturas serán exigibles a partir del momento en que sean recibidas por LA SOLICITANTE. En caso de que sea necesario realizar actividades no contempladas en nuestra estimación de honorarios, se consultará oportunamente con la administración de la compañía para acordar el cambio de alcance y honorarios correspondientes. 

\item \textit{\underline{Otras Consideraciones }}

Nuestro análisis estará basado en la información y registros de LA SOLICITANTE, la cual será proporcionada por funcionarios de la compañía. Por lo tanto, nuestras conclusiones y recomendaciones dependerán de la precisión y confiabilidad de los registros y demás información proporcionada por la compañía para el desarrollo de este proyecto. El análisis y cualquier documentación resultante se prepararán exclusivamente para los efectos indicados en esta propuesta. Ambas partes se comprometen a mantener la confidencialidad de la información y documentación compartida y preparada como parte de este proyecto. \\[5pt]

En caso de que alguna de las transacciones evaluadas y documentadas conforme a los términos de esta propuesta sea sujeto de una revisión de precios de transferencia por parte de la autoridad fiscal correspondiente, a solicitud de la compañía, la \textcolor{principal}{CP14NL} participaría en una junta con la autoridad fiscal y/o con la compañía. En esta reunión se presentaría y explicaría el contenido de la documentación comprobatoria preparada conforme a los términos de esta propuesta. De la misma forma, a solicitud de la compañía, la \textcolor{principal}{CP14NL} entregaría a la autoridad fiscal y/o a la compañía copias de la información y papeles de trabajo que tenga en sus archivos.\\[5pt]

\end{enumerate} 

Atentamente, \\

\textcolor{principal}{``CP14NL''}\\

\noindent
\textcolor{principal}{\textbf{Rodrigo Verastegui}}\\[1pt]
Socio Director\\[5pt]
\noindent
\textcolor{principal}{\textbf{Diego M. Perezcano}}\\[1pt]
Perito Valuador


\end{document}


