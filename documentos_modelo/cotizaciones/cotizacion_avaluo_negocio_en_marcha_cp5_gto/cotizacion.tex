\documentclass[5pt,letter]{report}
\usepackage{import}
\import{../../../sistema/}{rutas}
\input{0.preambulo1}
\usepackage{textcomp}
\decimalpoint
%------------Diseño de página --------------------------------
\usepackage[centering, letterpaper,margin=2cm,top=1cm, headsep=24pt, headheight=2cm,includehead, includefoot]{geometry}

%------------Colores----------------------

\definecolor{principal}{RGB}{0, 53, 73}
\definecolor{secundario}{RGB}{43, 82, 96}
\definecolor{terciario}{RGB}{43,82,96}

%--------------Hipervinculos en color, ligas-azul, archivos-magenta, url-azul----------------
\hypersetup{
    colorlinks=true,
    linkcolor=secundario,
    filecolor=magenta,      
    urlcolor=blue}

%-------------Encabezado---------------------
\pagestyle{fancy}
\fancyhf{}

\chead{\includegraphics[width=8cm]{\rutaImagenes/logo_valuami_fondo_blanco}}

%--------------Pie de página------------------
\cfoot{\textbf{\textcolor{principal}{\footnotesize{VALUAMI\tiny\textregistered}}}\\ \scriptsize{\textit{Amargura 50, Interior 7 y 8, Antigua Granada Parques de la Herradura.\\ Huixquilucan, Estado de M\'exico. CP 52786.\\ Tel. 52 94 76 80 / 55 89 96 34\\ \url{www.ami-mexico.com/valuami}}}}

%------------Títulos de seccion y subseccion--------------



%\renewcommand \thechapter {\Roman{chapter}}
\renewcommand \thesection {\Roman{section}}
%\renewcommand \thesubsection {\thesection.\arabic{subsection}}
%\renewcommand \thesubsubsection {\thesection.\arabic{subsection}.\arabic{subsubsection}}

\titleformat{\section}[hang]{\color{gray}\huge\bfseries}{\thesection.}{1em}{} 

%\chapterfont{\color{principal}}
\sectionfont{\color{principal}}
%\subsectionfont{\color{secundario}}
%\subsubsectionfont{\color{terciario}}

%------------------Profundidad del índice---------------------------

\setcounter{tocdepth}{3}
\setcounter{secnumdepth}{3}




%------------------Marca de agua---------------

\backgroundsetup{angle=0, contents={\includegraphics[width=8cm]{\rutaImagenes/logo_valuami_fondo_blanco}},opacity=.3, scale=1}


\newcommand{\fechaCotizacion}{\today}
\newcommand{\lugarCotizacion}{\inserta}
\newcommand{\empresaSolicitante}{\inserta}
\newcommand{\personaSolicitante}{\inserta}
\newcommand{\caracterSolicitante}{\inserta}
\newcommand{\honorarios}{\inserta}
\newcommand{\honorariosLetra}{\inserta}
\newcommand{\anticipo}{\inserta}
\newcommand{\liquidacion}{\inserta}
\newcommand{\totalMasIva}{\inserta}


\begin{document}

\begin{minipage}{12cm}

León, Guanajuato a 12 de mayo de 2023.

María Martha Arroyo de León
YUKEN Surface Technology S.A.
Presente.

\end{minipage}

León, Gto, 23 de marzo de 2022.

Japan External Trade Consulting, S.C
YUKEN Surface Technology S.A.
Presente.

A lo largo de este documento encontrará la propuesta en respuesta a su solicitud de proyecto de VALUACIÓN; para estimar el valor de liquidación de YUKEN Surface Technology S.A.; considerando:

Propuesta Financiera:

Los detalles específicos de esta propuesta son:

VALUACIÓN DE LA EMPRESA Y SU CAPITAL SOCIAL:

1.- Objetivo y Propósito de la Valuación:

•	El propósito: determinar el Valor de Liquidación de YUKEN Surface Technology S.A.
•	El uso del informe será para fines informativos.

2. Informe de Trabajo:

Se entregará una "Opinión de Valuación" considerando:

•	Antecedentes, Definiciones, Supuestos y hechos previos a la tasación, Enfoques aplicados, Conclusión de Valor Justo, resumen de valores y otros respaldos relevantes.

Nuestros informes de valoración se emitirán de acuerdo a las regulaciones aplicables de peritos públicos.

3.- Requisitos de Información:

Para el adecuado desarrollo de nuestro trabajo, será necesario proporcionarnos la siguiente documentación:


YUKEN Surface Technology S.A.:

• Estados Financieros Históricos (ESF, ER, EFE y EVCC) de 5 años a la fecha, preferentemente con una opinión de auditoría, incluyendo las NOTAS a sus estados financieros.
• Pronóstico de la Declaración de Ingresos.
• Antecedentes corporativos de la empresa.
• Presentación corporativa con las unidades de negocio actuales.
• Entrevista en línea con los responsables de las decisiones financieras.
• Copia de la identificación del representante legal y la documentación que respalde su rol en la empresa.

4.- Propuesta Financiera:

Informe de Valuación (En inglés) firmado por un Perito Certificado Mexicano:

	Valuación de la Empresa_____________________________________________ $ 18,384.00 USD + impuestos
Total: $ 21,325.44 (impuestos incluidos).


5.- Condiciones de Servicio:

Primer pago: Se requiere un anticipo del 50% al confirmar el trabajo y el 50% restante cuando se emita el informe final.

Tiempo de ejecución: Se entregará un informe provisional en 15 días hábiles. El informe completo se entregará en 18-20 días hábiles. El punto de inicio será cuando recibamos en nuestra cuenta la documentación financiera completa y el pago anticipado.

Cuenta Bancaria: Insertar datos bancarios 

Visita de Inspección: Durante el proceso de valuación, será necesario visitar una vez la dirección de la empresa, para realizar el levantamiento fotográfico del inmueble, planta y equipo observable.

6.- Información Adicional:

El propósito de este documento es establecer los detalles de la propuesta sobre la cual la Firma prestaría sus servicios, si se requiere información o documentación adicional, con gusto la proporcionaremos.

Por favor, firme abajo si la propuesta se ajusta a la solicitud. La interpretación y cumplimiento de este documento se basará en la regulación mexicana.

Gracias por su solicitud, esperamos su respuesta,

Atentamente,

Lic. Laura Adriana Pérez Cor
\noindent
\textcolor{principal}{\textbf{Lic. Laura Adriana Pérez Cornejoi}}\\[1pt]
Corredor Público No. 5 del Estado de Guanajuato\\[5pt]
\noindent
\textcolor{principal}{\textbf{Diego M. Perezcano}}\\[1pt]
Perito Valuador


\end{document}


