%\newcommand{\valorCapitalM}{906}
%\newcommand{\valorCapitalm}{356}
%\newcommand{\valorCapitalc}{853}
%\newcommand{\valorCapital}{\valorCapitalM,\valorCapitalm,\valorCapitalc}
%\newcommand{\valorCapitalLetra}{\Numberstringnum{\valorCapitalM}{} millones, \numberstringnum{\valorCapitalm}{} mil, \numberstringnum{\valorCapitalc}}

\subsection{FAIR VALUE OF EQUITY CAPITAL, WITH FIGURES AS OF \fechaValoresCorto}

Once the value of the business firm was estimated, the appraiser proceeded to subtract the Net Debt of the business, thus obtaining the fair value of the equity capital.\\

Next, the fair value of the equity capital (\textit{\gls{equityvalue}}) as of the valuation date is concluded:\\


\begin{figure}[H]
\centering
\textbf{\textcolor{principal}{Equity Capital Value as of \fechaValoresCorto:} \$\valorCapital{} MXN}\\
\includegraphics[width=8cm]{../0.imagenes_eng/valor_cap_acc}\\
(\textcolor{principal}{\valorCapitalLetra{} pesos 00/100 M.N.})


\end{figure}
