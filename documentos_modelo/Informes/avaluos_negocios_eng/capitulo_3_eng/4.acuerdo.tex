\textcolor{principal}{AGREEMENT ESTABLISHING GUIDELINES FOR PUBLIC BROKERS TO ISSUE APPRAISALS ISSUED BY THE SECRETARY OF COMMERCE AND INDUSTRIAL PROMOTION, PUBLISHED IN THE OFFICIAL GAZETTE ON MARCH 9, 1999.}\\[10pt]



\textit{``Article 2.- The valuation report issued by the public broker shall be composed of the following sections:}

\begin{enumerate}[I.]

\item Background;
\item Data of the asset or service subject to valuation; 
\item Legal basis and preliminary considerations; 
\item  Methodology employed;
\item  Development of the appraisal, and
\item  Conclusions.

\end{enumerate}

\textit{Furthermore, in all cases, the public broker must have complete documentary support regarding the market study conducted for the purposes of the appraisal. (...)}\\[10pt]

\textit{Article 12.- In valuations performed by public brokers for intangible assets, considering their nature or type, their values may be determined as follows:}

\begin{enumerate}[I.-]
\item By researching the market for similar or substitute goods and products based on commercial references, implied and calculated values, considering sales volumes and profitability, possible purchase and sale cases, or alternatively, royalty payments for the use and exploitation of patents, trademarks, or franchises;

\item In the case of projects, an analysis will be conducted of the infrastructure of services available, marketing characteristics, technology used, price determination, investment costs, loss of profit, financial performance, and profit margins, in order to diagnose their investment margins, cash flows, and break-even points;

\item Through the study of the best utilization of projects and the commercial value of real or potential gross rents generated, as well as calculating the equivalent capital capable of providing those rents under non-inflationary and low-risk conditions, considering whether it is a project valuation or an ongoing business, or

\item When it comes to transfer pricing, it shall be done by applying the procedures established in the Income Tax Law or, alternatively, using the appropriate method for the case. (...)''

\end{enumerate}