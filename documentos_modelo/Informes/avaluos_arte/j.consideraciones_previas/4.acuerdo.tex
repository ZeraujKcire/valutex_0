
\textcolor{principal}{ACUERDO QUE ESTABLECE LOS LINEAMIENTOS A SEGUIR POR LOS CORREDORES P\'UBLICOS PARA EMITIR AVAL\'UOS EMITIDO POR LA SECRETAR\'IA DE COMERCIO Y FOMENTO INDUSTRIAL, EN EL DIARIO OFICIAL DE FECHA 9 DE MARZO DE 1999.}\\[10pt]

\textit{``Art. 2.- El dictamen valuatorio que el corredor p\'ublico emita deber\'a estar integrado por las secciones siguientes}:

\begin{enumerate}[I.]
\item Antecedentes;
\item Datos del bien o servicio sujeto a valuaci\'on;
\item Fundamento jur\'idico y consideraciones previas;
\item Metodolog\'ia empleada;
\item Desarrollo del aval\'uo, y
\item Conclusiones.
\end{enumerate}


\textit{Adem\'as, en todo caso, el corredor p\'ublico deber\'a tener el soporte documental completo acerca del estudio de mercado realizado para efectos del aval\'uo. (...)}\\[10pt]

\textit{Art. 12.- En los aval\'uos practicados por corredor p\'ublico a bienes intangibles, en atenci\'on a su naturaleza o tipo, se podr\'an determinar sus valores de acuerdo a lo siguiente:}

\begin{enumerate}[I.-]
\item Mediante la investigaci\'on de mercado de bienes y productos similares o suced\'aneos con base a referencias comerciales, valores impl\'icitos y calculados, considerando vol\'umenes de venta y rentabilidad, posibles casos de compraventa o, en su defecto, pago de regal\'ias por el uso y explotaci\'on de patentes, marcas o franquicias;


\item En el caso de proyectos, se analizar\'a la infraestructura de servicios con que cuenta, caracter\'isticas de comercializaci\'on, tecnolog\'ia ocupada, fijaci\'on de precios, costos de inversi\'on, lucro cesante, comportamiento financiero y m\'argenes de utilidad, para con ello realizar un diagn\'ostico de sus m\'argenes de inversi\'on, flujos de caja y puntos de equilibrio;

\item A trav\'es del estudio del mejor aprovechamiento de los proyectos y el valor comercial de las rentas brutas reales o potenciales que genera, as\'i como calcular el capital equivalente capaz de proveer esas rentas en condiciones no inflacionarias y de bajo riesgo, considerando si se est\'a en presencia de una valuaci\'on de proyecto o un negocio en marcha, o

\item Cuando se trate de la fijaci\'on de precios de transferencia, se har\'a mediante la aplicaci\'on de los procedimientos establecidos en la Ley del Impuesto sobre la Renta o, en su defecto, utilizar\'a el m\'etodo apropiado al caso. (...)''
\end{enumerate}

