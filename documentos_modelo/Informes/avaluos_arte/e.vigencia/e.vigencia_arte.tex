%La vigencia del presente informe es de \textcolor{terciario}{\vigenciaInforme{} calendario}.\ifthenelse{\equal{\notaVigencia}{si}}{\footnote{A falta de disposici\'on expresa sobre la vigencia de este tipo de dict\'amenes de valuaci\'on, se tom\'o el plazo de 1 a\~no mencionado en el Art. 3 del Reglamento del C\'odigo Fiscal de la Federaci\'on.}}{ }\\[5pt]
%
%\textcolor{secundario}{Vigencia extr\'inseca o administrativa. }La vigencia de un aval\'uo est\'a determinada por su prop\'osito o destino y depender\'a de la temporabilidad que establezca en su caso la autoridad competente o instituci\'on administrativa que haga uso de dicho informe. \\[10pt]
%\textcolor{secundario}{Vigencia intr\'inseca. } Un informe conservar\'a su vigencia hasta en tanto no cambien de manera sustancial las condiciones y premisas fundamentales que dieron sustento al c\'alculo (\textit{ceteris paribus}); de tal manera que pudiera afectarse la fiabilidad de las cifras conclusivas de la estimaci\'on de valor.

El presente aval\'uo tiene vigencia de \textcolor{principal}{seis meses} a partir de la fecha de su emisi\'on, independientemente de las fechas de inicio y conclusi\'on del dictamen, en tanto no var\'ien las condiciones propias del Mercado o de la informaci\'on que el solicitante present\'o al suscrito para la elaboración del dictamen encomendado.

