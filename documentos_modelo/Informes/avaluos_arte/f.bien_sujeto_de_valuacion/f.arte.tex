\begin{enumerate}[1.]

\item Se tuvieron a la vista las obras artísticas objeto de valuaci\'on, mediante visita de inspecci\'on.
Adicionalmente, el solicitante entreg\'o al suscrito un reporte fotogr\'afico.
\item  El poseedor de las obras manifest\'o que la autor\'ia de las mismas se atribuye a (*)
\item \insertar cuenta con antecedentes sobre el origen de las obras.
\item \insertar cuenta con certificado de autenticidad de las obras.
\item \insertar cuenta con t\'itulo de propiedad que ampare la titularidad de la obra.

\end{enumerate}

\begin{table}[H]
\centering
\begin{tabular}{|c|m{.4\textwidth}|}
\hline
Obra& Dibujo	\\
\hline
Autor & Se atribuye a Ra\'ul Gonzalez Esquivel\\
\hline
T\'itulo & ``Figura S\'edente, Cultura Totonaca, Veracruz, Cl\'asico Tard\'io, Septiembre 29 de 1993'' \\
\hline
Tem\'atica & Representaci\'on de arte precolombino, im\'agenes de deidades, figuras antropomorfas, cabezas colosales, culto religioso ind\'igena y personajes mitol\'ogicos.\\
\hline
T\'ecnicas &Grafito sobre papel \\
\hline
Medidas & 41.5 cm $\times$ 59 cm \\
\hline
 Formato & Peque\~no \\
 \hline
 Firma & Se atribuye a Ra\'ul Gonzalez Esquivel \\
 \hline
 Fecha & Septiembre 29 de 1993. \\
 \hline
 Provenance & Colecci\'on particular\\
 \hline
 Certificado de autenticidad& Firmado por Pablo C. Goebel el 10 de abril de 1996.  \\
 \hline 
  Materiales empleados & Lapiz sobre papel\\
 \hline
 Caracter\'isticas especiales & Forma parte de una colecci\'on privada de 39 dibujos. 
Contiene dibujo con el texto ``Bufete Campos'' que corresponde a la obra por encargo. \\
\hline 
Ubicaci\'on & Calle Constituci\'on 124, colonia Centro, Veracruz, Veracruz, C.P. 91700.\\
\hline
\end{tabular}
\end{table}

\begin{table}[H]
\centering
\begin{tabular}{|c|m{.4\textwidth}|}
\hline
Obra& Dibujo	\\
\hline
Autor & Se atribuye a Ra\'ul Gonzalez Esquivel\\
\hline
T\'itulo & ``Dios Viejo (Remojadas, Estado de Veracruz). Cultura Totonaca. Septiembre 12 de 1993'' \\
\hline
Tem\'atica & Representaci\'on de arte precolombino, im\'agenes de deidades, figuras antropomorfas, cabezas colosales, culto religioso ind\'igena y personajes mitol\'ogicos.\\
\hline
T\'ecnicas &Grafito sobre papel \\
\hline
Medidas & 41.5 cm $\times$ 59 cm \\
\hline
 Formato & Peque\~no \\
 \hline
 Firma & Se atribuye a Ra\'ul Gonzalez Esquivel \\
 \hline
  Fecha & Septiembre 12 de 1993.\\
 \hline
 Provenance & Colecci\'on particular\\
 \hline
 Certificado de autenticidad& Firmado por Pablo C. Goebel el 10 de abril de 1996.  \\
 \hline 
  Materiales empleados & Lapiz sobre papel\\
 \hline
 Caracter\'isticas especiales & Forma parte de una colecci\'on privada de 39 dibujos. 
Contiene dibujo con el texto ``Bufete Campos'' que corresponde a la obra por encargo. \\
\hline 
Ubicaci\'on & Calle Constituci\'on 124, colonia Centro, Veracruz, Veracruz, C.P. 91700.\\
\hline

\end{tabular}
\end{table}

\begin{table}[H]
\centering
\begin{tabular}{|c|m{.4\textwidth}|}
\hline
Obra& Dibujo	\\
\hline
Autor & Se atribuye a Ra\'ul Gonzalez Esquivel\\
\hline
T\'itulo & ``Representaci\'on de Quetzalcoatl, Taj\'in, Ve. Cultura Totonaca. Septiembre 20 de 1993'.'' \\
\hline
Tem\'atica & Representaci\'on de arte precolombino, im\'agenes de deidades, figuras antropomorfas, cabezas colosales, culto religioso ind\'igena y personajes mitol\'ogicos.\\
\hline
T\'ecnicas &Grafito sobre papel \\
\hline
Medidas & 41.5 cm $\times$ 59 cm \\
\hline
 Formato & Peque\~no \\
 \hline
 Firma & Se atribuye a Ra\'ul Gonzalez Esquivel  \\
 \hline
 Fecha & Septiembre 20 de 1993. \\
 \hline
 Provenance & Colecci\'on particular\\
 \hline
 Certificado de autenticidad& Firmado por Pablo C. Goebel el 10 de abril de 1996.  \\
 \hline 
  Materiales empleados & Lapiz sobre papel\\
 \hline
 Caracter\'isticas especiales & Forma parte de una colecci\'on privada de 39 dibujos. 
Contiene dibujo con el texto ``Bufete Campos'' que corresponde a la obra por encargo. \\
\hline 
Ubicaci\'on & Calle Constituci\'on 124, colonia Centro, Veracruz, Veracruz, C.P. 91700.\\
\hline

\end{tabular}
\end{table}

\begin{table}[H]
\centering
\begin{tabular}{|c|m{.4\textwidth}|}
\hline
Obra& Dibujo	\\
\hline
Autor & Se atribuye a Ra\'ul Gonzalez Esquivel\\
\hline
T\'itulo & ``Nichos del edificio no. 16 del Taj\'in. El Taj\'in Veracruz. Septiembre 5 de 1993.''\\
\hline
Tem\'atica & Representaci\'on de arte precolombino, im\'agenes de deidades, figuras antropomorfas, cabezas colosales, culto religioso ind\'igena y personajes mitol\'ogicos.\\
\hline
T\'ecnicas &Grafito sobre papel \\
\hline
Medidas & 41.5 cm $\times$ 59 cm \\
\hline
 Formato & Peque\~no \\
 \hline
 Firma & Se atribuye a Ra\'ul Gonzalez Esquivel  \\
 \hline
  Fecha & Septiembre 5 de 1993.\\
 \hline
 Provenance & Colecci\'on particular\\
 \hline
 Certificado de autenticidad& Firmado por Pablo C. Goebel el 10 de abril de 1996.  \\
 \hline 
  Materiales empleados & Lapiz sobre papel\\
 \hline
 Caracter\'isticas especiales & Forma parte de una colecci\'on privada de 39 dibujos. 
Contiene dibujo con el texto ``Bufete Campos'' que corresponde a la obra por encargo. \\
\hline 
Ubicaci\'on & Calle Constituci\'on 124, colonia Centro, Veracruz, Veracruz, C.P. 91700.\\
\hline

\end{tabular}
\end{table}

\begin{table}[H]
\centering
\begin{tabular}{|c|m{.4\textwidth}|}
\hline
Obra& Dibujo	\\
\hline
Autor & Se atribuye a Ra\'ul Gonzalez Esquivel\\
\hline
T\'itulo & ``Cabeza de Anciano. Tres Zapotes, La Venta, Veracruz. Septiembre 11 de 1993.'' \\
\hline
Tem\'atica & Representaci\'on de arte precolombino, im\'agenes de deidades, figuras antropomorfas, cabezas colosales, culto religioso ind\'igena y personajes mitol\'ogicos.\\
\hline
T\'ecnicas &Grafito sobre papel \\
\hline
Medidas & 41.5 cm $\times$ 59 cm \\
\hline
 Formato & Peque\~no \\
 \hline
 Firma & Se atribuye a Ra\'ul Gonzalez Esquivel \\
 \hline
  Fecha & Septiembre 11 de 1993.\\
 \hline
 Provenance & Colecci\'on particular\\
 \hline
 Certificado de autenticidad& Firmado por Pablo C. Goebel el 10 de abril de 1996.  \\
 \hline 
  Materiales empleados & Lapiz sobre papel\\
 \hline
 Caracter\'isticas especiales & Forma parte de una colecci\'on privada de 39 dibujos. 
Contiene dibujo con el texto ``Bufete Campos'' que corresponde a la obra por encargo. \\
\hline 
Ubicaci\'on & Calle Constituci\'on 124, colonia Centro, Veracruz, Veracruz, C.P. 91700.\\
\hline

\end{tabular}
\end{table}

\begin{table}[H]
\centering
\begin{tabular}{|c|m{.4\textwidth}|}
\hline
Obra& Dibujo	\\
\hline
Autor & Se atribuye a Ra\'ul Gonzalez Esquivel\\
\hline
T\'itulo & `Èscultura Masculina S\'edente. El Zapotal, Mpio. de Ignacio de la Llave. S. VI-IX. Agosto 23 de 1993.'' \\
\hline
Tem\'atica & Representaci\'on de arte precolombino, im\'agenes de deidades, figuras antropomorfas, cabezas colosales, culto religioso ind\'igena y personajes mitol\'ogicos.\\
\hline
T\'ecnicas &Grafito sobre papel \\
\hline
Medidas & 41.5 cm $\times$ 59 cm \\
\hline
 Formato & Peque\~no \\
 \hline
 Firma &  Se atribuye a Ra\'ul Gonzalez Esquivel\\
 \hline
  Fecha & Agosto 23 de 1993. \\
 \hline
 Provenance & Colecci\'on particular\\
 \hline
 Certificado de autenticidad& Firmado por Pablo C. Goebel el 10 de abril de 1996.  \\
 \hline 
  Materiales empleados & Lapiz sobre papel\\
 \hline
 Caracter\'isticas especiales & Forma parte de una colecci\'on privada de 39 dibujos. 
Contiene dibujo con el texto ``Bufete Campos'' que corresponde a la obra por encargo. \\
\hline 
Ubicaci\'on & Calle Constituci\'on 124, colonia Centro, Veracruz, Veracruz, C.P. 91700.\\
\hline

\end{tabular}
\end{table}

\begin{table}[H]
\centering
\begin{tabular}{|c|m{.4\textwidth}|}
\hline
Obra& Dibujo	\\
\hline
Autor & Se atribuye a Ra\'ul Gonzalez Esquivel\\
\hline
T\'itulo & ``Mensajero del Sol. El Zapotal, Mpio. de Ignacio de la Llave. Cl\'asico Tard\'io. S. VI-IX D.C. Septiembre 03 de 1993.''\\
\hline
Tem\'atica & Representaci\'on de arte precolombino, im\'agenes de deidades, figuras antropomorfas, cabezas colosales, culto religioso ind\'igena y personajes mitol\'ogicos.\\
\hline
T\'ecnicas &Grafito sobre papel \\
\hline
Medidas & 41.5 cm $\times$ 59 cm \\
\hline
 Formato & Peque\~no \\
 \hline
 Firma & Se atribuye a Ra\'ul Gonzalez Esquivel \\
 \hline
  Fecha & Septiembre 3 de 1993.\\
 \hline
 Provenance & Colecci\'on particular\\
 \hline
 Certificado de autenticidad& Firmado por Pablo C. Goebel el 10 de abril de 1996.  \\
 \hline 
  Materiales empleados & Lapiz sobre papel\\
 \hline
 Caracter\'isticas especiales & Forma parte de una colecci\'on privada de 39 dibujos. 
Contiene dibujo con el texto ``Bufete Campos'' que corresponde a la obra por encargo. \\
\hline 
Ubicaci\'on & Calle Constituci\'on 124, colonia Centro, Veracruz, Veracruz, C.P. 91700.\\
\hline

\end{tabular}
\end{table}

\begin{table}[H]
\centering
\begin{tabular}{|c|m{.4\textwidth}|}
\hline
Obra& Dibujo	\\
\hline
Autor & Se atribuye a Ra\'ul Gonzalez Esquivel\\
\hline
T\'itulo & ``Sin t\'itulo, del Arq. Gonz\'alea Esquivel, aunque si muestra padre con hijo en sus brazos, o bien personaje con persona menor en brazos. Agosto 14 de 1993'' \\
\hline
Tem\'atica & Representaci\'on de arte precolombino, im\'agenes de deidades, figuras antropomorfas, cabezas colosales, culto religioso ind\'igena y personajes mitol\'ogicos.\\
\hline
T\'ecnicas &Grafito sobre papel \\
\hline
Medidas & 41.5 cm $\times$ 59 cm \\
\hline
 Formato & Peque\~no \\
 \hline
 Firma & Se atribuye a Ra\'ul Gonzalez Esquivel\\ 
 \hline
  Fecha &  Agosto 14 de 1993\\
 \hline
 Provenance & Colecci\'on particular\\
 \hline
 Certificado de autenticidad& Firmado por Pablo C. Goebel el 10 de abril de 1996.  \\
 \hline 
  Materiales empleados & Lapiz sobre papel\\
 \hline
 Caracter\'isticas especiales & Forma parte de una colecci\'on privada de 39 dibujos. 
Contiene dibujo con el texto ``Bufete Campos'' que corresponde a la obra por encargo. \\
\hline 
Ubicaci\'on & Calle Constituci\'on 124, colonia Centro, Veracruz, Veracruz, C.P. 91700.\\
\hline

\end{tabular}
\end{table}

\begin{table}[H]
\centering
\begin{tabular}{|c|m{.4\textwidth}|}
\hline
Obra& Dibujo	\\
\hline
Autor & Se atribuye a Ra\'ul Gonzalez Esquivel\\
\hline
T\'itulo & ``Dos cabezas, hombre y mujer, paga, Agosto 22 de 1993''\\
\hline
Tem\'atica & Representaci\'on de arte precolombino, im\'agenes de deidades, figuras antropomorfas, cabezas colosales, culto religioso ind\'igena y personajes mitol\'ogicos.\\
\hline
T\'ecnicas &Grafito sobre papel \\
\hline
Medidas & 41.5 cm $\times$ 59 cm \\
\hline
 Formato & Peque\~no \\
 \hline
 Firma & Se atribuye a Ra\'ul Gonzalez Esquivel\\ 
 \hline
  Fecha & Agosto 22 de 1993\\
 \hline
 Provenance & Colecci\'on particular\\
 \hline
 Certificado de autenticidad& Firmado por Pablo C. Goebel el 10 de abril de 1996.  \\
 \hline 
  Materiales empleados & Lapiz sobre papel\\
 \hline
 Caracter\'isticas especiales & Forma parte de una colecci\'on privada de 39 dibujos. 
Contiene dibujo con el texto ``Bufete Campos'' que corresponde a la obra por encargo. \\
\hline 
Ubicaci\'on & Calle Constituci\'on 124, colonia Centro, Veracruz, Veracruz, C.P. 91700.\\
\hline

\end{tabular}
\end{table}

\begin{table}[H]
\centering
\begin{tabular}{|c|m{.4\textwidth}|}
\hline
Obra& Dibujo	\\
\hline
Autor & Se atribuye a Ra\'ul Gonzalez Esquivel\\
\hline
T\'itulo & ``Tlalsolteotl, protectora de los partos y proporcionadora de los amores il\'icitos y las pasiones. Agosto de 1993.''\\
\hline
Tem\'atica & Representaci\'on de arte precolombino, im\'agenes de deidades, figuras antropomorfas, cabezas colosales, culto religioso ind\'igena y personajes mitol\'ogicos.\\
\hline
T\'ecnicas &Grafito sobre papel \\
\hline
Medidas & 41.5 cm $\times$ 59 cm \\
\hline
 Formato & Peque\~no \\
 \hline
 Firma & Se atribuye a Ra\'ul Gonzalez Esquivel\\ 
 \hline
  Fecha & Agosto de 1993.\\
 \hline
 Provenance & Colecci\'on particular\\
 \hline
 Certificado de autenticidad& Firmado por Pablo C. Goebel el 10 de abril de 1996.  \\
 \hline 
  Materiales empleados & Lapiz sobre papel\\
 \hline
 Caracter\'isticas especiales & Forma parte de una colecci\'on privada de 39 dibujos. 
Contiene dibujo con el texto ``Bufete Campos'' que corresponde a la obra por encargo. \\
\hline 
Ubicaci\'on & Calle Constituci\'on 124, colonia Centro, Veracruz, Veracruz, C.P. 91700.\\
\hline

\end{tabular}
\end{table}

\begin{table}[H]
\centering
\begin{tabular}{|c|m{.4\textwidth}|}
\hline
Obra& Dibujo	\\
\hline
Autor & Se atribuye a Ra\'ul Gonzalez Esquivel\\
\hline
T\'itulo & ``Dios Taj\'in, Dios de la lluvia y del trueno, edificio no. 5. Taj\'in Papantla, Veracruz. Septiembre de 1993.''\\
\hline
Tem\'atica & Representaci\'on de arte precolombino, im\'agenes de deidades, figuras antropomorfas, cabezas colosales, culto religioso ind\'igena y personajes mitol\'ogicos.\\
\hline
T\'ecnicas &Grafito sobre papel \\
\hline
Medidas & 41.5 cm $\times$ 59 cm \\
\hline
 Formato & Peque\~no \\
 \hline
 Firma & Se atribuye a Ra\'ul Gonzalez Esquivel\\ 
 \hline
  Fecha & Septiembre de 1993.\\
 \hline
 Provenance & Colecci\'on particular\\
 \hline
 Certificado de autenticidad& Firmado por Pablo C. Goebel el 10 de abril de 1996.  \\
 \hline 
  Materiales empleados & Lapiz sobre papel\\
 \hline
 Caracter\'isticas especiales & Forma parte de una colecci\'on privada de 39 dibujos. 
Contiene dibujo con el texto ``Bufete Campos'' que corresponde a la obra por encargo. \\
\hline 
Ubicaci\'on & Calle Constituci\'on 124, colonia Centro, Veracruz, Veracruz, C.P. 91700.\\
\hline

\end{tabular}
\end{table}

\begin{table}[H]
\centering
\begin{tabular}{|c|m{.4\textwidth}|}
\hline
Obra& Dibujo	\\
\hline
Autor & Se atribuye a Ra\'ul Gonzalez Esquivel\\
\hline
T\'itulo &``Carita SOnriente de los cerros. Cultura Totonaca (Veracruz). Mayo 15 de 1993.'' \\
\hline
Tem\'atica & Representaci\'on de arte precolombino, im\'agenes de deidades, figuras antropomorfas, cabezas colosales, culto religioso ind\'igena y personajes mitol\'ogicos.\\
\hline
T\'ecnicas &Grafito sobre papel \\
\hline
Medidas & 41.5 cm $\times$ 59 cm \\
\hline
 Formato & Peque\~no \\
 \hline
 Firma & Se atribuye a Ra\'ul Gonzalez Esquivel\\ 
 \hline
  Fecha & Mayo 15 de 1993.\\
 \hline
 Provenance & Colecci\'on particular\\
 \hline
 Certificado de autenticidad& Firmado por Pablo C. Goebel el 10 de abril de 1996.  \\
 \hline 
  Materiales empleados & Lapiz sobre papel\\
 \hline
 Caracter\'isticas especiales & Forma parte de una colecci\'on privada de 39 dibujos. 
Contiene dibujo con el texto ``Bufete Campos'' que corresponde a la obra por encargo. \\
\hline 
Ubicaci\'on & Calle Constituci\'on 124, colonia Centro, Veracruz, Veracruz, C.P. 91700.\\
\hline

\end{tabular}
\end{table}

\begin{table}[H]
\centering
\begin{tabular}{|c|m{.4\textwidth}|}
\hline
Obra& Dibujo	\\
\hline
Autor & Se atribuye a Ra\'ul Gonzalez Esquivel\\
\hline
T\'itulo & ``Carita Sonriente. Cultura Totonaca. Julio 19 de 1993.'' \\
\hline
Tem\'atica & Representaci\'on de arte precolombino, im\'agenes de deidades, figuras antropomorfas, cabezas colosales, culto religioso ind\'igena y personajes mitol\'ogicos.\\
\hline
T\'ecnicas &Grafito sobre papel \\
\hline
Medidas & 41.5 cm $\times$ 59 cm \\
\hline
 Formato & Peque\~no \\
 \hline
 Firma & Se atribuye a Ra\'ul Gonzalez Esquivel\\ 
 \hline
  Fecha & Julio 19 de 1993.\\
 \hline
 Provenance & Colecci\'on particular\\
 \hline
 Certificado de autenticidad& Firmado por Pablo C. Goebel el 10 de abril de 1996.  \\
 \hline 
  Materiales empleados & Lapiz sobre papel\\
 \hline
 Caracter\'isticas especiales & Forma parte de una colecci\'on privada de 39 dibujos. 
Contiene dibujo con el texto ``Bufete Campos'' que corresponde a la obra por encargo. \\
\hline 
Ubicaci\'on & Calle Constituci\'on 124, colonia Centro, Veracruz, Veracruz, C.P. 91700.\\
\hline

\end{tabular}
\end{table}

\begin{table}[H]
\centering
\begin{tabular}{|c|m{.4\textwidth}|}
\hline
Obra& Dibujo	\\
\hline
Autor & Se atribuye a Ra\'ul Gonzalez Esquivel\\
\hline
T\'itulo & `Hacha Votiva con Mono, Centro de Veracruz Post-Cl\'asico Tard\'io S. VI-IX D.C.''\\
\hline
Tem\'atica & Representaci\'on de arte precolombino, im\'agenes de deidades, figuras antropomorfas, cabezas colosales, culto religioso ind\'igena y personajes mitol\'ogicos.\\
\hline
T\'ecnicas &Grafito sobre papel \\
\hline
Medidas & 41.5 cm $\times$ 59 cm \\
\hline
 Formato & Peque\~no \\
 \hline
 Firma & Se atribuye a Ra\'ul Gonzalez Esquivel\\ 
 \hline
  Fecha & 6 de diciembre de 1993.\\
 \hline
 Provenance & Colecci\'on particular\\
 \hline
 Certificado de autenticidad& Firmado por Pablo C. Goebel el 10 de abril de 1996.  \\
 \hline 
  Materiales empleados & Lapiz sobre papel\\
 \hline
 Caracter\'isticas especiales & Forma parte de una colecci\'on privada de 39 dibujos. 
Contiene dibujo con el texto ``Bufete Campos'' que corresponde a la obra por encargo. \\
\hline 
Ubicaci\'on & Calle Constituci\'on 124, colonia Centro, Veracruz, Veracruz, C.P. 91700.\\
\hline

\end{tabular}
\end{table}

\begin{table}[H]
\centering
\begin{tabular}{|c|m{.4\textwidth}|}
\hline
Obra& Dibujo	\\
\hline
Autor & Se atribuye a Ra\'ul Gonzalez Esquivel\\
\hline
T\'itulo & ``Carita Sonriente de los Cerros. Cultura Totonaca (Veracruz). Mayo 8 de 1993.''\\
\hline
Tem\'atica & Representaci\'on de arte precolombino, im\'agenes de deidades, figuras antropomorfas, cabezas colosales, culto religioso ind\'igena y personajes mitol\'ogicos.\\
\hline
T\'ecnicas &Grafito sobre papel \\
\hline
Medidas & 41.5 cm $\times$ 59 cm \\
\hline
 Formato & Peque\~no \\
 \hline
 Firma & Se atribuye a Ra\'ul Gonzalez Esquivel\\ 
 \hline
  Fecha & Mayo 8 de 1993.\\
 \hline
 Provenance & Colecci\'on particular\\
 \hline
 Certificado de autenticidad& Firmado por Pablo C. Goebel el 10 de abril de 1996.  \\
 \hline 
  Materiales empleados & Lapiz sobre papel\\
 \hline
 Caracter\'isticas especiales & Forma parte de una colecci\'on privada de 39 dibujos. 
Contiene dibujo con el texto ``Bufete Campos'' que corresponde a la obra por encargo. \\
\hline 
Ubicaci\'on & Calle Constituci\'on 124, colonia Centro, Veracruz, Veracruz, C.P. 91700.\\
\hline

\end{tabular}
\end{table}

\begin{table}[H]
\centering
\begin{tabular}{|c|m{.4\textwidth}|}
\hline
Obra& Dibujo	\\
\hline
Autor & Se atribuye a Ra\'ul Gonzalez Esquivel\\
\hline
T\'itulo & ``Conjunto escult\'orico. Sacerdotes Portadores de un Incensario. Cultura Totonaca. Centro de Veracruz (Mixtequilla) Cl\'asico Tard\'io, S. VI-IX D.C.''\\
\hline
Tem\'atica & Representaci\'on de arte precolombino, im\'agenes de deidades, figuras antropomorfas, cabezas colosales, culto religioso ind\'igena y personajes mitol\'ogicos.\\
\hline
T\'ecnicas &Grafito sobre papel \\
\hline
Medidas & 41.5 cm $\times$ 59 cm \\
\hline
 Formato & Peque\~no \\
 \hline
 Firma & Se atribuye a Ra\'ul Gonzalez Esquivel\\ 
 \hline
  Fecha & \\
 \hline
 Provenance & Colecci\'on particular\\
 \hline
 Certificado de autenticidad& Firmado por Pablo C. Goebel el 10 de abril de 1996.  \\
 \hline 
  Materiales empleados & Lapiz sobre papel\\
 \hline
 Caracter\'isticas especiales & Forma parte de una colecci\'on privada de 39 dibujos. 
Contiene dibujo con el texto ``Bufete Campos'' que corresponde a la obra por encargo. \\
\hline 
Ubicaci\'on & Calle Constituci\'on 124, colonia Centro, Veracruz, Veracruz, C.P. 91700.\\
\hline

\end{tabular}
\end{table}

\begin{table}[H]
\centering
\begin{tabular}{|c|m{.4\textwidth}|}
\hline
Obra& Dibujo	\\
\hline
Autor & Se atribuye a Ra\'ul Gonzalez Esquivel\\
\hline
T\'itulo & ``Figura Sonriente. Pintada y articulada. El Zapotal. Mpio. Ignacio de la Llave, Cl\'asico Tard\'io S. VI-IX D.C.''\\
\hline
Tem\'atica & Representaci\'on de arte precolombino, im\'agenes de deidades, figuras antropomorfas, cabezas colosales, culto religioso ind\'igena y personajes mitol\'ogicos.\\
\hline
T\'ecnicas &Grafito sobre papel \\
\hline
Medidas & 41.5 cm $\times$ 59 cm \\
\hline
 Formato & Peque\~no \\
 \hline
 Firma & Se atribuye a Ra\'ul Gonzalez Esquivel\\ 
 \hline
  Fecha & \\
 \hline
 Provenance & Colecci\'on particular\\
 \hline
 Certificado de autenticidad& Firmado por Pablo C. Goebel el 10 de abril de 1996.  \\
 \hline 
  Materiales empleados & Lapiz sobre papel\\
 \hline
 Caracter\'isticas especiales & Forma parte de una colecci\'on privada de 39 dibujos. 
Contiene dibujo con el texto ``Bufete Campos'' que corresponde a la obra por encargo. \\
\hline 
Ubicaci\'on & Calle Constituci\'on 124, colonia Centro, Veracruz, Veracruz, C.P. 91700.\\
\hline

\end{tabular}
\end{table}

\begin{table}[H]
\centering
\begin{tabular}{|c|m{.4\textwidth}|}
\hline
Obra& Dibujo	\\
\hline
Autor & Se atribuye a Ra\'ul Gonzalez Esquivel\\
\hline
T\'itulo &``Carita Sonriente. Cultura Totonaca. Los Cerros, Veracruz. Cl\'asico Tard\'io S. VI-IX D.C. Octubre 8 de 1993.'' \\
\hline
Tem\'atica & Representaci\'on de arte precolombino, im\'agenes de deidades, figuras antropomorfas, cabezas colosales, culto religioso ind\'igena y personajes mitol\'ogicos.\\
\hline
T\'ecnicas &Grafito sobre papel \\
\hline
Medidas & 41.5 cm $\times$ 59 cm \\
\hline
 Formato & Peque\~no \\
 \hline
 Firma & Se atribuye a Ra\'ul Gonzalez Esquivel\\ 
 \hline
  Fecha & Octubre 8 de 1993.\\
 \hline
 Provenance & Colecci\'on particular\\
 \hline
 Certificado de autenticidad& Firmado por Pablo C. Goebel el 10 de abril de 1996.  \\
 \hline 
  Materiales empleados & Lapiz sobre papel\\
 \hline
 Caracter\'isticas especiales & Forma parte de una colecci\'on privada de 39 dibujos. 
Contiene dibujo con el texto ``Bufete Campos'' que corresponde a la obra por encargo. \\
\hline 
Ubicaci\'on & Calle Constituci\'on 124, colonia Centro, Veracruz, Veracruz, C.P. 91700.\\
\hline

\end{tabular}
\end{table}

\begin{table}[H]
\centering
\begin{tabular}{|c|m{.4\textwidth}|}
\hline
Obra& Dibujo	\\
\hline
Autor & Se atribuye a Ra\'ul Gonzalez Esquivel\\
\hline
T\'itulo & ``Figura Danzante. Cultura Totonaca, Veracruz. Octubre 08 de 1993.''\\
\hline
Tem\'atica & Representaci\'on de arte precolombino, im\'agenes de deidades, figuras antropomorfas, cabezas colosales, culto religioso ind\'igena y personajes mitol\'ogicos.\\
\hline
T\'ecnicas &Grafito sobre papel \\
\hline
Medidas & 41.5 cm $\times$ 59 cm \\
\hline
 Formato & Peque\~no \\
 \hline
 Firma & Se atribuye a Ra\'ul Gonzalez Esquivel\\ 
 \hline
  Fecha & Octubre 8 de 1993.\\
 \hline
 Provenance & Colecci\'on particular\\
 \hline
 Certificado de autenticidad& Firmado por Pablo C. Goebel el 10 de abril de 1996.  \\
 \hline 
  Materiales empleados & Lapiz sobre papel\\
 \hline
 Caracter\'isticas especiales & Forma parte de una colecci\'on privada de 39 dibujos. 
Contiene dibujo con el texto ``Bufete Campos'' que corresponde a la obra por encargo. \\
\hline 
Ubicaci\'on & Calle Constituci\'on 124, colonia Centro, Veracruz, Veracruz, C.P. 91700.\\
\hline

\end{tabular}
\end{table}

\begin{table}[H]
\centering
\begin{tabular}{|c|m{.4\textwidth}|}
\hline
Obra& Dibujo	\\
\hline
Autor & Se atribuye a Ra\'ul Gonzalez Esquivel\\
\hline
T\'itulo &``Carita Sonriente. Cultura Totonaca S. VI-XI D.C. Los Cerros, Veracruz. Octubre 08 de 1993.'' \\
\hline
Tem\'atica & Representaci\'on de arte precolombino, im\'agenes de deidades, figuras antropomorfas, cabezas colosales, culto religioso ind\'igena y personajes mitol\'ogicos.\\
\hline
T\'ecnicas &Grafito sobre papel \\
\hline
Medidas & 41.5 cm $\times$ 59 cm \\
\hline
 Formato & Peque\~no \\
 \hline
 Firma & Se atribuye a Ra\'ul Gonzalez Esquivel\\ 
 \hline
  Fecha & Octubre 8 de 1993.\\
 \hline
 Provenance & Colecci\'on particular\\
 \hline
 Certificado de autenticidad& Firmado por Pablo C. Goebel el 10 de abril de 1996.  \\
 \hline 
  Materiales empleados & Lapiz sobre papel\\
 \hline
 Caracter\'isticas especiales & Forma parte de una colecci\'on privada de 39 dibujos. 
Contiene dibujo con el texto ``Bufete Campos'' que corresponde a la obra por encargo. \\
\hline 
Ubicaci\'on & Calle Constituci\'on 124, colonia Centro, Veracruz, Veracruz, C.P. 91700.\\
\hline

\end{tabular}
\end{table}

\begin{table}[H]
\centering
\begin{tabular}{|c|m{.4\textwidth}|}
\hline
Obra& Dibujo	\\
\hline
Autor & Se atribuye a Ra\'ul Gonzalez Esquivel\\
\hline
T\'itulo & ``Carita Sonriente. Cultura Totonaca. S. VI-XI D.C. Los Cerros, Veracruz, Octubre 08 de 1993.'' \\
\hline
Tem\'atica & Representaci\'on de arte precolombino, im\'agenes de deidades, figuras antropomorfas, cabezas colosales, culto religioso ind\'igena y personajes mitol\'ogicos.\\
\hline
T\'ecnicas &Grafito sobre papel \\
\hline
Medidas & 41.5 cm $\times$ 59 cm \\
\hline
 Formato & Peque\~no \\
 \hline
 Firma & Se atribuye a Ra\'ul Gonzalez Esquivel\\ 
 \hline
  Fecha & Octubre 8 de 1993.\\
 \hline
 Provenance & Colecci\'on particular\\
 \hline
 Certificado de autenticidad& Firmado por Pablo C. Goebel el 10 de abril de 1996.  \\
 \hline 
  Materiales empleados & Lapiz sobre papel\\
 \hline
 Caracter\'isticas especiales & Forma parte de una colecci\'on privada de 39 dibujos. 
Contiene dibujo con el texto ``Bufete Campos'' que corresponde a la obra por encargo. \\
\hline 
Ubicaci\'on & Calle Constituci\'on 124, colonia Centro, Veracruz, Veracruz, C.P. 91700.\\
\hline

\end{tabular}
\end{table}

\begin{table}[H]
\centering
\begin{tabular}{|c|m{.4\textwidth}|}
\hline
Obra& Dibujo	\\
\hline
Autor & Se atribuye a Ra\'ul Gonzalez Esquivel\\
\hline
T\'itulo & ``Figura SOnriente Desnuda. El Zapotal. Mpio. de Ignacio de la Llave. Veracruz. Cultura Totonaca. Cl\'asico tard\'io. S. VI-IX D.C. Octubre 3 de 1993.''\\
\hline
Tem\'atica & Representaci\'on de arte precolombino, im\'agenes de deidades, figuras antropomorfas, cabezas colosales, culto religioso ind\'igena y personajes mitol\'ogicos.\\
\hline
T\'ecnicas &Grafito sobre papel \\
\hline
Medidas & 41.5 cm $\times$ 59 cm \\
\hline
 Formato & Peque\~no \\
 \hline
 Firma & Se atribuye a Ra\'ul Gonzalez Esquivel\\ 
 \hline
  Fecha & Octubree 3 de 1993.\\
 \hline
 Provenance & Colecci\'on particular\\
 \hline
 Certificado de autenticidad& Firmado por Pablo C. Goebel el 10 de abril de 1996.  \\
 \hline 
  Materiales empleados & Lapiz sobre papel\\
 \hline
 Caracter\'isticas especiales & Forma parte de una colecci\'on privada de 39 dibujos. 
Contiene dibujo con el texto ``Bufete Campos'' que corresponde a la obra por encargo. \\
\hline 
Ubicaci\'on & Calle Constituci\'on 124, colonia Centro, Veracruz, Veracruz, C.P. 91700.\\
\hline

\end{tabular}
\end{table}

\begin{table}[H]
\centering
\begin{tabular}{|c|m{.4\textwidth}|}
\hline
Obra& Dibujo	\\
\hline
Autor & Se atribuye a Ra\'ul Gonzalez Esquivel\\
\hline
T\'itulo & ``Figura S\'edente. Cultura Totonaca. El Salto. Mpio de Ignacio de la Llave. Veracruz. Octubre 4 de 1993.'' \\
\hline
Tem\'atica & Representaci\'on de arte precolombino, im\'agenes de deidades, figuras antropomorfas, cabezas colosales, culto religioso ind\'igena y personajes mitol\'ogicos.\\
\hline
T\'ecnicas &Grafito sobre papel \\
\hline
Medidas & 41.5 cm $\times$ 59 cm \\
\hline
 Formato & Peque\~no \\
 \hline
 Firma & Se atribuye a Ra\'ul Gonzalez Esquivel\\ 
 \hline
  Fecha & Octubre 4 de 1993. \\
 \hline
 Provenance & Colecci\'on particular\\
 \hline
 Certificado de autenticidad& Firmado por Pablo C. Goebel el 10 de abril de 1996.  \\
 \hline 
  Materiales empleados & Lapiz sobre papel\\
 \hline
 Caracter\'isticas especiales & Forma parte de una colecci\'on privada de 39 dibujos. 
Contiene dibujo con el texto ``Bufete Campos'' que corresponde a la obra por encargo. \\
\hline 
Ubicaci\'on & Calle Constituci\'on 124, colonia Centro, Veracruz, Veracruz, C.P. 91700.\\
\hline

\end{tabular}
\end{table}

\begin{table}[H]
\centering
\begin{tabular}{|c|m{.4\textwidth}|}
\hline
Obra& Dibujo	\\
\hline
Autor & Se atribuye a Ra\'ul Gonzalez Esquivel\\
\hline
T\'itulo & ``Figura Sonriente. Cultura Totonaca. Veracruz. Octubre 2 de 1993.'' \\
\hline
Tem\'atica & Representaci\'on de arte precolombino, im\'agenes de deidades, figuras antropomorfas, cabezas colosales, culto religioso ind\'igena y personajes mitol\'ogicos.\\
\hline
T\'ecnicas &Grafito sobre papel \\
\hline
Medidas & 41.5 cm $\times$ 59 cm \\
\hline
 Formato & Peque\~no \\
 \hline
 Firma & Se atribuye a Ra\'ul Gonzalez Esquivel\\ 
 \hline
  Fecha & Octubre 2 de 1993.\\
 \hline
 Provenance & Colecci\'on particular\\
 \hline
 Certificado de autenticidad& Firmado por Pablo C. Goebel el 10 de abril de 1996.  \\
 \hline 
  Materiales empleados & Lapiz sobre papel\\
 \hline
 Caracter\'isticas especiales & Forma parte de una colecci\'on privada de 39 dibujos. 
Contiene dibujo con el texto ``Bufete Campos'' que corresponde a la obra por encargo. \\
\hline 
Ubicaci\'on & Calle Constituci\'on 124, colonia Centro, Veracruz, Veracruz, C.P. 91700.\\
\hline

\end{tabular}
\end{table}

\begin{table}[H]
\centering
\begin{tabular}{|c|m{.4\textwidth}|}
\hline
Obra& Dibujo	\\
\hline
Autor & Se atribuye a Ra\'ul Gonzalez Esquivel\\
\hline
T\'itulo & ''Hacha Votiva con cara de guerrero. Diciembre 10 de 1993.'' \\
\hline
Tem\'atica & Representaci\'on de arte precolombino, im\'agenes de deidades, figuras antropomorfas, cabezas colosales, culto religioso ind\'igena y personajes mitol\'ogicos.\\
\hline
T\'ecnicas &Grafito sobre papel \\
\hline
Medidas & 41.5 cm $\times$ 59 cm \\
\hline
 Formato & Peque\~no \\
 \hline
 Firma & Se atribuye a Ra\'ul Gonzalez Esquivel\\ 
 \hline
  Fecha & Diciembre 10 de 1993. \\
 \hline
 Provenance & Colecci\'on particular\\
 \hline
 Certificado de autenticidad& Firmado por Pablo C. Goebel el 10 de abril de 1996.  \\
 \hline 
  Materiales empleados & Lapiz sobre papel\\
 \hline
 Caracter\'isticas especiales & Forma parte de una colecci\'on privada de 39 dibujos. 
Contiene dibujo con el texto ``Bufete Campos'' que corresponde a la obra por encargo. \\
\hline 
Ubicaci\'on & Calle Constituci\'on 124, colonia Centro, Veracruz, Veracruz, C.P. 91700.\\
\hline

\end{tabular}
\end{table}

\begin{table}[H]
\centering
\begin{tabular}{|c|m{.4\textwidth}|}
\hline
Obra& Dibujo	\\
\hline
Autor & Se atribuye a Ra\'ul Gonzalez Esquivel\\
\hline
T\'itulo & ``Hacha Votiva, representa a un guerrero muerto. Napatecuhtlan, Mpio. de Perote. Cl\'asico Tard\'io S. VI-IX. Diciembre 10 de 1993.'' \\
\hline
Tem\'atica & Representaci\'on de arte precolombino, im\'agenes de deidades, figuras antropomorfas, cabezas colosales, culto religioso ind\'igena y personajes mitol\'ogicos.\\
\hline
T\'ecnicas &Grafito sobre papel \\
\hline
Medidas & 41.5 cm $\times$ 59 cm \\
\hline
 Formato & Peque\~no \\
 \hline
 Firma & Se atribuye a Ra\'ul Gonzalez Esquivel\\ 
 \hline
  Fecha & Diciembre 10 de 1993.\\
 \hline
 Provenance & Colecci\'on particular\\
 \hline
 Certificado de autenticidad& Firmado por Pablo C. Goebel el 10 de abril de 1996.  \\
 \hline 
  Materiales empleados & Lapiz sobre papel\\
 \hline
 Caracter\'isticas especiales & Forma parte de una colecci\'on privada de 39 dibujos. 
Contiene dibujo con el texto ``Bufete Campos'' que corresponde a la obra por encargo. \\
\hline 
Ubicaci\'on & Calle Constituci\'on 124, colonia Centro, Veracruz, Veracruz, C.P. 91700.\\
\hline

\end{tabular}
\end{table}

\begin{table}[H]
\centering
\begin{tabular}{|c|m{.4\textwidth}|}
\hline
Obra& Dibujo	\\
\hline
Autor & Se atribuye a Ra\'ul Gonzalez Esquivel\\
\hline
T\'itulo & ``Tlasolteotl. Santana, Mpio. de Tlalixcoyan. Cl\'asico Tard\'io S. VI-IX D.C. Noviembre 26 de 1993.''\\
\hline
Tem\'atica & Representaci\'on de arte precolombino, im\'agenes de deidades, figuras antropomorfas, cabezas colosales, culto religioso ind\'igena y personajes mitol\'ogicos.\\
\hline
T\'ecnicas &Grafito sobre papel \\
\hline
Medidas & 41.5 cm $\times$ 59 cm \\
\hline
 Formato & Peque\~no \\
 \hline
 Firma & Se atribuye a Ra\'ul Gonzalez Esquivel\\ 
 \hline
  Fecha & Noviembre 26 de 1993. \\
 \hline
 Provenance & Colecci\'on particular\\
 \hline
 Certificado de autenticidad& Firmado por Pablo C. Goebel el 10 de abril de 1996.  \\
 \hline 
  Materiales empleados & Lapiz sobre papel\\
 \hline
 Caracter\'isticas especiales & Forma parte de una colecci\'on privada de 39 dibujos. 
Contiene dibujo con el texto ``Bufete Campos'' que corresponde a la obra por encargo. \\
\hline 
Ubicaci\'on & Calle Constituci\'on 124, colonia Centro, Veracruz, Veracruz, C.P. 91700.\\
\hline

\end{tabular}
\end{table}

\begin{table}[H]
\centering
\begin{tabular}{|c|m{.4\textwidth}|}
\hline
Obra& Dibujo	\\
\hline
Autor & Se atribuye a Ra\'ul Gonzalez Esquivel\\
\hline
T\'itulo & ``Cinuateteotl no. 2. El Cocuhite, Mpio. de Tlalixcoyan. Cl\'asico Tard\'io S. VI-IX D.C. Noviembre 27 de 1993.''\\
\hline
Tem\'atica & Representaci\'on de arte precolombino, im\'agenes de deidades, figuras antropomorfas, cabezas colosales, culto religioso ind\'igena y personajes mitol\'ogicos.\\
\hline
T\'ecnicas &Grafito sobre papel \\
\hline
Medidas & 41.5 cm $\times$ 59 cm \\
\hline
 Formato & Peque\~no \\
 \hline
 Firma & Se atribuye a Ra\'ul Gonzalez Esquivel\\ 
 \hline
  Fecha & Noviembre 27 de 1993.\\
 \hline
 Provenance & Colecci\'on particular\\
 \hline
 Certificado de autenticidad& Firmado por Pablo C. Goebel el 10 de abril de 1996.  \\
 \hline 
  Materiales empleados & Lapiz sobre papel\\
 \hline
 Caracter\'isticas especiales & Forma parte de una colecci\'on privada de 39 dibujos. 
Contiene dibujo con el texto ``Bufete Campos'' que corresponde a la obra por encargo. \\
\hline 
Ubicaci\'on & Calle Constituci\'on 124, colonia Centro, Veracruz, Veracruz, C.P. 91700.\\
\hline

\end{tabular}
\end{table}

\begin{table}[H]
\centering
\begin{tabular}{|c|m{.4\textwidth}|}
\hline
Obra& Dibujo	\\
\hline
Autor & Se atribuye a Ra\'ul Gonzalez Esquivel\\
\hline
T\'itulo & Cihuateteotl no.1, del Cocuhite, Mpio. de Tlalixcoyan. Cl\'asico Tard\'io. S. VI-IX. Noviembre 18 de 1998.''\\
\hline
Tem\'atica & Representaci\'on de arte precolombino, im\'agenes de deidades, figuras antropomorfas, cabezas colosales, culto religioso ind\'igena y personajes mitol\'ogicos.\\
\hline
T\'ecnicas &Grafito sobre papel \\
\hline
Medidas & 41.5 cm $\times$ 59 cm \\
\hline
 Formato & Peque\~no \\
 \hline
 Firma & Se atribuye a Ra\'ul Gonzalez Esquivel\\ 
 \hline
  Fecha & Noviembre 18 de 1993.\\
 \hline
 Provenance & Colecci\'on particular\\
 \hline
 Certificado de autenticidad& Firmado por Pablo C. Goebel el 10 de abril de 1996.  \\
 \hline 
  Materiales empleados & Lapiz sobre papel\\
 \hline
 Caracter\'isticas especiales & Forma parte de una colecci\'on privada de 39 dibujos. 
Contiene dibujo con el texto ``Bufete Campos'' que corresponde a la obra por encargo. \\
\hline 
Ubicaci\'on & Calle Constituci\'on 124, colonia Centro, Veracruz, Veracruz, C.P. 91700.\\
\hline

\end{tabular}
\end{table}

\begin{table}[H]
\centering
\begin{tabular}{|c|m{.4\textwidth}|}
\hline
Obra& Dibujo	\\
\hline
Autor & Se atribuye a Ra\'ul Gonzalez Esquivel\\
\hline
T\'itulo & ``Danzante. Centro de Veracruz. Cl\'asico Tard\'io S. VI-IX. Noviembre 10 de 1993.'' \\
\hline
Tem\'atica & Representaci\'on de arte precolombino, im\'agenes de deidades, figuras antropomorfas, cabezas colosales, culto religioso ind\'igena y personajes mitol\'ogicos.\\
\hline
T\'ecnicas &Grafito sobre papel \\
\hline
Medidas & 41.5 cm $\times$ 59 cm \\
\hline
 Formato & Peque\~no \\
 \hline
 Firma & Se atribuye a Ra\'ul Gonzalez Esquivel\\ 
 \hline
  Fecha & Noviembre 10 de 1993.\\
 \hline
 Provenance & Colecci\'on particular\\
 \hline
 Certificado de autenticidad& Firmado por Pablo C. Goebel el 10 de abril de 1996.  \\
 \hline 
  Materiales empleados & Lapiz sobre papel\\
 \hline
 Caracter\'isticas especiales & Forma parte de una colecci\'on privada de 39 dibujos. 
Contiene dibujo con el texto ``Bufete Campos'' que corresponde a la obra por encargo. \\
\hline 
Ubicaci\'on & Calle Constituci\'on 124, colonia Centro, Veracruz, Veracruz, C.P. 91700.\\
\hline

\end{tabular}
\end{table}

\begin{table}[H]
\centering
\begin{tabular}{|c|m{.4\textwidth}|}
\hline
Obra& Dibujo	\\
\hline
Autor & Se atribuye a Ra\'ul Gonzalez Esquivel\\
\hline
T\'itulo & ``Figura Sonriente Femenina. El Zapotal, Mpio. de Ignacio de la Llave. Cl\'asico Tard\'io S. VI-IX. Noviembre 13 de 1993.''\\
\hline
Tem\'atica & Representaci\'on de arte precolombino, im\'agenes de deidades, figuras antropomorfas, cabezas colosales, culto religioso ind\'igena y personajes mitol\'ogicos.\\
\hline
T\'ecnicas &Grafito sobre papel \\
\hline
Medidas & 41.5 cm $\times$ 59 cm \\
\hline
 Formato & Peque\~no \\
 \hline
 Firma & Se atribuye a Ra\'ul Gonzalez Esquivel\\ 
 \hline
  Fecha & Noviembre 13 de 1993. \\
 \hline
 Provenance & Colecci\'on particular\\
 \hline
 Certificado de autenticidad& Firmado por Pablo C. Goebel el 10 de abril de 1996.  \\
 \hline 
  Materiales empleados & Lapiz sobre papel\\
 \hline
 Caracter\'isticas especiales & Forma parte de una colecci\'on privada de 39 dibujos. 
Contiene dibujo con el texto ``Bufete Campos'' que corresponde a la obra por encargo. \\
\hline 
Ubicaci\'on & Calle Constituci\'on 124, colonia Centro, Veracruz, Veracruz, C.P. 91700.\\
\hline

\end{tabular}
\end{table}

\begin{table}[H]
\centering
\begin{tabular}{|c|m{.4\textwidth}|}
\hline
Obra& Dibujo	\\
\hline
Autor & Se atribuye a Ra\'ul Gonzalez Esquivel\\
\hline
T\'itulo & ``Jugador de Pelota. Mata Redonda, Mpio. de Villa Cuauht\'emoc. S. II-IX D.C. Periodo Panuco. Noviembre 1 de 1993.''\\
\hline
Tem\'atica & Representaci\'on de arte precolombino, im\'agenes de deidades, figuras antropomorfas, cabezas colosales, culto religioso ind\'igena y personajes mitol\'ogicos.\\
\hline
T\'ecnicas &Grafito sobre papel \\
\hline
Medidas & 41.5 cm $\times$ 59 cm \\
\hline
 Formato & Peque\~no \\
 \hline
 Firma & Se atribuye a Ra\'ul Gonzalez Esquivel\\ 
 \hline
  Fecha & Noviembre 1 de 1993.\\
 \hline
 Provenance & Colecci\'on particular\\
 \hline
 Certificado de autenticidad& Firmado por Pablo C. Goebel el 10 de abril de 1996.  \\
 \hline 
  Materiales empleados & Lapiz sobre papel\\
 \hline
 Caracter\'isticas especiales & Forma parte de una colecci\'on privada de 39 dibujos. 
Contiene dibujo con el texto ``Bufete Campos'' que corresponde a la obra por encargo. \\
\hline 
Ubicaci\'on & Calle Constituci\'on 124, colonia Centro, Veracruz, Veracruz, C.P. 91700.\\
\hline

\end{tabular}
\end{table}

\begin{table}[H]
\centering
\begin{tabular}{|c|m{.4\textwidth}|}
\hline
Obra& Dibujo	\\
\hline
Autor & Se atribuye a Ra\'ul Gonzalez Esquivel\\
\hline
T\'itulo & ``Tlasolteotl, con Gran Quechquemitl  Napiola, Mpio. de Tierra Blanca, Ver. Noviembre 03 de 1993.''\\
\hline
Tem\'atica & Representaci\'on de arte precolombino, im\'agenes de deidades, figuras antropomorfas, cabezas colosales, culto religioso ind\'igena y personajes mitol\'ogicos.\\
\hline
T\'ecnicas &Grafito sobre papel \\
\hline
Medidas & 41.5 cm $\times$ 59 cm \\
\hline
 Formato & Peque\~no \\
 \hline
 Firma & Se atribuye a Ra\'ul Gonzalez Esquivel\\ 
 \hline
  Fecha & Noviembre 3 de 1993.\\
 \hline
 Provenance & Colecci\'on particular\\
 \hline
 Certificado de autenticidad& Firmado por Pablo C. Goebel el 10 de abril de 1996.  \\
 \hline 
  Materiales empleados & Lapiz sobre papel\\
 \hline
 Caracter\'isticas especiales & Forma parte de una colecci\'on privada de 39 dibujos. 
Contiene dibujo con el texto ``Bufete Campos'' que corresponde a la obra por encargo. \\
\hline 
Ubicaci\'on & Calle Constituci\'on 124, colonia Centro, Veracruz, Veracruz, C.P. 91700.\\
\hline

\end{tabular}
\end{table}

\begin{table}[H]
\centering
\begin{tabular}{|c|m{.4\textwidth}|}
\hline
Obra& Dibujo	\\
\hline
Autor & Se atribuye a Ra\'ul Gonzalez Esquivel\\
\hline
T\'itulo & ``Hacha Votiva en forma de garza. Mpio de Coxquihui, VEracruz. Cla\'asico Tard\'io S. VI-IX D.C. Diciembre 10 de 1993.''\\
\hline
Tem\'atica & Representaci\'on de arte precolombino, im\'agenes de deidades, figuras antropomorfas, cabezas colosales, culto religioso ind\'igena y personajes mitol\'ogicos.\\
\hline
T\'ecnicas &Grafito sobre papel \\
\hline
Medidas & 41.5 cm $\times$ 59 cm \\
\hline
 Formato & Peque\~no \\
 \hline
 Firma & Se atribuye a Ra\'ul Gonzalez Esquivel\\ 
 \hline
  Fecha & Diciembre 10 de 1993.\\
 \hline
 Provenance & Colecci\'on particular\\
 \hline
 Certificado de autenticidad& Firmado por Pablo C. Goebel el 10 de abril de 1996.  \\
 \hline 
  Materiales empleados & Lapiz sobre papel\\
 \hline
 Caracter\'isticas especiales & Forma parte de una colecci\'on privada de 39 dibujos. 
Contiene dibujo con el texto ``Bufete Campos'' que corresponde a la obra por encargo. \\
\hline 
Ubicaci\'on & Calle Constituci\'on 124, colonia Centro, Veracruz, Veracruz, C.P. 91700.\\
\hline

\end{tabular}
\end{table}

