

\textcolor{principal}{REGLAMENTO DEL C\'ODIGO FISCAL DE LA FEDERACI\'ON}\\


\textcolor{secundario}{ART\'ICULO 3.-} \textit{Los aval\'uos que se practiquen para efectos fiscales tendr\'an vigencia de un a\~no, contado a partir de la fecha en que se emitan, para lo cual, las Autoridades Fiscales aceptar\'an los aval\'uos en relaci\'on con los bienes que se ofrezcan para garantizar el inter\'es fiscal o cuando sea necesario contar con un aval\'uo en t\'erminos de lo previsto en el Cap\'itulo III del Título V del C\'odigo..} \\

\textit{Los aval\'uos a que se refiere el p\'arrafo anterior, deber\'an ser practicados por los peritos valuadores siguientes}

\begin{enumerate}[\indent I.]

\item El Instituto de Administraci\'on y Aval\'uos de Bienes Nacionales; 
\item Instituciones de cr\'edito; 
\item Corredores p\'ublicos que cuenten con registro vigente ante la Secretar\'ia de Econom\'ia, y 
\item Empresas dedicadas a la compraventa o subasta de bienes. 
\end{enumerate}

\textit{La Autoridad Fiscal en los casos que proceda y mediante el procedimiento que al efecto establezca el Servicio de Administraci\'on Tributaria mediante reglas de car\'acter general, podr\'a solicitar la pr\'actica de un segundo aval\'uo. El valor determinado en dicho aval\'uo ser\'a el que prevalezca. } \\

\textit{En aquellos casos en que despu\'es de realizado el aval\'uo se lleven a cabo construcciones, instalaciones o mejoras permanentes al bien inmueble de que se trate, los valores consignados en dicho aval\'uo quedar\'an sin efecto, aun cuando no haya transcurrido el plazo se\~nalado en el primer p\'arrafo de este art\'iculo} \\

\textit{En los aval\'uos referidos a una fecha anterior a aquella en que se practiquen, se proceder\'a conforme a lo siguiente}''\\

\begin{enumerate}[\indent a)]

\item Se determinar\'a el valor del bien a la fecha en que se practique el aval\'uo; 
\item La cantidad obtenida conforme a la fracci\'on anterior se dividir\'a entre el factor que se obtenga de dividir el \'indice Nacional de Precios al Consumidor del mes inmediato anterior a aqu\'el en que se practique el aval\'uo, entre el \'indice del mes al cual es referido el mismo, y 
\item El resultado que se obtenga conforme a la operaci\'on a que se refiere el inciso anterior ser\'a el valor del bien a la fecha a la que el aval\'uo sea referido. El valuador podr\'a efectuar ajustes a este valor cuando existan razones que as\'i lo justifiquen, antes de la presentaci\'on del aval\'uo, las cuales deber\'an se\~nalarse expresamente en el mismo documento. 

\end{enumerate}

