\subsection{APLICACIÓN DEL MODELO DE PRESUPUESTO DE CAPITAL.}\label{5.1}

\subsubsection{Estimación de los parámetros para la integración de un Flujo de Caja libre (FCL) unitario de los productos relacionados con la marca.}

Para poder llevar a cabo la estimación de un Flujo de Caja libre unitario correspondiente a los productos amparados por la marca INNOMINADA 694670, el valuador tuvo que consultar a través de las bases de datos de banca inversión, la estructura financiera de la sociedad más parecida a la solicitante BURBERRY LIMITED, habiéndose encontrado la información pública de la entidad Burberry Group PLC (BRBY.L); la cual sirvió como el comparable idóneo para obtener los elevadores de valor del presupuesto marcario (value drivers) y para determinar los siguientes parámetros indispensables para una estimación de Flujo de Caja:

\begin{enumerate}[a)]
\item \textbf{Entradas de efectivo (INFLOWS): }
	\begin{itemize}
	\item Margen bruto (\textit{Gross Profit})
	\item Margen operativo (antes y después de Depreciación, amortización, intereses e impuestos), \textit{EBITDA} y \textit{EBIT}. 
	\item Margen de operación neto después de impuestos (\textit{NOPAT}).

	\end{itemize}

\item \textbf{Salidas de efectivo (OUTFLOWS): }
	\begin{itemize}
	\item Tasa de Reinversión. Para determinar el CAPEX Neto del proyecto y los probables egresos por cambios al capital de trabajo (\textit{NWC Changes}).

	\end{itemize}
\item \textbf{Parámetros de productividad marcaria:}
	\begin{itemize}
	\item ROA (Rendimiento de los Activos).- Return on Average Total Assets - \% 
	\item ROIC (Rendimiento del Capital Invertido).- Return on Invested Capital - \%
	\item ROE (Rendimiento del Patrimonio Neto). Return on Average Common Equity - \%
	\end{itemize}	
\end{enumerate}

\begin{figure}[H]
	\centering
	\includegraphics[width=10cm]{../0.imagenes/company_fundamentals}
\end{figure}

\begin{figure}[H]
	\centering
	\includegraphics[width=\textwidth]{../0.imagenes/financial_summary}
\end{figure}

\subsubsection{Estimación de los ingresos marcarios y el Flujo de Caja Libre Unitario para la determinación del Lucro Cesante y el Deterioro financiero de la marca.}

El valuador aplicó a cada producto una medida de estadística descriptiva para determinar el valor de los ingresos aplicable por cada SKU. En este caso el valuador eligió el estadístico conocido como ``MODA''; y en algunos casos en que no hubiera repetición de los datos del precio de cada  prenda,  se eligió el estadístico ``MEDIANA'', como medida central. Una vez estimados los ingresos unitarios por producto, el valuador aplicó a dichos importes los porcentajes aplicables a cada margen financiero, para integrar un presupuesto de ingresos y egresos marcario (P\&L), según se muestra a continuación:

\begin{figure}[H]
	\centering
	\includegraphics[width=\textwidth]{../0.imagenes/flujo_de_caja_libre_1}
\end{figure}

\begin{figure}[H]
	\centering
	\includegraphics[width=\textwidth]{../0.imagenes/flujo_de_caja_libre_2}
\end{figure}

\begin{figure}[H]
	\centering
	\includegraphics[width=.7\textwidth]{../0.imagenes/flujo_de_caja_libre_3}
\end{figure}

\subsubsection{Estimación del Deterioro y el Lucro Cesante por Producto correspondiente a la marca, por el método de Flujos de Caja Libre (FCL) conforme al modelo \ref{5.1}:}

\begin{figure}[H]
	\centering
	\includegraphics[width=\textwidth]{../0.imagenes/flujo_de_caja_libre_4}
\end{figure}

\begin{figure}[H]
	\centering
	\includegraphics[width=\textwidth]{../0.imagenes/flujo_de_caja_libre_5}
\end{figure}

\begin{figure}[H]
	\centering
	\includegraphics[width=.7\textwidth]{../0.imagenes/flujo_de_caja_libre_6}
\end{figure}
