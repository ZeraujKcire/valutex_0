\subsection{Marco te\'orico del DCF en su modalidad para empresas estresadas o declinantes.}

El profesor Aswath Damodaran\footnote{Cuya metodolog\'ia est\'a creada para valuar empresas complejas, en situaci\'on de estr\'es o en etapas tempranas (Distressed, Complex and Young Companies)}, en su libro The Dark Side of Valuation, propone en su cap\'itulo de empresas declinantes una modalidad para la estimaci\'on del Modelo de Flujos de Efectivo Descontados para empresas en situación de estr\'es o p\'erdidas; la cual consiste en los siguientes pasos:

\begin{enumerte}[I.]

\item Estimaci\'on del monto porcentual de p\'erdidas operativas respecto a ingresos: Si se trata de una empresa con p\'erdidas operativas hist\'oricas y constantes, es necesario que el analista financiero estime el promedio de dichas p\'erdidas operativas a nivel porcentual, respecto de sus ``ingresos netos''. Una vez obtenida dicha m\'etrica, se requiere evaluar la probabilidad emp\'irica de que dichas p\'erdidas contin\'uen o no en el corto o mediano plazo.

\item Proyecci\'on financiera de las p\'erdidas operativas: En caso de que sea probable que dichos gastos operativos se mantengan y contin\'uen, se requiere proyectar las p\'erdidas operativas (NOPAT \footnote{ Net operating profit after tax.})a un plazo razonable, mas nunca a perpetuidad. En algunos casos excepcionales, habr\'a que impactar tambi\'en en dicha proyecci\'on la tasa de reinversi\'on del negocio en sus componentes de Capital de Trabajo y CAPEX.

\item Estimaci\'on de la tasa de descuento aplicable a los flujos negativos: Una vez que el analista cuenta con una proyecci\'on financiera de las p\'erdidas operativas del negocio, ser\'a necesario calcular una tasa de descuento adecuada para dichos flujos negativos, la cual deber\'a impactar el costo de oportunidad del dinero y el riesgo importante que enfrentan este tipo de negocios en situación de estrés; cuyas betas reapalancadas tienden a ser mayores de 1.0, incluso com\'unmente en rangos superiores a 2.0. Generalmente se utiliza la tasa de capital ponderada que contempla el peso de la deuda y del capital para cada tipo de fondeo (WACC).

\item Aplicaci\'on del enfoque residual contra el capital invertido. Se estimar\'an las p\'erdidas operativas a valor presente y dicha cifra se contrastar\'a respecto del capital invertido de la empresa, a manera de un valor de liquidaci\'on, donde las inversiones de una empresa deber\'an amortizar las p\'erdidas operativas futuras a valor presente, bajo dl enfoque residual.

Como ejemplo, el profesor Damodaran utiliza la analog\'ia de una persona que pierde su empleo y ver\'a interrumpido sus ingresos, sin embargo, es probable que sus gastos personales y familiares contin\'uen, por lo que deberá hacer un P\&L y proyectar el monto de sus p\'erdidas operativos en el corto o mediano plazo hasta que nuevamente consiga empleo, estimar dichos flujos negativos de dinero a valor presente; para posteriormente utilizar sus ahorras y tal vez realizar una parte importante de sus inversiones en efectivo (lo que implicará vender algunos activos probablemente); y con dicho dinero hacer frente a sus próximas sentido.

\end{enumerate}
