\textcolor{secundario}{ESTIMACI\'ON DEL COSTO PROMEDIO PONDERADO DE CAPITAL o WACC.} Consiste en la tasa a la cual son descontados todos los proyectos de la entidad. Se dice que la rentabilidad m\'inima producto de su operaci\'on debiera ser la \gls{wacc}, ya que considera en su c\'alculo la proporci\'on de deuda expl\'icita y capital propio, beneficio fiscal a la deuda, as\'i como la intervenci\'on de las tasas \gls{kd} y \gls{ke}, dada la mezcla de capitales en inversi\'on.\\[5pt]

Para determinar una estimaci\'on del costo promedio ponderado de capital (\gls{wacc}) de todo el capital invertido de la empresa a valor de mercado, se utiliz\'o la f\'ormula que se muestra a continuaci\'on:

$$WACC=\left(kd(1-t)\frac{D}{D+E}\right)+\left(ke\frac{E}{D+E}\right)$$