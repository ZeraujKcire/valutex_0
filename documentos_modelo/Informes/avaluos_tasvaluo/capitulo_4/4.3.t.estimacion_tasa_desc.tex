\begin{rightcolumn}

\subsection{{MARCO TE\'ORICO SOBRE LA ESTIMACI\'ON DE LA TASA DE DESCUENTO}. 

La tasa de descuento puede ser calculada bajo los modelos \gls{wacc} o \gls{wara}. Dicha tasa de descuento consiste en el rendimiento m\'Inimo esperado para la organizaci\'on (\autoref{fig:op_acc} y \autoref{fig:narr_val} ).

\begin{enumerate}[i)]
\item \textcolor{secundario}{\gls{wacc} (costo promedio ponderado de capital).} El costo de capital es la compensaci\'on que los inversionistas exigen de parte de las firmas que utilizan sus fondos (costo de oportunidad).
\item \textcolor{secundario}{\gls{wara} (rendimiento del promedio ponderado de los activos).}Representa el promedio ponderado de las tasas de retorno de los Activos contributivos involucrados en una valuaci\'on de negocios.
\end{enumerate}

\begin{figure}[H]
\centering
\caption{Tasa de descuento WACC \& WARA.\label{fig:wacc_wara}}
\includegraphics[width=10cm]{\rutaImagenes/wacc_wara}
\end{figure}


\end{rightcolumn}

\begin{leftcolumn}

\begin{figure}[H]
\centering
\caption{Costo de oportunidad de los Accionistas\label{fig:op_acc}}
\includegraphics[height=5cm]{\rutaImagenes/costo_oportunidad_accionistas}
\footnotesize{Fuente: Valuaci\'on de Activos Intangibles. DEAL ADVISORY MEXICO. KPMG C\'ARDENAS DOSAL, S.C. KPMG ``D.R.'' \copyright 2016}
\end{figure}

\begin{figure}[H]
\centering
\caption{Fundamentos de la Narrativa Valuatoria\label{fig:narr_val}}
\includegraphics[height=5.5cm]{\rutaImagenes/narrativa_valuatoria}\\
\footnotesize{Fuente:Damodaran, A. ``Session 14. Narrative to numbers''. NYU/STERN.}

\end{figure}

\end{leftcolumn}


