\begin{rightcolumn}

\subsection{Valuaci\'on por Flujos de Efectivo Descontados (\gls{dcf} Valuation Method).}

Para la valuaci\'on de una empresa es indispensable el conocimiento de los siguientes cuatro conceptos: i) Flujos de efectivo (Cash Flows \$). ii) Tasa de crecimiento de ingresos (\gls{growthrate} \%). iii) Tasa de descuento. (Disc. Rate \%). iv) Valor terminal. (\gls{terminalvalue} \$).\\

El modelo de Flujos de Efectivo descontados, es un m\'etodo din\'amico de valuaci\'on que toma en consideraci\'on el valor del dinero en el tiempo y permite evaluar el efecto concreto de las variables en los rendimientos y comportamientos futuros de la empresa. Se basa en medir la capacidad de generar riqueza futura, por lo que se proyecta el Flujo de Efectivo el cual se actualiza mediante su tasa de descuento. Su formulaci\'on se aprecia en la \textcolor{secundario}{\autoref{fig:dcf}}.\\

\end{rightcolumn}

\begin{leftcolumn}

\begin{figure}[H]
\centering
\caption{\textcolor{principal}{DCF Valuation Method (DCF)\label{fig:dcf}}}
$$VF=\left(\sum_{t=1}^n\frac{FCFE_t}{(1+WACC)_t}\right) + \frac{\left[\frac{(FCFE_n)(1+G_{lt})}{WACC_n-G_{lt}}\right]}{(1+WACC)_n}$$

\begin{minipage}{7cm}
\footnotesize
$t$: \'indice de periodo de proyecci\'on.\\
$FCFF$: Flujo de caja libre del periodo $t$.\\
$WACC_t$: Costo promedio ponderado de capital del periodo $t$.\\
$G_{lt}$: Crecimiento de largo plazo.\\
$VF$: Valor de la firma.\\
$N$: Horizonte de proyecci\'on.\\

\end{minipage}
\end{figure}

\end{leftcolumn}