\textcolor{principal}{\textbf{AN\'ALISIS DE PRESUPUESTO DE CAPITAL. Mediante la aplicaci\'on del enfoque de ingresos.} }\\

Consiste en el proceso de Planeaci\'on para la compra de los activos cuyos flujos de efectivo se espera contin\'uen mas all\'a de un a\~no.\\

\begin{center}
\begin{minipage}{8cm}
\textit{Existen cuatro etapas esenciales dentro del proceso de generaci\'on de un presupuesto de capital: i) Generar propuestas de proyectos de inversi\'on. ii) Estimar los flujos de efectivo. iii) Evaluar las opciones y elegir los proyectos que se instrumentarán, iv) Revisar el desempe\~no de un proyecto despu\'es de que se ha instrumentado y auditar su desempe\~no tras su conclusi\'on.}
\end{minipage}
\end{center}

Los proyectos de inversi\'on pueden generarse a partir de oportunidades de crecimiento, reducci\'on de costos y para cumplir requisitos legales y normas sanitarias o de seguridad.  La realizaci\'on de dicha metodolog\'ia requiere el c\'alculo de 2 premisas generalmente: i) Estimación de la Inversi\'on Neta, ii) Estimaci\'on del Flujo de efectivo de operaci\'on neto. (\autoref{fig:inv_neta} y \autoref{fig:feop_neto})

\begin{figure}[H]
\centering
\begin{minipage}{8cm}
\caption{Elementos de la Inversi\'on neta\label{fig:inv_neta}}
\includegraphics[height=5cm]{\rutaImagenes/capex_1}
\end{minipage}
\quad
\begin{minipage}{8cm}
\caption{Estimaci\'on del Flujo de efectivo de operaci\'on neto.\label{fig:feop_neto}}
\includegraphics[height=5.5cm]{\rutaImagenes/capex_2}\\
\end{minipage}
\end{figure}

En el \'ultimo a\~no del ciclo de vida de un proyecto, esta definici\'on de flujo de efectivo neto puede modificarse para reflejar la repercusi\'on de la inversi\'on de capital de trabajo neto y cualquier valor de recuperaci\'on de los activos generados despu\'es de impuestos.
