El perito valuador determinar\'a el valor razonable de mercado de las \textcolor{principal}{Inversiones Tangibles e Intangibles} de la unidad Econ\'omica inmobiliaria en la que se encuentran insertos los terrenos sujetos de valuaci\'on, como partidas indispensables que deben ejecutarse para poder realizar cualquier construcci\'on en el Inmueble y le otorguen plusval\'ia al Terreno.\\ 

Dentro de estos conceptos se se\~nalan de manera enunciativa m\'as no limitativa, los siguientes bloques y partidas:

\subsection{LICENCIAS, ESTUDIOS, TR\'AMITES Y GESTOR\'IA:} 
\begin{itemize}
\item Levantamiento topogr\'afico (estudio t\'ecnico y descriptivo de un terreno, examinando la superficie terrestre y sus caracter\'isticas f\'isicas).

\item Mec\'anica de Suelo (permite conocer el tipo de material del que est\'a compuesto el predio en el que se va ejecutar el proyecto con el objetivo de calcular y dise\~nar la cimentaci\'on y estructura adecuada).

\item Estudio de mercado (evaluaci\'on del proyecto a desarrollar mediante la b\'usqueda de inmuebles similares en cuanto a proyecto y uso particular en el estado de Quintana Roo).

\item Anteproyecto (planos centrales que conforman la idea y concepto de un proyecto plantas arquitect\'onicas, cortes y fachadas).

\item Proyecto ejecutivo (conjunto de planos donde se especifica y detalla cada parte que compone a la edificaci\'on para su construcci\'on).

\item Tr\'amites de agua y electricidad.

\item Tr\'amites gubernamentales (licencias, permisos y gestor\'ia).

\item Comisiones por corretaje de inmueble, y 

\item Compensaciones por servicios de desarrollo y advisory.

\item Entre otras.

\end{itemize}