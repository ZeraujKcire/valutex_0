
\textcolor{principal}{C\'odigo de Comercio:}\\[10pt]

\textit{``Art\'iculo 1252.- Los peritos deben tener t\'itulo en la ciencia, arte, t\'ecnica, oficio o industria a que pertenezca la cuesti\'on sobre la que ha de o\'irse su parecer, si la ciencia, arte, t\'ecnica, oficio o industria requieren t\'itulo para su ejercicio.}\\[10pt]

\textit{Si no lo requirieran o requiri\'endolo, no hubiere peritos en el lugar, podr\'an ser nombradas cualesquiera personas entendidas a satisfacci\'on del juez, aun cuando no tengan t\'itulo.}\\

\textit{La prueba pericial s\'olo ser\'a admisible cuando se requieran conocimientos especiales de la ciencia, arte, t\'ecnica, oficio o industria de que se trate, m\'as no en lo relativo a conocimientos generales que la ley presupone como necesarios en los jueces, por lo que se desechar\'an de oficio aquellas periciales que se ofrezcan por las partes para ese tipo de conocimientos, o que se encuentren acreditadas en autos con otras pruebas, o tan s\'olo se refieran a simples operaciones aritm\'eticas o similares.} \\[10pt] 

\textit{El t\'itulo de habilitaci\'on de corredor p\'ublico acredita para todos los efectos la calidad de perito valuador.''} (...)\\[10pt]

\textit{``Art\'iculo 1257.- Los jueces podr\'an designar peritos de entre aqu\'ellos autorizados como auxiliares de la administraci\'on de justicia por la autoridad local respectiva, o a solicitar que el perito sea propuesto por colegios, asociaciones o barras de profesionales, art\'isticas, t\'ecnicas o cient\'ificas o de las instituciones de educaci\'on superior p\'ublicas o privadas, o las c\'amaras de industria, comercio, o confederaciones de c\'amaras a la que corresponda al objeto del peritaje.  (...)''}\\[10pt]

\textit{``Art\'iculo 1300.- Los aval\'uos har\'an prueba plena.}\\[10pt]

