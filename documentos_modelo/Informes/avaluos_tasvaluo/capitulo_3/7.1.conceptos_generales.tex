\subsection{Conceptos Generales}
\begin{enumerate}[\indent\itshape 2.1.]
\item \textcolor{principal}{\textit{Valor de mercado. }}

\textit{Cuant\'ia estimada por la que un bien o activo, o pasivo deber\'ia intercambiarse en la fecha de valuaci\'on entre un comprador dispuesto a comprar y un vendedor dispuesto a vender, en una transacci\'on libre, tras una comercializaci\'on adecuada en las que las partes hayan actuado con conocimiento, de manera prudente y sin coacci\'on.''}

\item \textcolor{principal}{\textit{Valor razonable.} }

\textit{Precio estimado para la transacci\'on de un bien o activo, o pasivo entre dos partes identificadas, informadas y dispuestas que refleja los intereses respectivos de dichas partes. }\\

\textit{Esta definici\'on no es aplicable en el caso de valuaciones para uso en la elaboraci\'on de estados financieros, ver 5.1.''}

\item  \textcolor{principal}{\textit{Precio, Costo y Valor. }}

\begin{enumerate}[\indent\itshape 2.\theenumi.1.]
\item \textcolor{principal}{\textit{Precio. }}

\textit{Es la cantidad expresada en unidades monetarias, ofrecida o pagada por un bien o activo. Debido a la capacidad financiera, motivaci\'on o intereses especiales de un comprador o vendedor dado, el precio pagado pue de ser diferente del valor que podr\'ia asignarse a los bienes y servicios por otros.''}

\item  \textcolor{principal}{\textit{Costo .}}

\textit{Es la cantidad expresada en unidades monetarias requerida para crear o producir el bien o activo. Cuando ese bien o activo ha sido completado, su costo es un hecho. El precio se relaciona con el costo puesto que el precio pagado por un bien o servicio se convierte en su costo para el comprador.''}

\item \textcolor{principal}{ \textit{Valor. }}

\textit{Es una opini\'on de: }
\begin{enumerate}[\indent\itshape a)]

\item\textit{ El precio m\'as probable que habr\'ia de pagarse por un bien o activo en un intercambio, o;} 
\item \textit{Los beneficios econ\'omicos de tener en propiedad esos bienes o activos.''}

\end{enumerate}
\end{enumerate}

\item  \textcolor{principal}{\textit{Valor Comercial. }}

\textit{Valor Comercial el precio estimado y m\'as probable, por el cual una propiedad se intercambiar\'ia en la fecha del avalu\'o, entre un comprador y un vendedor actuando por voluntad propia en una transacci\'on sin intermediarios, con un plazo razonable de exposici\'on donde ambas partes act\'uan con conocimiento de los hechos pertinentes, con prudencia y sin compulsi\'on.''}\\

\textit{As\'i mismo, es importante aclarar que para efectos del presente dictamen deber\'a entenderse por Inversiones de Terreno a todas aquellas inversiones y preparaciones necesarias para el desarrollo de proyectos con el m\'aximo y mejor uso que presentan los inmuebles, mismo que dan plusval\'ia y el valor de mercado actual. }\\

\textit{Los conceptos desglosados en el presente aval\'uo son, por norma indispensables y necesarias a considerar en la ejecuci\'on de un proyecto arquitect\'onico, con excepci\'on del estudio de mercado, mismo que se realiz\'o con el objeto de verificar la viabilidad para construir a futuro y que dicho producto sea absorbido por el mercado de la zona. }\\

\item \textcolor{principal}{\textit{Valor De Reposici\'on Neto.}}

\textit{Cantidad estimada en t\'erminos monetarios a partir del valor de reposici\'on nuevo, deduciendo dem\'eritos existentes debidos al deterioro f\'isico, a la obsolescencia funcional y a la obsolescencia econ\'omica de cada porci\'on de construcci\'on del inmueble valuado.}

\item \textcolor{principal}{ \textit{Valor Residual.}}

\textit{El valor residual es el que resulta del an\'alisis de los beneficios y de los costos para un inversionista que adquiere un terreno urbano en bre\~na o con construcciones, para desarrollar en \'el un proyecto inmobiliario de aprovechamiento del mismo.}

\end{enumerate}
