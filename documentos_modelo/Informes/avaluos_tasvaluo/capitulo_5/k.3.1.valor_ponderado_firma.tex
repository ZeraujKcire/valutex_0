%\newcommand{\valorFirmaM}{601}
%\newcommand{\valorFirmam}{912}
%\newcommand{\valorFirmac}{958}
%\newcommand{\valorFirma}{\valorFirmaM,\valorFirmam,\valorFirmac}
%\newcommand{\valorFirmaLetra}{\Numberstringnum{\valorFirmaM}{} millones, \numberstringnum{\valorFirmam}{} mil, \numberstringnum{\valorFirmac}}

\newcommand{\ponda}{\peersa}
\newcommand{\pondaPorcentage}{10}
\newcommand{\pondb}{\peersb}
\newcommand{\pondbPorcentage}{20}
\newcommand{\pondc}{\peersc}
\newcommand{\pondcPorcentage}{10}
\newcommand{\pondd}{\peersd}
\newcommand{\ponddPorcentage}{30}
\newcommand{\ponde}{\peersd}
\newcommand{\pondePorcentage}{30}

\subsection{VALOR RAZONABLE PONDERADO DEL NEGOCIO EN MARCHA, CON CIFRAS AL \fechaValoresCorto.}

A continuaci\'on se concluye el valor razonable de la firma (\textit{\gls{firmvalue}}), a la fecha de valores, habi\'endose dado la siguiente importancia a los modelos en la ponderaci\'on: i) Un peso del \pondaPorcentage\% al modelo de valuaci\'on relativa conocido como \ponda{}, ii) Un peso del \pondbPorcentage\%  al modelo de valuaci\'on relativa (\gls{peers}) conocido como \pondb{} x; iii) Un peso del \pondcPorcentage\%  al modelo de valuaci\'on relativa (\gls{peers}) conocido como \pondc{} x; iv) Un peso del \ponddPorcentage\%  al modelo de valuaci\'on relativa (\gls{peers}) conocido como \pondd{} x;  v) Un peso del \pondePorcentage\%  al modelo de valuaci\'on relativa (\gls{peers}) conocido como \ponde{} x; seg\'un se muestra a continuaci\'on:

\begin{figure}[H]
\centering
\includegraphics[width=12cm]{../0.imagenes/valor_ponderado_firma}\\

\textbf{\textcolor{principal}{Valor de la Firma al  \fechaValoresCorto:} \$\valorFirma{} USD}\\
(\textcolor{principal}{\valorFirmaLetra{} d\'olares 00/100 USD})
\end{figure}
