\subsubsection{VALUACIÓN  DE MARCA MEDIANTE EL MODELO RFR.}

\textcolor{principal}{Aplicación del método}. Este modelo involucra la estimación del valor justo de un activo intangible, cuantificando el valor presente del flujo de pagos de regalías que el propietario del activo intangible está exento.\\

\textcolor{principal}{Estimación Flujos de ingresos para el modelo \gls{rfr}}. Para el cálculo de los flujos de ingresos del negocio, el perito valuador llevó a cabo una proyección financiera para el periodo \periodoProyeccion; según se detalla a continuación:

\begin{figure}[H]
\centering
\includegraphics[width=16cm]{../0.imagenes/rfr_1}
\end{figure}

\textcolor{principal}{Estimación de la tasa justa de regalías:} Del análisis de mercado para la obtención de la tasa justa de regalías, se obtuvo una medida central del \tasaRegalias\%\footnote{\estadisticoTasaRegalias.} sobre la base de Ventas Netas de los ingresos proyectados del activo intangible.\footnote{ El Boletín C-8 de las Mex NIF establece la cantidad de 10\% con un límite empírico para este tipo de tasas de regalías, lo cual resulta consistente con el estudio que se muestra.}:

\begin{figure}[H]
\centering
\includegraphics[width=16cm]{../0.imagenes/rfr_2}
\end{figure}

\textcolor{principal}{\textbf{Estimación del Flujo de Ahorro en Regalías y el Valor Terminal:}}

\begin{figure}[H]
\centering
\includegraphics[width=16cm]{../0.imagenes/rfr_3}\\[10pt]

\includegraphics[width=16cm]{../0.imagenes/rfr_4}
\end{figure}

\espacio{1cm}

\textcolor{principal}{INDICADOR DE VALOR DE LA MARCA POR FLUJOS DE AHORRO EN REGALíAS (\gls{rfr})}. El valuador aplicó la metodología descrita en este inciso para obtener el indicador de valor por este enfoque al \textcolor{principal}{\fechaValores}, según se detalla:



\begin{figure}[H]
\centering
\includegraphics[width=5cm]{../0.imagenes/rfr_final}
\end{figure}

