\begin{leftcolumn}


\subsection{VALUACIÓN  DE MARCA MEDIANTE EL MODELO RFR.}

\subsubsection{ANÁLISIS FINANCIERO DE LOS INGRESOS}

El perito valuador llevó a cabo un análisis financiero de los ingresos, con el objeto de establecer las bases para la proyección financiera de los flujos de pago de regalías del activo intangible:\\

\begin{figure}
\centering
\includegraphics[width=\textwidth]{../0.imagenes/rfr_01}
\end{figure}

Del anterior análisis el perito valuador decidió establecer el volumen de cajas surtidas como la base de su proyección de ingresos, en vista de que el parámetro $R^2$ del modelo lineal es de 0.9063; lo que se considera como una correlación adecuada para realizar una aproximación lineal del crecimiento del volumen de ventas para el periodo 2023-2027 (\textcolor{principal}{textit{Figura 13}}).

\subsubsection{ANÁLISIS FINANCIERO DEL FLUJO DE OPERACIÓN NETO.}

El perito valuador llevó a cabo la estimación del NOPAT atribuible al SKU de las paletas laser, con los siguientes resultados (\textcolor{principal}{\textit{Figura 14}}).

\begin{figure}
\centering
\includegraphics[width=\textwidth]{../0.imagenes/rfr_02}
\end{figure}

\subsubsection{PROYECCIONES FINANCIERAS.}

\textcolor{principal}{Proyección financiera de Ingresos 2023-2027.} El perito valuador llevó  una proyección de ingresos del negocio para la aplicación de los modelos EVA y Flujos Descontados de Regalías, partiendo del presupuesto de ingresos y egresos (P\&L Meridius) proporcionado por el solicitante, con base en los fundamentales del análisis financiero mencionado en el inciso anterior. Una vez proyectado el volumen de ventas, se llevó a cabo una estimación del precio unitario por caja al 2022, al cual se le aplicó un crecimiento conforme a los pronósticos de inflación subyacente obtenidos de la encuesta de expectativas de los analistas de Banxico (\textcolor{principal}{\textit{Figura 16}), para así estimar una proyección de ventas adecuada para el SKU sujeto de análisis.  Se presentan a continuación los resultados:\\

\begin{figure}
\centering
\includegraphics[width=\textwidth]{../0.imagenes/rfr_03}
\end{figure}

\textcolor{principal}{Proyección financiera del NOPAT 2023-2027.} El perito valuador llevó  una proyección del flujo de operación neto  del negocio (SKU) (\textcolor{principal}{\textit{Figura 17}}), para la aplicación del modelo EVA, partiendo del análisis financiero mencionado en el inciso correspondiente. Se presentan a continuación los resultados:\\

\begin{figure}
\centering
\includegraphics[width=\textwidth]{../0.imagenes/rfr_04}
\end{figure}

\subsubsection{Estimación de la Tasa justa de mercado de Regalías (Fair Royalty Rate).}

Se obtuvo una muestra de mercado de acuerdos de licenciamiento de marca, con base en el análisis de comparabilidad aplicable al activo intangible sujeto de valuación, obteniéndose un resultado de \textcolor{principal}{5.321\%}\footnote{ Media aritmética de una muestra de 14 comparables.}  sobre \textcolor{principal}{ventas netas}; según se muestra en la \textcolor{principal}{\textit{Figura 18}} y a continuación:

\begin{figure}
\centering
\includegraphics[width=\textwidth]{../0.imagenes/rfr_05}
\end{figure}

\textcolor{principal}{Proyección financiera del Pago de Regalías por SKU.}  El perito valuador llevó  una proyección del flujo de regalías del negocio (SKU) \textcolor{principal}{\textit{Figura 19}}; como base para la aplicación del modelo de Flujos Descontados de Regalías.  Se presentan a continuación los resultados:\\

\begin{figure}
\centering
\includegraphics[width=\textwidth]{../0.imagenes/rfr_06}
\end{figure}

\textcolor{principal}{Estimación de los elevadores de valor.} El valuador desarrolló una proyección con base en los elevadores de valor del negocio (\textit{Ceteris paribus}), haciendo énfasis en los siguientes premisas: i) Tasa de crecimiento de ingresos: 11.34\% Anual (CAGR 2022-2027). ii) Tasa fiscal (MTR): 30.00\% (vigente).  iii) WACC: 13.14\%. iv) Margen Nopat: 19.82\% , v) Deuda Exp. / (D + C): 50.93\%. \\

\subsubsection{MÉTODO DE FLUJOS DESCONTADOS DE REGALÍAS. RFR en la modalidad Greenfield.}

 El valuador llevó a cabo la proyección de flujo de pago de regalías del titular del activo intangible (Figuras 20 y 21), tomando como base las premisas analizadas y desarrolladas en el inciso anterior:
 
 \begin{figure}
\centering
\includegraphics[width=\textwidth]{../0.imagenes/rfr_07}
\end{figure}

\textcolor{principal}{Estimación del valor terminal del SKU.}\\

Se realizó bajo la premisa de que se está en presencia de un negocio con la firme intención de continuar con sus operaciones en el mediano y largo plazo. Conforme al marco teórico de las finanzas corporativas y los lineamientos internacionales de valuación, resulta adecuado capitalizar un flujo terminal  con posterioridad al horizonte explícito de proyección, mediante una perpetuidad creciente, con un resultado según se aprecia en la Figura 22. Se muestran a continuación los resultados, con una cantidad de \$35.2 mdp:

 \begin{figure}
\centering
\includegraphics[width=\textwidth]{../0.imagenes/rfr_08}
\end{figure}

\subsubsection{INDICADOR DE VALOR POR MÉTODO DE FLUJOS DESCONTADOS DE REGALÍAS. }

Se analizó el modelo mencionado con proyección a 5 años y a valor terminal, de acuerdo con la información financiera presentada por el solicitante, mismo modelo que cuenta con amplia aceptación en las finanzas a nivel mundial, con un resultado en cifras redondas al 31/12/2022 de \$25’385,845.00 MXN:

 \begin{figure}
\centering
\includegraphics[width=\textwidth]{../0.imagenes/rfr_09}
\end{figure}

El valuador le asignó a este método de valuación una ponderación del 70.00\% (setenta por ciento) total en el valor conclusivo de la firma, como una regla empírica del medio bursátil y financiero, en vista del amplio reconocimiento que tiene este modelo en el boletín C-8 de la NIF y  la NIIF 38.

\end{leftcolumn}

\begin{rightcolumn}

 \begin{figure}
\centering
\includegraphics[width=\textwidth]{../0.imagenes/rfr_11}
\end{figure}

 \begin{figure}
\centering
\includegraphics[width=\textwidth]{../0.imagenes/rfr_12}
\end{figure}

 \begin{figure}
\centering
\includegraphics[width=\textwidth]{../0.imagenes/rfr_13}
\end{figure}

 \begin{figure}
\centering
\includegraphics[width=\textwidth]{../0.imagenes/rfr_14}
\end{figure}

 \begin{figure}
\centering
\includegraphics[width=\textwidth]{../0.imagenes/rfr_15}
\end{figure}

 \begin{figure}
\centering
\includegraphics[width=\textwidth]{../0.imagenes/rfr_16}
\end{figure}

 \begin{figure}
\centering
\includegraphics[width=\textwidth]{../0.imagenes/rfr_17}
\end{figure}

 \begin{figure}
\centering
\includegraphics[width=\textwidth]{../0.imagenes/rfr_18}
\end{figure}

 \begin{figure}
\centering
\includegraphics[width=\textwidth]{../0.imagenes/rfr_19}
\end{figure}

 \begin{figure}
\centering
\includegraphics[width=\textwidth]{../0.imagenes/rfr_110}
\end{figure}

 \begin{figure}
\centering
\includegraphics[width=\textwidth]{../0.imagenes/rfr_111}
\end{figure}

 \begin{figure}
\centering
\includegraphics[width=\textwidth]{../0.imagenes/rfr_112}
\end{figure}

 \begin{figure}
\centering
\includegraphics[width=\textwidth]{../0.imagenes/rfr_113}
\end{figure}

 \begin{figure}
\centering
\includegraphics[width=\textwidth]{../0.imagenes/rfr_114}
\end{figure}

 \begin{figure}
\centering
\includegraphics[width=\textwidth]{../0.imagenes/rfr_115}
\end{figure}

 \begin{figure}
\centering
\includegraphics[width=\textwidth]{../0.imagenes/rfr_116}
\end{figure}

 \begin{figure}
\centering
\includegraphics[width=\textwidth]{../0.imagenes/rfr_117}
\end{figure}


\end{rightcolumn}