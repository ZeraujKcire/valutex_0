\begin{rightcolumn}

\subsubsection{CIFRAS CONCLUSIVAS. Valor Ponderado del Activo Intangible.}

 Después de haberse realizado el análisis del valor razonable de la marca conforme a la aplicación de 2 (dos) modelos financieros descritos en el desarrollo del presente capítulo (Figuras 28 y 29), el valuador llevó a cabo para el presente análisis conclusivo la ponderación de cada uno de los modelos,  con el objeto del estimar el valor razonable del activo intangible, como capital invertido en uso a valor de mercado,  con un resultado en cifras redondas al 31/12/2022 de \$25'987,401.00 MXN:\\
 
 \begin{figure}[H]
 	\centering
	\includegraphics[width=\textwidth]{../0.imagenes/valor_ponderado_activo_intangible_01}
 \end{figure}
 
 A su vez y conforme al marco teórico de las valuaciones de activos intangibles, se le adicionó al valor ponderado de la marca el valor presente del beneficio fiscal de la amortización.\\

\textcolor{principal}{Estimación del beneficio por amortización fiscal.} La teoría de valuaciones asume que los activos intangibles podrían ser vendidos individualmente o agrupados con otros activos en una transacción. Para estos casos la Ley de Impuesto sobre la Renta de México permite que ciertos intangibles sean amortizados disminuyendo la base gravable y por lo tanto el pago de impuestos, creando un escudo fiscal  (Figura 30); con un resultado de \$2'715,880.00 MXN (Figura 31).

\subsubsection{VALOR RAZONABLE DE LA MARCA}

Con base en lo anterior, se obtuvo un resultado de estimación del valor de la marca, conforme a los modelos descritos anteriormente de:

 \begin{figure}[H]
 	\centering
	\includegraphics[width=\textwidth]{../0.imagenes/valor_ponderado_activo_intangible_02}
 \end{figure}
 
 \begin{center}
Valor de la marca Candy Laser \textregistered al 31/12/2022:
\$28'703,281.00 MXN\\

Veintiocho Millones, setecientos tres mil, doscientos ochenta y un pesos 00/100 M.N.
\end{center}



\end{rightcolumn}

\begin{leftcolumn}

\end{leftcolumn}