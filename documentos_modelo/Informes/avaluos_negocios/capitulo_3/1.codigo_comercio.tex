
\textcolor{principal}{Código de Comercio:}\\[10pt]

\textit{``Artículo 1252.- Los peritos deben tener título en la ciencia, arte, técnica, oficio o industria a que pertenezca la cuestión sobre la que ha de oírse su parecer, si la ciencia, arte, técnica, oficio o industria requieren título para su ejercicio.}\\[10pt]

\textit{Si no lo requirieran o requiriéndolo, no hubiere peritos en el lugar, podrán ser nombradas cualesquiera personas entendidas a satisfacción del juez, aun cuando no tengan título.}\\

\textit{La prueba pericial sólo será admisible cuando se requieran conocimientos especiales de la ciencia, arte, técnica, oficio o industria de que se trate, más no en lo relativo a conocimientos generales que la ley presupone como necesarios en los jueces, por lo que se desecharán de oficio aquellas periciales que se ofrezcan por las partes para ese tipo de conocimientos, o que se encuentren acreditadas en autos con otras pruebas, o tan sólo se refieran a simples operaciones aritméticas o similares.} \\[10pt] 

\textit{El título de habilitación de corredor público acredita para todos los efectos la calidad de perito valuador.''} (...)\\[10pt]

\textit{``Artículo 1257.- Los jueces podrán designar peritos de entre aquéllos autorizados como auxiliares de la administración de justicia por la autoridad local respectiva, o a solicitar que el perito sea propuesto por colegios, asociaciones o barras de profesionales, artísticas, técnicas o científicas o de las instituciones de educación superior públicas o privadas, o las cámaras de industria, comercio, o confederaciones de cámaras a la que corresponda al objeto del peritaje.  (...)''}\\[10pt]

\textit{``Artículo 1300.- Los avalúos harán prueba plena.}\\[10pt]

