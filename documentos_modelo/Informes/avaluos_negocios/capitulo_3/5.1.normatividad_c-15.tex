\textcolor{principal}{NORMATIVIDAD BOLETÍN C-15 E IAS 36:}\\[10pt]

``\textit{Los objetivos principales de las normas son los siguientes:}\\

\begin{enumerate}[\itshape (a)]

\item \textit{Proporcionar criterios para la identificación de evidencias respecto a un posible deterioro en el valor de los activos de larga duración, tangibles e intangibles.}

\item \textit{Definir reglas para el cálculo y reconocimiento de pérdidas por deterioro de activos y su reversión.}

\item \textit{Establecer reglas de presentación y revelación en los casos en que se haya presentado deterioro y aquellas aplicables a la descontinuación de operaciones.}
\end{enumerate}


\textit{Aplicación de las normas:}\\

\textit{Las normas son aplicables a todos los activos de larga duración, tangibles e intangibles, incluyendo el crédito mercantil.}\\

\textit{Los activos de larga duración son aquéllos que permanecen en el largo plazo, necesarios para la operación de una entidad de los que se espera la generación de beneficios económicos futuros o que, adquiridos con esos fines, se decide su disposición. (...)}”\\


