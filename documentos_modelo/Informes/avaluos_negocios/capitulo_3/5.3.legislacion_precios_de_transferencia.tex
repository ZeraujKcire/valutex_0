\subsection{LEGISLACIÓN EN PRECIOS DE TRANSFERENCIA}

Las disposiciones en materia de precios de transferencia que deben acatar las empresas establecidas o con operaciones en los Estados Unidos Mexicanos emanan de la LISR, particularmente en los artículos 179 y 180. La LISR considera que dos o más personas son partes relacionadas cuando una participa de manera directa o indirecta en la administración, control o capital de la otra, o bien, cuando una persona o grupo de personas participen de manera directa o indirecta en la administración, control o capital de dichas personas.\\

A partir de enero de 1997, la LISR establece que los contribuyentes que celebren operaciones con partes relacionadas están obligados para efectos de dicha Ley, a determinar sus ingresos acumulables y deducciones autorizadas, considerando para esas operaciones los precios y montos de contraprestaciones que hubieran utilizado con o entre partes independientes en operaciones comparables. Esto es, que en sus transacciones con partes relacionadas las empresas cumplan con el principio \textit{arm's-length}. En caso contrario, la LISR especifica que las autoridades fiscales pueden determinar los ingresos acumulables y deducciones autorizadas de los contribuyentes mediante la determinación del precio o monto de la contraprestación en operaciones celebradas entre partes relacionadas, considerando para ello, los precios y montos de contraprestaciones que hubieran utilizado partes independientes en operaciones comparables.\\

En el caso de los contribuyentes que celebren operaciones con partes relacionadas en el extranjero, la LISR establece -en el artículo 76 fracción IX- que éstos están obligados a obtener y conservar la documentación comprobatoria. Dicha documentación deberá dar el soporte para demostrar que el monto de sus ingresos y deducciones se efectuaron de acuerdo con los precios o montos de contraprestaciones que hubieran utilizado partes independientes en operaciones similares.\\

Asimismo, la LISR considera como comparables a aquellas operaciones o empresas cuyas diferencias no afecten significativamente el precio o monto de la contraprestación o el margen de utilidad a que hacen referencia los métodos establecidos en el artículo 180 de la LISR. \\

En el caso en que existan diferencias significativas, éstas pueden ser eliminadas mediante ajustes razonables. La determinación de dichas diferencias considera los siguientes elementos:\\

\begin{enumerate}[1.]
\item Las características de las operaciones, incluyendo:
\begin{enumerate}[i)]
\item En el caso de operaciones de financiamiento, elementos tales como el monto principal, plazo, garantía, solvencia del deudor y tasas de interés;
\item En el caso de prestación de servicios, elementos tales como la naturaleza del servicio, y si el servicio involucra o no una experiencia o conocimiento técnico;
\item  En caso de uso, goce o enajenación de bienes tangibles, elementos tales como las características fiscales, calidad y disponibilidad del bien; y
\item  En el caso de que se conceda la explotación o transferencias se transmita un bien intangible (patente, marca, nombre comercial o transferencia de tecnología), la duración y el grado de protección.

\end{enuemrate}

\item Las funciones o actividades, incluyendo los activos utilizados y riesgos asumidos en las operaciones de cada una de las partes involucradas en la operación;

\item Los términos contractuales;

\item Las circunstancias económicas; 

\item Las estrategias de negocios, incluyendo las relaciones con la penetración, permanencia y ampliación del mercado.

\end{enumerate}

El artículo 180 de la LISR provee métodos para determinar si las transacciones entre dos o más empresas cumplen con el principio \textit{arm's length}. Estos métodos se enuncian a continuación:
\begin{enumerate}[1.]
\item  Método de Precio Comparable No Controlado (MPC)
\item  Método de Precio de Reventa (MPR)
\item  Método de Costo Adicionado (MCA)
\item  Método de Partición de Utilidades (MPU)
\item  Método Residual de Participación de Utilidades (MRPU)
\item  Método de los Márgenes Transaccionales de Utilidad de Operación (MMTUO)
\end{enumerate}

\subsubsection*{Factores de Comparabilidad}

En general, dos transacciones no necesitan ser idénticas para ser consideradas como comparables, pero deberán ser suficientemente similares para obtener un resultado arm’s length confiable. Si existieran diferencias materiales entre las transacciones controladas y las no controladas (inexactitud de comparables), pueden aplicarse ajustes a los factores que las provocan, teniendo como objetivo mejorar la calidad de los resultados.\\

La magnitud y racionalidad de algún ajuste afectará la fiabilidad del análisis. Los siguientes factores determinan la comparabilidad de transacciones controladas y no controladas.\\

\subsubsection*{Funciones}

Una evaluación del grado de comparabilidad entre transacciones controladas y no controladas requieren de un ``Análisis Funcional'' que examina las actividades significativas en términos económicos, bajo condiciones tanto controladas como no controladas para el contribuyente, y que considere los recursos empleados en conjunción con las actividades bajo estudio.\\ 

Para que dos transacciones sean funcionalmente comparables, las entidades deben tener funciones similares con respecto a las transacciones que llevan a cabo. \\

Generalmente, las funciones que deben ser analizadas incluyen: investigación y desarrollo, diseño de productos e ingeniería, manufactura, producción y procesos de ingeniería, fabricación del producto, extracción, y maquila o ensamble, compra y manejo de materiales, funciones de mercado y distribución, incluyendo manejo de inventarios, administración de la garantía, y actividades publicitarias; transportación y almacenaje; y gestión, legal, contable y financiera, crédito y cobranza, entrenamiento, y servicios de administración de personal.\\


\subsubsection*{Términos Contractuales}

Para determinar el grado de comparabilidad entre una transacción controlada y otra no controlada también se requiere de una comparación de los términos contractuales significativos que puedan afectar los resultados de las dos transacciones. Los términos que deben ser considerados incluyen la forma de considerar las transacciones cargadas o pagadas, el volumen de compras y ventas, la amplitud y los términos de las garantías otorgadas, actualización de derechos, revisiones o modificaciones; la duración de las licencias, contratos u otros acuerdos, y la terminación o renegociación de los derechos; transacciones colaterales o relaciones de negocios entre el comprador y el vendedor, incluyendo acuerdo para la provisión de servicios subsidiarios; y la extensión en términos de créditos y pagos. \\

Los términos contractuales, incluyen la asignación de los riesgos entre las partes, que son acordados por escrito antes de que las transacciones ocurran. Generalmente éstas serán respetadas si reflejan la sustancia económica de las transacciones.\\

\subsubsection*{Riesgos}

Para determinar el grado de comparabilidad entre transacciones controladas y no controladas, es necesaria una comparación de los riesgos que podrían afectar los precios cargados o las ganancias obtenidas de las transacciones. Los riesgos que deben ser considerados incluyen: riesgos del mercado, considerando fluctuaciones de costo, demanda, precios, y niveles de inventario; riesgos asociados con los éxitos o fallas en las actividades de investigación y desarrollo; riesgos financieros, incluyendo fluctuaciones en las tasas de cambio externas y los tipos de interés; riesgos asociados a los créditos y cobranza, riesgos en la calidad de los productos, y riesgos generales de los negocios relacionados con la titularidad de la propiedad de planta y equipo. \\

Considerando que los contribuyentes asignan esos riesgos en un contrato, y actúan de acuerdo con él, cada asignación de los riesgos será respetada, a menos que el contrato sea ejecutado después de que el impacto de los riesgos es conocido o claramente previsible. \\

En caso de no existir una asignación clara de los riesgos, los siguientes factores serán considerados en la determinación de qué parte absorberán el riesgo:\\

\begin{enumerate}[i.]

\item 	Si la conducta de las partes es consistente a lo largo del tiempo.
\item Si el contribuyente controlado cargase en última instancia las consecuencias de un riesgo.
\item La consideración de que cada contribuyente controla cualquier actividad que influya el resultado de un riesgo en particular.\\

\subsubsection*{Condiciones Económicas}

Determinar el grado de comparabilidad entre transacciones controladas y no controladas requiere una comparación de las condiciones económicas que podrían afectar los precios cargados o las ganancias obtenidas en las dos transacciones. \\

Cada condición económica incluye: la similitud geográfica de los mercados, el nivel de mercado, las secciones relevantes del mercado para los productos, propiedades, servicios transferidos provistos, la colocación específica de los costos de los factores de producción transferidos o provistos, la colocación específica de los costos de los factores de producción y distribución, la extensión de la competencia en cada mercado con respecto a las propiedades o servicios bajo revisión, las condiciones económicas de la industria en particular, incluyendo si los mercados están en contracción o en expansión, y las alternativas realmente posibles para el comprador y el vendedor.\\

\subsubsection*{Estrategias de Ampliación del Mercado}

En algunas circunstancias, los contribuyentes podrían incrementar temporalmente los gastos en el desarrollo de mercados o reducir los precios de reventa con el objetivo de entrar a nuevos mercados o de incrementar la presencia de los productos de un mercado existente. De acuerdo con esto, los precios cargados, o las ganancias realizadas en una transacción controlada podrían ser diferentes de un precio arm’s length durante la implementación de estas estrategias de ampliación de mercados.\\

La reducción en los precios o las ganancias son aceptables para propósitos de los precios de transferencia sólo si se puede demostrar que un contribuyente no controlado implicado en una estrategia comparable haría lo mismo bajo circunstancias comparables por un periodo de tiempo comparable. \\

Además, los contribuyentes deben demostrar que: los costos incurridos por la implementación de la estrategia de ampliación de los mercados han surgido como consecuencia de la esperanza del contribuyente controlado de obtener ganancias futuras y que dichas ganancias serán suficientes para cubrir los costos que resultaron de la implementación de la estrategia. \\

Se entiende que la estrategia de ampliación de mercados es llevada a cabo sólo por un periodo razonable de tiempo; y dicha estrategia considera, los costos relacionados y las ganancias esperadas. Además, cualquier acuerdo entre los contribuyentes controlados para repartir los costos debe ser establecido antes de la implementación de la estrategia.\\

\subsubsection*{Diferentes Mercados Geográficos}

Si la información respecto a las transacciones de las comparables no relacionadas no está disponible en el mismo mercado geográfico en el que operan los contribuyentes relacionados, se puede considerar como comparable una transacción entre partes no relacionadas aun cuando éstas operen en un mercado geográfico distinto. \\

Sin embargo, algunos ajustes deben ser hechos para considerar cualquier diferencia entre los dos mercados. Si no hay información disponible para ajustar las diferencias, puede utilizarse la información del mercado más similar donde haya transacciones entre partes no relacionadas que se puedan considerar como comparables. Siempre considerando que cualquier diferencia podría afectar el grado de comparabilidad entre las transacciones controladas y las no controladas.\\


\subsubsection*{Transacciones no Consideradas como Comparables}

Un conjunto de transacciones no será considerado como una medida confiable de un resultado arm’s length si ellas no están conducidas en el curso ordinario de los negocios, o si un propósito principal de las transacciones no controladas fue establecer un rango arm’s length con respecto de las transacciones controladas.\\

 
