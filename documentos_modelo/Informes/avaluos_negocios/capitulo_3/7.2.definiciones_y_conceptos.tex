\textcolor{principal}{DEFINICIONES Y CONCEPTOS.}

\begin{enumerate}[\indent\itshape a)]
\item \textcolor{principal}{\textit{CAPITAL CONTABLE }}

\textit{Es la diferencia entre los activos y pasivos de la empresa y está constituido por la suma de todas las cuentas de capital, es decir, incluye capital social, reservas, utilidades acumuladas y utilidades del ejercicio.}

\item \textcolor{principal}{\textit{CAPITAL INVERTIDO}}

\textit{Son los bienes que constituyen el Activo Tangible de una Sociedad. El Capital Invertido por lo general refleja el desembolso realizado por los inversionistas para iniciar una Empresa y las adiciones de Capital realizadas durante su funcionamiento. Regularmente su cálculo corresponde al Activo Total menos Pasivo Circulante o en segunda forma al Activo No Circulante más Capital de Trabajo.}

\item  \textcolor{principal}{\textit{FECHA DE REPORTE}}

\textit{Corresponde a la fecha en que fue realizado y firmado el documento de valuación. Puede ser igual o distinta a la fecha de valores.}

\item  \textcolor{principal}{\textit{FECHA DE VALORES}}

\textit{Es la fecha que el valuador asentar\'a al momento del cierre de valores en su trabajo valuatorio. Puede ser igual o distinta a la fecha del reporte.}

\item \textcolor{principal}{ \textit{PROP\'OSITO DEL AVAL\'UO}}

\textit{Es la intenci\'on expresa de determinar un tipo de valor que ser\'a estimado en funci\'on de los bienes a valuar, a la especialidad valuatoria y al uso del aval\'uo se\~nalado por el solicitante.}

\item \textcolor{principal}{\textit{USO DEL AVAL\'UO}}

\textit{Es el destino que se le pretende dar al dictamen y que expresamente se\~nala el solicitante del servicio.}


\item  \textcolor{principal}{\textit{VALOR EN LIBROS}}

\textit{Es el importe con que un rengl\'on contable aparece registrado en los libros de contabilidad, ya sea que represente el costo, inicial, el actualizado, el estimado o el de aval\'uo. Representa el valor con que se registra en los libros de contabilidad cualquier propiedad, derecho, bien, cr\'edito u obligaci\'on. El valor en libros representa \'unicamente ``cifras en libros'' y eso puede ser diferente del valor comercial, del valor en el mercado, del valor real, del valor de reposici\'on, del valor de liquidaci\'on, etc.}

\item \textcolor{principal}{\textit{VALOR DE REALIZACI\'ON ORDENADA}}

\textit{Es el precio estimado que podr\'ia ser obtenido a partir de una venta en el mercado libre, en un periodo de tiempo apenas suficiente para encontrar un comprador o compradores, en donde el vendedor tiene urgencia de vender, donde ambas partes act\'uan con conocimiento y bajo la premisa de que los bienes se venden en el lugar y en el estado en que se encuentran.}

\item \textcolor{principal}{ \textit{VALOR DE LIQUIDACI\'ON FORZADA}}

\textit{Es la cantidad bruta, expresada en t\'erminos monetarios, que se espera obtener por concepto de una venta p\'ublica debidamente anunciada y llevada a cabo en el mercado abierto, en la que el vendedor se ve en la obligaci\'on de vender de inmediato por mandato judicial ``tal como est\'a y donde se ubica'' el activo. En algunos casos, puede involucrar un vendedor no deseoso y un comprador o compradores que compran con conocimiento de la desventaja para el vendedor.}

\item \textcolor{principal}{ \textit{VALOR RAZONABLE}}

\textit{Conforme a su definici\'on insertada en la publicaci\'on de NIIF, (IAS) International Accounting Standards Committee Foundation y por el (IMCP) Instituto Mexicano de Contadores P\'ublicos, que a la letra indica (sic)  `...El importe por el que puede ser intercambiado un activo o cancelado un pasivo, entre partes interesadas y debidamente informadas, en una transacción realizada en condiciones de independencia mutua…''}

\end{enumerate}
