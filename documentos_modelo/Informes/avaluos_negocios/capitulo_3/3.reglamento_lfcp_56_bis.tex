

\textcolor{principal}{REGLAMENTO DE LA LEY FEDERAL DE CORREDUR\'IA P\'UBLICA}\\


``\textcolor{secundario}{ARTICULO 56 Bis.-} \textit{El corredor p\'ublico en ejercicio de sus funciones como perito valuador, podr\'a estimar, cuantificar y valorar los bienes, servicios, derechos y obligaciones que se sometan a su consideraci\'on por nombramiento privado o por mandato de autoridad competente.} \\

\textit{El informe de valuaci\'on debe formularse de manera clara y objetiva, presentando el razonamiento y la informaci\'on suficiente con las cuales se obtiene el valor conclusivo del bien, servicio, derecho u obligaci\'on, y deber\'a contener cuando menos los siguientes rubros enunciativos: }

\begin{enumerate}[a)]

\item Nombre completo, n\'umero y plaza del Corredor as\'i como su firma y sello; 
\item Datos del solicitante; 
\item Datos del propietario, indicando en su caso la informaci\'on en que se basa; 
\item Tipo de servicio de valuaci\'on; 
\item Vigencia del aval\'uo, siendo este requisito obligatorio cuando exista una disposici\'on legal que as\'i lo establezca; 
\item Descripci\'on del bien, derecho, servicio u obligaci\'on materia del aval\'uo; 
\item  Cuando proceda, ubicaci\'on del bien materia de la valuaci\'on; 
\item Prop\'osito del informe de valuaci\'on; 
\item Uso del informe de valuaci\'on; 
\item Consideraciones previas a la valuaci\'on; 
\item Descripci\'on de enfoques de valuaci\'on aplicados; 
\item Fecha de la Inspecci\'on; 
\item En su caso fecha de referencia de valor; 
\item Fecha del informe de valuaci\'on; 
\item Fuentes de informaci\'on; 
\item Consideraciones previas a la conclusi\'on; 
\item Conclusi\'on de valor; 
\item Reporte fotogr\'afico, y 
\item En su caso, anexos. 

\end{enumerate}

\textit{Cualquier observaci\'on respecto a enfoques, fuentes de informaci\'on, elementos, limitaciones generales, entre otros, que incidan en la conclusi\'on del valor, deber\'an ser mencionadas en el informe.} \\

\textit{Cuando en raz\'on del servicio valuatorio, territorio, prop\'osito, uso u objeto del dictamen solicitado al corredor se desprenda que, con base en una normatividad particular expedida por autoridad competente que sea de car\'acter obligatorio, deba expedir o elaborar el dictamen utilizando leyes, normas, lineamientos, manuales o reglas espec\'ificas, el corredor podr\'a optar por sujetarse \'unicamente a dicha normatividad.} \\

\textit{Trat\'andose de remates, aval\'uos para efectos judiciales o procedimientos administrativos o aval\'uos solicitados por autoridades donde sea f\'isica o materialmente imposible realizar la inspecci\'on f\'isica del bien objeto de valuaci\'on, u obtener del solicitante o propietario la documentaci\'on correspondiente, se deber\'a se\~nalar expresamente en el dictamen y se realizar\'a el aval\'uo con los datos e informaci\'on de los que pueda allegarse el Corredor en el momento con los medios a su alcance.}''\\


