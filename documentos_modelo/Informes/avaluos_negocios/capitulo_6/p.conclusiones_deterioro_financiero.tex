El deterioro financiero provocado en NAVALMEX por la calificación crediticia en buró de crédito (APÉNDICE 2) incrementó notablemente la percepción de riesgo de la empresa para las instituciones financieras. Lo anterior conlleva un menor acceso para su Dirección financiera a recursos crediticios; con tasas de interés más onerosas para la obtención de financiamientos. Dicho fenómeno repercute en términos financieros en una reducción de la liquidez disponible para la operación, un menor flujo de caja, menores utilidades y dividendos para los accionistas, así como proyecciones de valor futuro menores a las normalmente alcanzables. \\

Por lo anterior y según quedó explicado en el \autoref{cap:5} de este dictamen, el valor razonable del deterioro financiero de ``Navalmex'', con fecha de valores al 31/12/2022:

\begin{figure}[H]
\centering
\textcolor{principal}{\textbf{\$487’280,547.00 MXN}}\\
(Cuatrocientos ochenta y siete millones, doscientos ochenta mil, \\
quinientos cuarenta y siete pesos 00/100 M.N.)\\[10pt]

\includegraphics[width=8cm]{../0.imagenes/logo}

\end{figure}



