
Esta opinión de precios de transferencia tuvo como objeto revisar la siguiente operación llevada a cabo por la compañía con su parte relacionada.\\

\begin{enumerate}[1.]
\item \textcolor{principal}{\textbf{Intereses devengados a favor y a cargo.}\\

Durante 2022 TDN recibió ingresos por concepto de intereses devengados a favor con sus partes relacionadas residentes en México, derivado de diversos financiamientos denominados en pesos. El método seleccionado para evaluar si dicha operación se encuentra a valores de mercado fue el Método PC.\\

Como se puede observar, considerando el rango de intereses obtenidos, se puede concluir que las tasas de interés pactadas por TDN con sus partes relacionadas, se encuentran dentro de los rangos Intercuartiles de mercado, por lo que TDN en la operación de Intereses devengados a favor, cumple con el principio de plena competencia, así como con lo establecido en los artículos 76 fracción XII, 179 y 180 de la LISR vigentes durante 2022.\\

\item \textcolor{principal}{\textbf{Pago por subarrendamiento.}}\\

Como se muestra en la Tabla II, considerando los precios por metro cuadrado pactados en subarrendamientos comparables, se puede establecer un rango que va de \$135.375 en el primer cuartil y \$150.41 en el tercer cuartil, mientras que TDN pagó \$138.93 por metro cuadrado a su parte relacionada residente en México.