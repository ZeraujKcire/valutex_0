\section*{PRESENTACIÓN}

\textcolor{principal}{\empresaSolicitante}\\ 

\textcolor{principal}{At'n.- \personaSolicitante}\\ 
\textcolor{principal}{\caracterSolicitante}\\

\begin{minipage}{.4\textwidth}
\ubicacionBien\\[10pt]
\end{minipage}


Estimado \textcolor{principal}{\personaSolicitante}:\\[10pt]

En fechas recientes nos fue solicitada a \textcolor{principal}{VALUAMI}, como empresa especializada en consultoría financiera, la determinación del \textcolor{principal}{VALOR RAZONABLE DEL DETERIORO FINANCIERO DE LOS ACTIVOS DE LARGA DURACIÓN PROVOCADO POR LA REDUCCIÓN SIGNIFICATIVA DE LAS OPERACIONES},  de la sociedad \textcolor{principal}{\empresaSolicitante}, en lo sucesivo \textcolor{principal}{\empresaCorto} con fecha de valores al \textcolor{principal}{\fechaValores} de conformidad con la información financiera, legal y de negocio presentada por el solicitante a la fecha de la valuación.\\


Por lo anterior se procedió de acuerdo a lo establecido por \textcolor{principal}{Norma C-15 DETERIORO EN EL VALOR DE LOS ACTIVOS DE LARGA DURACIÓN}, publicado por las normas de información financiera mexicanas; de acuerdo a los siguientes criterios:

``\textit{Los objetivos principales de las normas son los siguientes:}\\

\begin{enumerate}[\itshape (a)]

\item \textit{Proporcionar criterios para la identificación de evidencias respecto a un posible deterioro en el valor de los activos de larga duración, tangibles e intangibles.}

\item \textit{Definir reglas para el cálculo y reconocimiento de pérdidas por deterioro de activos y su reversión.}

\item \textit{Establecer reglas de presentación y revelación en los casos en que se haya presentado deterioro y aquellas aplicables a la descontinuación de operaciones.}
\end{enumerate}


\textit{Aplicación de las normas:}\\

\textit{Las normas son aplicables a todos los activos de larga duración, tangibles e intangibles, incluyendo el crédito mercantil.}\\

\textit{Los activos de larga duración son aquéllos que permanecen en el largo plazo, necesarios para la operación de una entidad de los que se espera la generación de beneficios económicos futuros o que, adquiridos con esos fines, se decide su disposición. (...)}”\\


Una vez habiéndose determinado las UGE's sujetas a deterioro,  se procedió a estimar el \textcolor{principal}{Precio Neto de Venta} y el \textcolor{principal}{Valor en Uso} de las UGE's sujetas de deterioro.

El valor razonable del deterioro financiero de los Activos de Larga Duración provocado por las reducciones significativas en las operaciones de \textcolor{principal}{\empresaSolicitante}, con fecha de valores al \fechaValores{} es por la cantidad de:


\begin{center}
\begin{minipage}{8cm}
\textcolor{principal}{\$\valorDeterioro{} \monedaCode}

\textcolor{principal}{\valorDeterioroLetra{} \moneda{} 00/100 M.N.}

\end{minipage}
\quad
\begin{minipage}{4cm}
\includegraphics[width=3cm]{../0.imagenes/logo_1}
\end{minipage}

\begin{figure}[H]
\centering
\includegraphics[width= 9cm]{../0.imagenes/cuadro_conclusivo_1}
\end{figure}

\end{center}

Los resultados detallados de la valuación del deterioro financiero de los Activos de Larga Duración se describe en los siguientes capítulos del presente dictamen. Es importante resaltar que el valor razonable obtenido está fuertemente influenciado por la información financiera histórica y que fue proporcionado por el solicitante a \textcolor{principal}{Valuami}.\\ 

El solicitante manifiesta que el uso que se le dará al presente dictamen valuatorio será para \textcolor{principal}{\textit{\usoAvaluo}}; no debiéndose asumir ningún otro uso que no se encuentre debidamente informado a \textcolor{principal}{Valuami}.\\

El presente dictamen deja sin efectos cualquier otro preparado o discutido por \textcolor{principal}{Valuami} y el solicitante con anterioridad. Las condiciones limitantes y supuestos que implica el presente informe se incluyen en el desarrollo de la presente valuación.\\

De igual manera, desde ahora confirmamos nuestra independencia respecto del solicitante, lo cual ha permitido realizar un estudio imparcial; para lo cual manifestamos que la contraprestación cobrada a usted como su titular por la preparación del presente dictamen valuatorio no es contingente o se encuentra ligada al resultado del mismo.\\


\textcolor{principal}{En \lugarInforme, a \fechaInforme.}\\

\textcolor{principal}{Valuami.}\\

\begin{table}[H]
\centering
	\begin{tabular}{cm{1cm}c}
	\begin{minipage}{7cm}
	\begin{center}
		
		
		\rule{7cm}{.4pt}\\
		Mtro. \nombrePerito\\
		\textcolor{principal}{\descripcionFirmaPerito}
		
		
	\end{center}
	\end{minipage}&&
%	\begin{minipage}{7cm}
%	\begin{center}
%		
%		\rule{7cm}{.4pt}\\
%		Julio A. Ramos Galindo\\
%		Controlador
%		
%	\end{center}
%	\end{minipage}
	
	\end{tabular}
\end{table}


