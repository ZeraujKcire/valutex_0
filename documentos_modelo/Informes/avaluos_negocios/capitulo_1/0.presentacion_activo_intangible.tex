\section*{PRESENTACIÓN}

\textcolor{principal}{\empresaSolicitante}\\ 

\textcolor{principal}{At'n.- \personaSolicitante}\\ 
\textcolor{principal}{\caracterSolicitante}\\

\begin{minipage}{.4\textwidth}
\ubicacionBien\\[10pt]
\end{minipage}


Estimado \textcolor{principal}{\personaSolicitante}:\\[10pt]

En fechas recientes nos fue solicitada a \textcolor{principal}{TASVALÚO, S.A.P.I. de C.V.} (en lo sucesivo ``\textcolor{principal}{Tasvalúo}''), como empresa especializada en consultoría financiera, la determinación del \textcolor{principal}{VALOR RAZONABLE} del \descripcionBien, en lo sucesivo \textcolor{principal}{\empresaCorto} con fecha de valores al \textcolor{principal}{\fechaValores} de conformidad con la información financiera, legal y de negocio presentada por el solicitante a la fecha de la valuación.\\



Por lo anterior se aplicaron dos técnicas de valuación de negocios con amplia aceptación en el medio financiero: \textcolor{principal}{(i) El método de Flujos de Efectivo Descontados a los Accionistas (DCFE Valuation}) y \textcolor{principal}{ii) El modelo de valuación Relativa (PEERS)}. A su vez, \textcolor{principal}{Tasvalúo} utilizó una metodología de valuación de negocios basada en el punto número 2 del pentágono de explotación de oportunidades (Pentágono de McKinsey), conocido como valor actual interno. Lo anterior  implica que \textcolor{principal}{Tasvalúo} llevó a  cabo el análisis financiero del Negocio en Marcha y de su principal \tipoAvaluo{} en las condiciones que opera al día de hoy.\\


El valuador llevó a cabo la \textcolor{principal}{proyección financiera} bajo parámetros de mercado, con base en técnicas de valuación generalmente aceptadas y soportadas por el marco teórico de las finanzas corporativas. Posteriormente, se calcularon los flujos de efectivo en su modalidad para entidades financieras (FCFE) y el valor terminal de la firma a valor presente, mediante la aplicación de un factor de descuento  que impacte adecuadamente el riesgo intrínseco del negocio.\\

Una vez habiéndose obtenido el indicador del Negocio en Marcha bajo dos modelos financieros, se llevó a cabo el cálculo del \textcolor{principal}{valor razonable ponderado}, aplicándose el \textcolor{principal}{modelo DCFE} en un \textcolor{principal}{80\%} y el modelo PEERS en un \textcolor{principal}{20\%}, obteniéndose así el \textcolor{principal}{VALOR RAZONABLE del Capital Accionario de la sociedad \empresaSolicitante}; conforme al propósito y uso establecidos en este informe.\\

Se muestran a continuación los resultados del valor del negocio en marcha y de su capital accionario, con cifras al 31 de Diciembre de 2023:

\begin{figure}[H]
\centering
\includegraphics[width= 10cm]{../0.imagenes/cuadro_conclusivo_2}
\end{figure}

A su vez, se aplicaron dos técnicas de valuación de activos intangibles con amplia aceptación en el medio financiero: \textcolor{principal}{(i) El método de Ahorro en Regalías (Modelo RFR)} y \textcolor{principal}{ii) El modelo residual}. Para este fin, \textcolor{principal}{Tasvalúo} utilizó una metodología de valuación de negocios basada en el punto número 2 del pentágono de explotación de oportunidades (Pentágono de McKinsey), conocido como valor actual interno. Lo anterior  implica que \textcolor{principal}{Tasvalúo} llevó a  cabo el análisis financiero del activo intangible en las condiciones que opera al día de hoy.\\

Por lo anterior, el \textcolor{principal}{VALOR RAZONABLE de la marca \marca} cuyo titular es \textcolor{principal}{\textbf{\empresaCorto}},  al \fechaValores, en cifras redondeadas, es por la cantidad de:\\

\begin{center}
\begin{minipage}{8cm}
\textcolor{principal}{\$\valorActivoIntangible{} \monedaCode}

\textcolor{principal}{\valorActivoIntangibleLetra{} \moneda{} 00/100 M.N.}

\end{minipage}
\quad
\begin{minipage}{4cm}
\includegraphics[width=3cm]{../0.imagenes/logo_1}
\end{minipage}

\begin{figure}[H]
\centering
\includegraphics[width= 9cm]{../0.imagenes/cuadro_conclusivo_1}
\end{figure}

\end{center}

Los resultados detallados de la valuación del Negocio en Marcha y del \tipoAvaluo{} se describen en los siguientes capítulos del presente dictamen. Es importante resaltar que el valor razonable obtenido está fuertemente influenciado por la información financiera histórica y que fue proporcionado por el solicitante a \textcolor{principal}{Tasvalúo}.\\ 

El solicitante manifiesta que el uso que se le dará al presente dictamen valuatorio será para \textcolor{principal}{\textit{\usoAvaluo}}; no debiéndose asumir ningún otro uso que no se encuentre debidamente informado a \textcolor{principal}{Tasvalúo}.\\

El presente dictamen deja sin efectos cualquier otro preparado o discutido por \textcolor{principal}{Tasvalúo} y el solicitante con anterioridad. Las condiciones limitantes y supuestos que implica el presente informe se incluyen en el desarrollo de la presente valuación.\\

De igual manera, desde ahora confirmamos nuestra independencia respecto del solicitante, lo cual ha permitido realizar un estudio imparcial; para lo cual manifestamos que la contraprestación cobrada a usted como su titular por la preparación del presente dictamen valuatorio no es contingente o se encuentra ligada al resultado del mismo.\\


\textcolor{principal}{En \lugarInforme, a \fechaInforme.}\\

\textcolor{principal}{TASVALÚO.}\\

\begin{table}[H]
\centering
	\begin{tabular}{cm{1cm}c}
	\begin{minipage}{7cm}
	\begin{center}
		
		
		\rule{7cm}{.4pt}\\
		Mtro. Diego M. Perezcano\\
		Valuador
		
	\end{center}
	\end{minipage}&&
	\begin{minipage}{7cm}
	\begin{center}
		
		\rule{7cm}{.4pt}\\
		Julio A. Ramos Galindo\\
		Controlador
		
	\end{center}
	\end{minipage}
	
	\end{tabular}
\end{table}


