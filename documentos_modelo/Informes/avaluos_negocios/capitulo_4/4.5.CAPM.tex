\textcolor{secundario}{MODELO DE COSTO DE ACTIVOS DE CAPITAL (CAPM).} El fundamento del \gls{capm} establece que los accionistas deben de ganar por su inversi\'on al menos una tasa libre de riesgo, m\'as un diferencial que compense el riesgo sistem\'atico de dicha inversi\'on.\\[10pt]

A continuaci\'on se detalla la formulaci\'on del modelo:
\begin{center}
\begin{minipage}{8cm}
\begin{itemize}
\small
				\item El costo del capital accionario se estima utilizando el \textit{Capital Asset Pricing Model} (\gls{capm}):
				$$Rk=Rf+\beta\times(ERP)+SP+RP$$
				 \item Donde:
				 \item $Rf$= Tasa Libre de Riesgo
				 \item $ERP$= Prima de Riesgo del Mercado de Capitales
				 \item $\beta$=\gls{beta}
				 \item $SP$= Prima por tama\~no
				 \item $RP$ = Prima por Riesgo Pa\'is
			\end{itemize}
	\footnotesize{Fuente: Valuaci\'on de Activos Intangibles. DEAL ADVISORY MEXICO. KPMG C\'ARDENAS DOSAL, S.C. KPMG ``D.R.''\copyright 2016.}
		\end{minipage}
\end{center}

