\textcolor{principal}{\textbf{ENFOQUE F\'ISICO} }\\


Es el enfoque que estima el valor de un inmueble considerando los elementos necesarios para construir uno igual o similar. Los elementos que se toman en consdieraci\'on para la estimaci\'on del valor son: 

\begin{enumerate}[a)]

\item \textcolor{principal}{Valor del terreno}, mediante el análisis de mercado de terrenos comparables en zonas similares y a los cuales se les aplicar\'a el modelo de la homologaci\'on directa 

\item \textcolor{principal}{Valor de las construcciones} mediante un an\'alisis param\'etrico de construcciones similares presupuestadas en el periodo en el que se realice el aval\'uo, semejantes al sujeto valuado. A dicho valor param\'etrico se le demeritar\'a debido all desgaste, uso, deterioro o falta de mantenimiento que presente, para obtener as\'i el valor de las construcciones en su estado actual

\item \textcolor{principal}{Valor de las Obras Complementarias}, Instalaciones Especiales y Elementos Accesorios, mediante un an\'alisis param\'etrico de dichos elementos presupuestadas en el periodo en el que realice el aval\'uo semejantes a los que presente el sujeto valuado. A dichos valores param\'etricos de cada elementos se les demeritar\'a de igual forma que a las construcciones.

\end{enumerate}

\textcolor{principal}{El valor Físico = a)+ b) + c)}

Dicho enfoque es utilizado principalmente en inmuebles a\'iípicos o de uso espec\'ifico que no presentan mercado similar en la zona ya sea por sus caracter\'isticas o tipo de inmueble.\\

Con base a la experiencia en valuaci\'on de inmuebles e intangibles, los complejos hoteleros de esta \'indole obtiene su valor mediante los ingresos que produce. Por lo que se llevar\'a a cabo dicho enfoque como comprobaci\'on del valor obtenido mediante el enfoque de ingresos.\\

El enfoque f\'isico se basa en el principio de sustituci\'on el cual infiere que ninguna persona prudente pagar\'a por una propiedad m\'as que el precio de la tierra y el costo de construcci\'on.\\

\textcolor{principal}{Metodolog\'ia}

\begin{enumerate}[1.-]
\item Estimar el valor de la tierra mediante el mercado obtenido dentro de la zona con caracter\'isticas similares al inmueble; valuado mediante los m\'etodos de homologaci\'on y an\'alisis de mercado natural.

\item  Estimar el costo de \textcolor{principal}{Valor de Reposici\'on Nuevo (VRN)} de las construcciones. 

\item  Estimar la depreciaci\'on de las construcciones con base a la edad, estado de conservaci\'on, calidad y mantenimiento del sujeto.

\item Sustraer la depreciación al VRN de las construcciones para obtener el \textcolor{principal}{Costo de Reposici\'on Neto.}

\item Valor de la tierra + CRN  = Enfoque Físico (estimación de mercado).

\end{enumerate}



