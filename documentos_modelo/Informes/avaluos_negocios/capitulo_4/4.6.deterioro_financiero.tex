
\subsection{Análisis de Deterioro Financiero.}

\textcolor{principal}{\textbf{Deterioro financiero.}} Condición existente cuando los beneficios económicos futuros, es decir, el Valor de Recuperación de los Activos de Larga Duración en uso o en disposición son menores a su Valor Neto en Libros.\\

\textcolor{principal}{\textbf{Revisión de indicios de deterioro:}}\\

Existen indicios de un posible deterioro, dados por diversos eventos y circunstancias a considerar; de los cuales algunos de los más significativos para efectos de este estudio son las siguientes:\\

\begin{itemize}
\item Alguna modificación adversa de carácter legal o en el ambiente de un negocio que pueda afectar el valor de su capital invertido en uso.
\item Si existe una disminución significativa en el valor de mercado del negocio.
\item La pérdida de mercado de los productos o servicios que presta la entidad.
\item Cambios significativos en el destino o utilización los activos de una entidad.
\item La suspensión de algún derecho o licencia.
\item Pérdidas de operación o flujos de efectivo negativos en el periodo combinado con un historial o proyecciones de pérdidas, que confirman la tendencia de pérdidas continuas.
\item Pérdida bruta en la entidad o en alguno de sus componentes significativos.

\end{itemize}

\textcolor{principal}{\textbf{Reglas de Valuación.-}} Ante la presencia de alguno de los indicios de deterioro del valor de un negocio, como resulta el caso de la anotación al buró de crédito mencionada en el APÉNDICE B según lo declarado por el solicitante, se debe estimar el valor razonable de dicha pérdida por deterioro. \\

Para este efecto, es necesario determinar el Valor de Recuperación de la Unidad Generadora de Efectivo.\\

\subsubsection{Determinación del valor de recuperación}

Se define al \textcolor{principal}{\textbf{valor de recuperación}} de una unidad generadora de efectivo como el monto mayor entre el  precio neto de venta y su valor de uso.\\

Si cualquiera de los valores que confirman el Valor de Recuperación excede al Valor Neto en Libros de la UGE, se debe considerar que no hay deterioro.\\

\subsubsection{Determinación del precio neto de venta.} 

El \textcolor{principal}{\textbf{precio neto de venta}} es el monto verificable que se obtendría por la realización de la unidad generadora de efectivo entre partes interesadas y dispuestas en una transacción de libre competencia, menos su correspondiente costo de disposición; además, debe existir un mercado observable.\\

La mejor evidencia del precio neto de venta lo constituye la existencia de un precio dentro de un compromiso formal de venta, en una transacción libre, ajustado por los costos acumulados que pudieran ser directamente atribuibles a la disposición del activo.\\

Dado que la unidad generadora de efectivo se encuentra mantenida para su uso, no existe un compromiso formal  de venta, sin embargo, puede acceder a un mercado activo de compraventa observable y verificable; en este caso, el precio neto de venta estaría constituido por el precio en el mercado de los activos que conforman la unidad generadora de efectivo (UGE), menos los costos de disposición.\\ 

El precio de mercado adecuado es, normalmente, el precio de venta que se establece a cada momento. Cuando no se disponga del precio de venta en el momento, un precio de la transacción más reciente puede proporcionar la base adecuada para estimar el precio neto de venta, suponiendo que no se han suscitado cambios significativos en las circunstancias económicas, entre la fecha de la transacción y la fecha en que se realiza la estimación.\\

En caso de no existir un mercado activo, la entidad podría considerar transacciones recientes con activos similares efectuadas en el mismo sector industrial.\\

\subsubsection{Determinación del valor de uso}

El \textcolor{principal}{\textbf{valor de uso}} de los activos que conforman la unidad generadora de efectivo (UGE), es el valor presente de los flujos de efectivo futuros asociados con dicha unidad, aplicando una tasa apropiada de descuento.\\

Para determinar el valor de uso de los activos la entidad debe utilizar modelos técnicos de valuación reconocidos en el ámbito financiero y contar con sustento suficiente, confiable y comprobable para las estimaciones que utilice en la aplicación de los modelos mencionados.\\

El valor de uso puede determinarse a través de la técnica de valor presente estimado o la técnica de valor presente esperado.\\

El valor presente estimado implica la utilización de una sola estimación futura de flujos de efectivo, y los riesgos asociados con la unidad generadora de efectivo se consideran en la tasa de descuento a utilizar. Por otra parte, si se usa el valor presente esperado, los flujos de efectivo futuros a descontar se obtienen de ponderar diferentes escenarios que incorporan los riesgos asociados con la unidad en función de su probabilidad de ocurrencia, se utiliza una tasa de descuento libre de riesgo.\\

Indistintamente de la técnica de valor presente que se utilice, deberá evitarse la duplicación de los riesgos asociados con la unidad generadora de efectivo. Es decir, los riesgos que se consideren en la estimación de los flujos de efectivo no deberán considerarse en la tasa apropiada de descuento.\\

\subsubsection{Unidad Generadora de Efectivo (UGE)}

La \textcolor{principal}{\textbf{Unidad Generadora de Efectivo}} se refiere a la agrupación mínima identificable de activos que en su conjunto generan flujo de efectivo.\\

Algunos ejemplos son: activos intangibles como marcas, patentes, líneas de producción, divisiones y segmentos de negocio, subsidiarias, entre otros.\\
