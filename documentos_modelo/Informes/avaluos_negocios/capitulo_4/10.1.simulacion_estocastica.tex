\subsection{Metodo de Simulación estocástica}

La simulación estocástica es una herramienta matemática ampliamente utilizada en la valuación de activos financieros. Su objetivo principal es modelar y analizar fenómenos que presentan incertidumbre y variabilidad. En el contexto de la valuación de activos, como acciones, bonos o derivados financieros, la simulación estocástica permite estimar su valor y evaluar los riesgos asociados.\\

El proceso metodológico de la simulación estocástica se puede resumir en los siguientes pasos:

\begin{enumerate}
 \item Definición de variables y distribuciones de probabilidad: Se identifican las variables clave que afectan el valor del activo a ser valuado, como los precios de mercado, volatilidades, tasas de interés u otros factores relevantes. Para cada variable, se selecciona una distribución de probabilidad que describa su comportamiento.

\item Generación de trayectorias aleatorias: Utilizando las distribuciones de probabilidad seleccionadas, se generan múltiples trayectorias aleatorias de los valores de las variables clave a lo largo del tiempo. Esto se realiza mediante técnicas como el método de Monte Carlo, que permite simular diferentes escenarios posibles.

\item Modelado del activo: Se utiliza un modelo matemático que describe el comportamiento del activo financiero. Esto puede incluir modelos como el de Black-Scholes para opciones, el de Vasicek para tasas de interés o cualquier otro modelo adecuado. Las trayectorias generadas en el paso anterior se utilizan para introducir la incertidumbre en el modelo del activo.

\item Evaluación de indicadores de valuación y riesgo: Se calculan los indicadores de valuación deseados, como el precio del activo, el valor presente neto (VPN), el valor en riesgo (VaR) u otros. Estos indicadores proporcionan una medida cuantitativa del valor y riesgo asociados al activo valuado.

\item Análisis de resultados y conclusiones: Los resultados de la simulación estocástica se analizan y se obtienen conclusiones relevantes para la toma de decisiones. Esto puede incluir análisis estadísticos, sensibilidad de los resultados a cambios en las variables clave, evaluación de escenarios extremos, entre otros.

\end{enumerate}

La simulación estocástica es una metodología matemática poderosa para la valuación de activos financieros. Permite tener en cuenta la incertidumbre inherente a estos activos y proporciona una visión más completa de su valor y riesgo. Al incluir este enfoque en un reporte de valuación, se brinda una perspectiva más realista y fundamentada en el análisis de la incertidumbre y variabilidad de los activos financieros.\\

