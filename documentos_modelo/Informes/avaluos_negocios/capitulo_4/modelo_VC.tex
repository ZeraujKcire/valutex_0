\textcolor{principal}{ENFOQUE DE CAPITAL EMPRENDEDOR. Modelo Venture Capital (VC Approach)}. Se aplic\'o la metodolog\'ia conocida como \textit{venture capital approach (VC)} simulando la forma en que un fondo de capital posiblemente invertir\'ia en la empresa en su fase temprana. \\

El modelo es un tipo de financiaci\'on destinada a empresas en fases tempranas conocidas como  \textit{start-ups}, pero que bien ya demuestran o esperan obtener elevadas tasas de crecimiento en el corto plazo.  Los fondos de  \textit{venture capital}  invierten en estas empresas a cambio de una participaci\'on en el \textit{equity} del negocio. Generalmente se destina a empresas con un componente tecnol\'ogico o un modelo de negocio disruptivo, en sectores en los que se espera un crecimiento por encima de la media de la econom\'ia.\\

Este modelo financiero tiene 4 etapas:
\begin{enumerate}[\indent i)}
 \item Proyecci\'on de los ingresos y utilidades del activo valuado  durante el periodo expl\'icito de proyecci\'on. 
 \item  Capitalizaci\'on del activo al final del periodo proyecci\'on utilizando un m\'ultiplo de salida adecuado (\textit{exit multiple}), 
 \item  Valor presente de la estimaci\'ion a valor de del activo sujeto de an\'alisis, eligiendo la tasa de descuento adecuada que impacte el riesgo del capital emprendedor (valuaci�n \textit{pre-money}),
  \item   Asignaci\'on a los fundadores del porcentaje de la firma (\textit{pre-money}) y la valuaci\'on del capital accionario de la empresa (\textit{post-money}) para todos los inversionistas.
