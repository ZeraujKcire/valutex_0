MPEEM (Multy-period Excess Earnings Method)

Enfoque del ingreso

Esta dise\~nado para aplicarse a solamente un activo (o grupo de activos similares) el cual es el mas relevante en la generaci\'on de flujo de efectivo.\\

MPEEM se basa en el hecho de que un activo puede general flujos de efectivo solo en conjunto con otros activos, llamados ``Contributory assets''.\\

Pasos para llevar a cabo el m\'etodo:
\begin{enumerate}
	\item Determinar Ingresos atribuibles al activo.
	\item Vida util remanente
	\item Ingreso operativo
	\item Cargos de ``Contributory assets'' (CACs)
	\item Tasa de descuento
	\item Beneficio por amortizaci\'on fiscal
\end{enumerate}

\textbf{Ingresos atribuibles al activo}\\

Los ingresos generados por el activo principal ( el que se quiere valuar) debe poderse aislar de manera confiable.\\

Para facilitar el proceso, es conveniente agrupar los``Contributory assets'' en  categor\'ias con caracter�sticas similares\\

En algunas casos, poco frecuentes, puede ser imposible identificar un solo activo principal. Si dos activos intangibles son considerados igualmente relevantes en terminos de generaci\'on de ingresos, es posible aplicar el MPEEM para los dos activos de manera conjunta. Utilizar dos MPEEM de manera simultanea puede llevar a distintos resultados, a menos que los ingresos de ambos activos sean mutuamente excluyentes.

Si no es posible separar por completo los ingresos generados por los dos activos, se puede calcular el valor de uno de ellos por alg\'un otro m\'etodo aplicable y despues utilizarlo como ``Contributory asset'' para el calculo del valor del otro activo.\\

\textbf{Vida util remanente}
La vida util remanente es muy importante para valuar activos intangibles ya determina el periodo de tiempo en el cual se espera recibir beneficios econ\'omicos del activo. El c\'alculo de la vida util remanente del activo depender\'a en gran medida de la naturaleza del activo intangible.\\
En algunos casos como marcas, se puede considerar una vida util infinita, ya sea porque no existe restricci\'on de tiempo para el uso del intangible, o porque no es posible determinar el periodo durante el cual se espera recibir beneficios econ\'omicos. Utilizar una vida util remanente infinita  no se considera apropiado para algunos activos intangibles como los de tecnolog\'ia.

\textbf{ingreso operativo}
Despues de identificar los ingresos relevantes para el activo intangible, se deben hacer ajustes a los flujos de caja futuros para remover los efectos de actividades no relacionadas al activo intangible que se est\'a valuando.

\textbf{Cargos de activo contributivos}
Determinar los ingresos generados de manera exclusiva por un activo intangible rara vez es posible ya que otros activos contribuyen a la generaci\'on de estos ingresos. Los cargos de activos contributivos (CAC's) nos permiten aislar el flujo de caja relativo al activo intangibleEl CAC representan el retorno de inversi\'on que el propietario del activo requerir\'ia.\\
Para obtener el CAC, primero se determina el valor razonable de cada activo, as\'i como una tasa de descuento espec\'ifica y vida util remanente para cada Activo contributivo, para posteriormente convertir el valor razonable del activo en una anualidad. Esta anualidad es el CAC.\\
El nivel del CAC dependen de la depreciaci\'on del activo y de su rendimiento. La mejor forma para determinar el rendimiento de un activo es a traves del WACC.\\

\textbf{Tasa de descuento}




