\subsection{Método de Precio Comparable No Controlado (PC)}

El método \textcolor{principal}{PC} compara las tasas de interés pactadas en una operación controlada, con tasas de intereses pactadas en circunstancias y condiciones similares en una operación no controlada\\

Una operación controlada es aquella que se realiza entre partes vinculadas. Por su parte una operación no controlada es aquella que se efectúa entre partes independientes. Dichas operaciones son comparables entre sí, si se cumplen alguna de las siguientes condiciones:\\

\begin{itemize}
\item No existen diferencias que afecten los precios a ser comparados; o
\item Pueden efectuarse ajustes razonables para eliminar los efectos materiales de dichas
diferencias.
\end{itemize}


Cuando se pueden encontrar operaciones comparables no controladas, el método PC es el método mas directo y confiable. Este método podría ser aplicado en los siguientes casos:\\

\begin{itemize}
\item Cuando existen operaciones comparables internas, esto significa operaciones comparables realizadas entre \empresaCorto{} y terceros independientes;
\item Cuando existen operaciones comparables externas, esto es operaciones realizadas entre partes independientes que son muy similares o que cumplen con los requisitos de comparabilidad; y
\item Cuando existe información pública sobre precios de referencia en el mercado abierto.
\end{itemize}

Los factores que deben ser tomados en cuenta para propósitos de determinar la comparabilidad de productos y la aplicabilidad del Método PC incluyen, entre otros, los siguientes:\\

\begin{itemize}
\item Naturaleza y magnitud de los servicios;
\item Fecha de la operación;
\item Estrategia asociada con la misma
\item Alternativas reales disponibles para las partes involucradas

\end{itemize}

