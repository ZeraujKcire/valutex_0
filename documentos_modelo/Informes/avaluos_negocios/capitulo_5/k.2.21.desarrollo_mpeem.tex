\subsubsection{Desarrollo del método MPEEM.}

Se llevo a cabo  el método ``MPEEM''  para estimar el Valor Razonable del activo intangible de la sociedad \textcolor{principal} bao el enfoque de ingresos. Como paso preliminar, se identificaron los activos intangibles sujetos a ser valuados por el método ``MPEEM'' de acuerdo a la información proporcionada por el solicitate:

\begin{figure}[H]
\centering
	\includegraphics[widtj=.6\textwidth]{../0.imagenes/mpeem_activos}
\end{figure}

\subsubsection*{Estimación de los ingresos futuros provenientes de los activos intangibles específicos.}

Para determinar los ingresos futuros provenientes de los activos seleccionados, se procedió por el método residual. El primer paso consistió en identificar el tipo y valor da los \textcolor{principal}{Activos Contributivos}, necesarios para sostener la generación de ingresos de los activos intangibles. 

\begin{figure}[H]
\centering
	\includegraphics[widtj=.6\textwidth]{../0.imagenes/mpeem_contrib_1}
\end{figure}

A continuación, se procedió a estimar la tasa de descuento para cada \textcolor{principal}{Activo Contributivo} basado en el riesgo estimado asociado al activo. Se tomó para este fin la tasa libre de riesgo para México, tomando el pronóstico de rendimiento del bono mexicano a 10 años de \textbf{9.08 \%}, el costo de la deuda después de impuestos \textbf{\kdValor \%} y el costo de capital  para México de \textbf{20.07 \%}. Finalmente se ponderó la tasa de los \textcolor{principal}{Activos Contributivos} de acuerdo a la composición al cierre del ultimo ejercicio, y así obtener la tasa de rendimiento promedio ponderada (\textcolor{principal}{WARA}).\\

\begin{figure}[H]
\centering
	\includegraphics[widtj=.6\textwidth]{../0.imagenes/wara}
\end{figure}

Una vez hecho lo anterior, se procedió a calcular el excedente de ingresos asociado a los activos intangibles identificados, restando los cargos de los \textcolor{principal}{Activos Contributivos} al flujo de efectivo libre a la firma del ultimo periodo.\\

\begin{figure}[H]
\centering
	\includegraphics[widtj=.6\textwidth]{../0.imagenes/mpeem_int_fcf}
\end{figure}

El siguiente paso consiste en proyectar el flujo de caja libre y los cargos de los Activos Contributivos para así obtener el Exceso correspondiente a los activos intangibles:\\

\begin{figure}[H]
\centering
	\includegraphics[widtj=.6\textwidth]{../0.imagenes/mpeem_int_fcf_2}
\end{figure}

El siguiente paso consiste en Calcular y sumar el valor presente de los beneficios económicos proyectados de los activos intangibles identificados y se resta el valor razonable de la marca \marca obtenido en la sección anterior, obteniendo así el valor de los activos intangibles identificados en su conjunto; según se muestra a continuación:

\begin{figure}[H]
\centering
	\includegraphics[widtj=.6\textwidth]{../0.imagenes/valor_mpeem}
\end{figure}

El valuador le asign\'o a este m\'etodo de valuaci\'on una ponderaci\'on del 50.00\% total en el valor conclusivo de los activos intangibles identificados.//% se\~nalado en el \autoref{sec:f}).

\subsubsection{Valor Razonable de los Activos Identificados.}

Para determinar el Valor razonable de los activos identificados por el método MPEEM, se tomo como base el valor de Mercado de los activos determinado por el Avalúo antecedente y que fue proporcionado al valuador por parte del solicitante; y se ponderó el valor obtenido por el método MPEEM de acuerdo a los montos proporcionados:

\begin{figure}[H]
\centering
	\includegraphics[widtj=.6\textwidth]{../0.imagenes/valor_activos_mpeem}
\end{figure}



\subsection{Aplicaci\'on del Enfoque Residual para la Estimaci\'on del valor de los Activos Intangibles Identificados.}

El perito valuador aplic\'o la metodolog\'ia residual para determinar el valor razonable de los activos intangibles de la empresa, como paso previo a la estimaci\'on del valor de marca del siguiente cap\'itulo; seg\'un se muestra:\\

\begin{figure}[H]
\centering
\includegraphics[width=.6\textwidth]{../0.imagenes/valor_residual_intangibles_identificados}\\[10pt]


\textcolor{principal}{\textbf{Valor de los Activos Intangibles  por Enfoque Residual al \fechaValoresCorto: \$\valorResidual{} \monedaCode.}}\\
(\textcolor{principal}{\valorResidualLetra{} \moneda 00/100 M.N.})

\end{figure}


El valuador le asign\'o a este m\'etodo de valuaci\'on una ponderaci\'on del 50.00\% total en el valor conclusivo de los  activos intangibles identificados.


\subsection{VALOR RAZONABLE PONDERADO DE LOS ACTIVOS INTANGIBLES IDENTIFICADOS.}

Despu\'es de haberse realizado el an\'alisis de valor de los activos intangibles identificados, conforme a la aplicaci\'on de los modelos financieros descritos en este presente cap\'itulo, el valuador llev\'o a cabo como parte de su an\'alisis conclusivo, la ponderaci\'on de los dos modelos: i) MPEEM (50\%), ii) Método Residual  (50\%):

\begin{figure}[H]
\centering
\includegraphics[width=.7\textwidth]{../0.imagenes/valor_pond_activos_intangibles}\\[10pt]

\textbf{\textcolor{principal}{Valor de los  Activos Intangibles identificados al \fechaValoresCorto:} \$\valorActivoIntangible{} \monedaCode}\\[5pt]
(\textcolor{principal}{\valorActivoIntangibleLetra{} \moneda{} 00/100 M.N.})


\end{figure}





