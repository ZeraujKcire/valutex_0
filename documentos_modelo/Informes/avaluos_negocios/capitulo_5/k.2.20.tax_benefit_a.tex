\subsubsection{Estimación del Valor Presente del Beneficio Fiscal por Amortización.}

 La teoría de valuaciones asume que los activos intangibles podrían ser vendidos individualmente o agrupados con otros activos en una transacción. Para estos casos la Ley de Impuesto sobre la Renta de México permite que ciertos intangibles sean amortizados disminuyendo la base gravable y por lo tanto el pago de impuestos, creando un escudo fiscal. La estimación de un beneficio por amortización fiscal únicamente se realiza en aquellos casos en los que el valor razonable del activo intangible se haya estimado a través del Enfoque de Ingresos. Esto se debe a que el Enfoque de Ingresos considera una proyección de pago de impuestos, mientras que en el Enfoque de Mercado el valor es estimado a partir de precios de mercado pagados por activos comparables; y por lo tanto todos los beneficios deberían estar ya capturados en el precio pagado.\\

Se muestran a continuación los resultados de dicha estimación\footnote{Se llevó a cabo la estimación del valor presente la amortización fiscal conforme al tratamiento de los ``gastos diferidos'', amortizándose únicamente el 5\% anual del valor del intangible.}:


\begin{figure}[H]
\centering
\includegraphics[width=13cm]{../0.imagenes/bfa}
\end{figure}

\subsubsection{Cifras conclusivas del valor del Activo Intangible}

\begin{figure}[H]
\centering
	\includegraphics[width=8cm]{../0.imagenes/valor_raz_cap_int}


\textcolor{principal}{\textbf{VALOR DEL ACTIVO INTANGIBLE al \fechaValoresCorto:}} \textbf{\$\valorCapitalIntangible{} \monedaCode} 

\end{figure}
