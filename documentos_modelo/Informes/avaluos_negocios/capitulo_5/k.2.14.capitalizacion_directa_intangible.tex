\subsection{APLICACIÓN DEL MÉTODO DE CAPITALIZACIÓN DIRECTA. Mediante el enfoque de ingresos.} 

\subsubsection*{ANÁLISIS FINANCIERO PREVIO A LA ESTIMACIÓN DEL FLUJO DE OPERACIÓN NETO ANUAL DEL NEGOCIO.} 

Previo a la estimación del flujo de efectivo del negocio, el valuador analizó el resultado histórico de los ingresos del negocio, así como el comportamiento de la reinversión al capital de trabajo y al activo no corriente del negocio; con el objetivo de establecer bases de proyección para la estimación del activo intangible, según se aprecia:

\begin{figure}[H]
\centering
\includegraphics[width=10cm]{../0.imagenes/capitalizacion_directa_1}
\end{figure}

\begin{figure}[H]
\centering
\includegraphics[width=\textwidth]{../0.imagenes/capitalizacion_directa_2}
\end{figure}

En vista del alto grado de variabilidad de los datos históricos de la empresa, el valuador llevó a cabo la normalización de los tres indicadores anteriores removiendo posibles partidas extraordinarias, no recurrentes y/o no operativas del negocio.\\

Del análisis financiero anterior, el valuador concluye que no es posible llevar a cabo una proyección del negocio en el corto plazo, con base en los indicadores obtenidos anteriormente; razón por la cual se optó por el método de capitalización directa o estático, basándose en la obtención del flujo de operación neto del negocio más reciente desde 2 perspectivas diferentes: \textit{i) Flujo de operación neto normalizado a 2021 (Normalized Nopat). ii) Flujo de operación neto del año más reciente 2022 (NOPAT 2022).}

\begin{enumerate}[1)]
\item ESTIMACIÓN DEL FLUJO DE OPERACIÓN NETO DEL NEGOCIO 2022 (Nopat 2022). El valuador llevó a cabo la estimación del flujo de operación neto después de impuestos del negocio con cifras a 2022, basándose en 3 indicadores: 

\begin{enumerate}[I.]

\item Pronóstico de Ingresos anuales 2022E de \$6’365,243.55 MXN.
\item Margen operativo del 11.17\%.
\item Tasa fiscal efectiva de  21.85\%, como resultado de la media del periodo 2018 a 2021, sin tomar en cuenta el año 2019 que resultó atípico en el comportamiento de la tasa efectiva del negocio.

\end{enumerate}

\begin{figure}[H]
\centering
\includegraphics[width=8cm]{../0.imagenes/capitalizacion_directa_3}
\end{figure}

Por lo anterior, se estimó un flujo de operación neto (anual) de \$558,081.00 MXN, en cifras redondas.\\

\textbf{Indicador por enfoque de capitalización directa del NOPAT 2022:}\\

\begin{itemize}

\item Estimación de la tasa de largo plazo mediante la aplicación de la ecuación Fisher; basado en el pronóstico de crecimiento de Banxico:

\begin{figure}[H]
\centering
\includegraphics[width=8cm]{../0.imagenes/capitalizacion_directa_4}
\end{figure}

\item Aplicación de una perpetuidad creciente para la estimación del valor del activo intangible por el método Greenfield:

\begin{figure}[H]
\centering
\includegraphics[width=8cm]{../0.imagenes/capitalizacion_directa_5}\\
\textbullet\textbf{Se obtuvo un valor razonable de la firma de \$6'601,556.00 MXN, en cifras redondas.}
\end{figure}
\end{itemize}

\item \textbf{ESTIMACIÓN DEL FLUJO DE OPERACIÓN NETO DEL NEGOCIO 2021 (Nopat 2021).} El valuador llevó a cabo la estimación del flujo de operación neto después de impuestos del negocio, basándose en 4 indicadores:

\begin{enumerate}[I.]

\item Margen Nopat: 5.60\%
\item Rotación del Capital Invertido (IC Turnover): 1.24x
\item Retorno del Capital Invertido 6.92\%

\end{enumerate}

\begin{figure}[H]
\centering
\includegraphics[width=8cm]{../0.imagenes/capitalizacion_directa_6}
\end{figure}


Por lo anterior, se estimó un flujo de operación neto anual de \$731,966.00 MXN, en cifras redondas.\\

\textbf{Indicador por enfoque de capitalización directa del Nopat 2021:}\\

\begin{itemize}

\item Estimación de la tasa de crecimiento de largo plazo mediante la ecuación Fisher:

\begin{figure}[H]
\centering
\includegraphics[width=8cm]{../0.imagenes/capitalizacion_directa_7}
\end{figure}

\item Aplicación de una perpetuidad creciente para la estimación del valor del activo intangible por el método Greenfield:

\begin{figure}[H]
\centering
\includegraphics[width=10cm]{../0.imagenes/capitalizacion_directa_8}\\
\textbullet\textbf{Se obtuvo un valor razonable del negocio de \$8’744,635.00 MXN, en cifras redondas.}
\end{figure}

\end{itemize}
\end{enumerate}



