\subsubsection{An\'alisis Financiero}

Se recibieron los Estados financieros hist\'oricos de \EFde{} y a \EFhasta, por parte del  solicitante,  a los cuales el valuador les realiz\'o algunos ajustes para poder analizar el capital intangible, eliminando en el balance las partidas no ordinarias, no recurrentes o no operativas; seg\'un se muestran a continuaci\'on:

\begin{figure}[H]
\centering
\caption{Estados de Situaci\'on Financiera Hist\'oricos. Activos \EFdeHasta \label{fig:ESF}}
\includegraphics[width=15cm]{imagenes/EF_activo}\\[5pt]

\end{figure}
\begin{figure}[H]
\centering
\caption{Estados de Situaci\'on Financiera Hist\'oricos. Pasivos y Capital \EFdeHasta \label{fig:ESF_2}}
\includegraphics[width=15cm]{imagenes/EFpasivo_capital}\\

\end{figure}

\begin{figure}[H]
\centering
\caption{Estados de Resultados Hist\'oricos \EFdeHasta \label{fig:ESF_3}}
\includegraphics[width=16cm]{imagenes/ER}\\
\end{figure}

Para el an\'alisis financiero de dicha empresa, se llevaron a cabo los siguientes m\'etodos:
\begin{enumerate}
\item An\'alisis del Capital Invertido (IC\footnote{Investment Capital}).
\item Razones financieras de liquidez y solvencia (Ratios bancarios).
\item An\'alisis Dupont de 3 elementos.
\item An\'alisis de la Rentabilidad del Capital (ROC\footnote{Return on Capital}).
\item An\'alisis de la Rentabilidad del Capital Operativo Neto (ROIC\footnote{Return on Investment Capital}).
\end{enumerate}
\newpage
\begin{center}
\underline{An\'alisis del Capital Invertido (IC)}\\[10pt]
\includegraphics[width=14cm]{imagenes/IC}\\[10pt]

\underline{Razones financieras de liquidez y solvencia}\\[10pt]
\includegraphics[width=14cm]{imagenes/ratios}\\[10pt]

\underline{An\'alisis Dupont}\\[10pt]
\includegraphics[width=14cm]{imagenes/dupont}\\[10pt]

\underline{An\'alisis ROC}\\[10pt]
\includegraphics[width=14cm]{imagenes/roc}\\[10pt]



\underline{An\'alisis ROIC}\\[10pt]
\includegraphics[width=14cm]{imagenes/roic}\\[10pt]
\end{center}