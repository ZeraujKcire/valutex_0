
\subsection{Aplicaci\'on del M\'etodo de valuaci\'on por VC Approach}

\subsubsection{Proyecci\'on Financiera de Ingresos}

Para el pron\'ostico de 2021, a partir de las cifras de Octubre de 2021, se estim\'o un ingreso esperado aproximado de 55.4 mdp (2021E). \\


A su vez, el valuador llev\'o a cabo una proyecci\'on de ingresos de corto plazo para el negocio a un horizonte de 5 a\~nos, a partir de la tasa geom\'etrica hist\'orica de crecimiento en ventas de 8.77\% (CAGR 2017 – 2020); para aplicar dicho pron\'ostico a partir de 2021; con un resultado esperado de 6.95\% A/A geom\'etrico en el periodo de 2020 a 2025; seg\'un se aprecia:\\

\begin{figure}[H]
\centering
\includegraphics[width=\textwidth]{../0.imagenes/vc_1}
\end{figure}


\textcolor{principal}{Estimaci\'on de la tasa de descuento aplicable al VC Approach.} El valuador eligi\'o el modelo de Beta total para la estimaci\'on de una tasa de descuento para capital emprendedor, con el objetivo de tener una tasa que refleje el riesgo total de este tipo de negocios en el sector tecnol\'ogico con crecimiento de ingresos pero p\'ordidas operativas que requieren altas tasas de reinversi\'on y est\'a expuesto a diversos riesgos que impactan el costo de oportunidad de sus accionistas.\\

\begin{figure}[H]
\centering
\includegraphics[width=12cm]{../0.imagenes/vc_2}
\end{figure}

\textcolor{principal}{Estimaci\'on del m\'ultiplo de mercado bajo el modelo PEERS.} Se tom\'o la cifra de proyecci\'on de ingresos a 5 a\~nos bajo el modelo de m\'ultiplos de cotizaci\'on, en el cual se utiliz\'o como pron\'ostico para dicho enfoque el Promedio de la muestra, con un indicador de 4.20x (Ev to Total Revenues x), con base en un an\'alisis de comparabilidad basado en activos, funciones y riesgos:\\

\begin{figure}[H]
\centering
\includegraphics[width=16cm]{../0.imagenes/vc_peers}
\end{figure}

\textcolor{principal}{Obtenci\'on del indicador de valor por el modelo VC Approach:}  Al pron\'ostico de ingresos se le aplic\'o una tasa de capital de riesgo del 24.21\% (\textit{Distressed rate}) y estim\'andose una cobertura emp\'irica de riesgo de 2.0 a 1.0; la cual impacta el alto grado de incertidumbre de un proyecto de esta naturaleza en materializaci\'on de sus pron\'osticos; cuyos resultados se muestran:\\

\begin{figure}[H]
\centering
\includegraphics[width=10cm]{../0.imagenes/vc_3}
\end{figure}





