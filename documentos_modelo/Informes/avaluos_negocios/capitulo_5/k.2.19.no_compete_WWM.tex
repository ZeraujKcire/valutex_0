\subsection{VALUACIÓN DE CONTRATO DE NO COMPETENCIA. Método WWM (With-and-Without Method).}

 Para la aplicación del presente método de valuación respecto del bien mencionado en el \autoref{sec:f}, el perito valuador llevó a cabo el modelaje de dos escenarios financieros para el negocio, uno CON el activo intangible y otro SIN el activo intangible. Ambos escenarios parten de una proyección de flujos de efectivo del negocio, con cifras esperadas a 2022E y a un horizonte de proyección de 5 años (2022E-2026F); lo cual es consistente con la vigencia de dicho contrato de no competencia que se agrega al presente como APÉNDICE 3.


\subsubsection{Escenario financiero SIN el contrato de no competencia.}

  Como premisas de dicho escenario, el valuador consideró con base en lo declarado por el solicitante en la entrevista con la dirección de finanzas (\autoref{sec:l}), una estimación respecto de la \textbf{probabilidad de pérdida de participación de mercado SIN  la protección de una Cláusula de no competencia} (\autoref{sec:f}) que pudiera ascender hasta el 70\% en los escenarios modelados por la empresa; derivada del análisis de los resultados de un estudio de mercado interno\footnote{El valuador no tuvo a la vista dicho estudio de mercado realizado por el solicitante, por manifestarse como información confidencial y privilegiada por parte del solicitante.}  realizado con anterioridad a la firma del Contrato (APÉNDICE 3) y del cual se derivó la firma de dicho contrato según se menciona en el \textcolor{secundario}{inciso F.\ref{f2}})\\
 
Al respecto, se llevó a cabo el modelaje financiero, según los detalles que se aprecian a continuación: 



\begin{figure}[H]
\centering
\includegraphics[width=.8\textwidth]{../0.imagenes/wwm_1}
\end{figure}


\subsubsection{Escenario financiero CON el contrato de no competencia.}
  Como premisas de dicho escenario, el valuador consideró los supuestos con los cuales se llevó a cabo la valuación por el modelo de Flujos de Efectivo Descontados (DCF Valuation) mencionada en el \autoref{k.1.2}, según se muestra a continuación:

\begin{figure}[H]
\centering
\includegraphics[width=.8\textwidth]{../0.imagenes/wwm_2}
\end{figure}


\subsubsection{Obtención del indicador de valor mediante el Método WWM (With-and-Without Method).}

 Una vez realizado el modelaje financiero CON y SIN la presencia del contrato de no competencia en las finanzas de la sociedad, se muestran a continuación los resultados:

\begin{figure}[H]
\centering
\includegraphics[width=.8\textwidth]{../0.imagenes/wwm_3}
\end{figure}


\subsubsection{Estimación del Valor Presente del Beneficio Fiscal por Amortización.}

 La teoría de valuaciones asume que los activos intangibles podrían ser vendidos individualmente o agrupados con otros activos en una transacción. Para estos casos la Ley de Impuesto sobre la Renta de México permite que ciertos intangibles sean amortizados disminuyendo la base gravable y por lo tanto el pago de impuestos, creando un escudo fiscal. La estimación de un beneficio por amortización fiscal únicamente se realiza en aquellos casos en los que el valor razonable del activo intangible se haya estimado a través del Enfoque de Ingresos. Ésto se debe a que el Enfoque de Ingresos considera una proyección de pago de impuestos, mientras que en el Enfoque de Mercado el valor es estimado a partir de precios de mercado pagados por activos comparables; y por lo tanto todos los beneficios deberían estar ya capturados en el precio pagado.\\

Se muestran a continuación los resultados de dicha estimación\footnote{Se llevó a cabo la estimación del valor presente la amortización fiscal conforme al tratamiento de los ``gastos diferidos'', amortizándose únicamente el 55\% anual del valor del intangible.}:

\begin{figure}[H]
\centering
\includegraphics[width=16cm]{../0.imagenes/wwm_4}
\end{figure}

Una vez estimado dicho beneficio fiscal por amortización a valor presente, se muestra el valor preliminar del Contrato al \fechaValores:

\begin{figure}[H]
\centering
\includegraphics[width=12cm]{../0.imagenes/wwm_5}
\end{figure}

\subsubsection{Cifras conclusivas del valor razonable de la cláusula de no competencia.}

 El solicitante le manifestó al valuador que en línea con la investigación de mercado interna mencionada en el Inciso K.4.1., se ha determinado que existe una probabilidad aproximada del 66.67\%\footnote{El solicitante declara que el actor con quien se firmó el ``non compete'' controla al menos el 2/3 partes del mercado local, por lo que otros agentes económicos representan un 1/3 del mercado local (Morelia y poblaciones conurbanas).}  para un escenario de competencia que se podría materializar en caso de no haberse pactado un non compete. Sin embargo, el valuador aplicó una cobertura empírica de riesgo a dicha pronóstico del 20\%\footnote{A manera de tasa de error (Error rate \%).} , obteniéndose una tasa a números redondos del 39\%\footnote{(0.3879) Mediante la ecuación Fisher: $[(1 + 0.6667) / (1 + 0.2008)] - 1 = 0.3879$}; según se muestra:

\begin{figure}[H]
\centering
\includegraphics[width=12cm]{../0.imagenes/wwm_6}
\end{figure}


\subsection{VALOR RAZONABLE DE LOS ACTIVOS INTANGIBLES VALUADOS.}

Después de haberse realizado el análisis de valor del negocio en marcha con sus activos intangibles, conforme a la aplicación de los modelos financieros descritos en este presente capítulo, el valuador concluye el presente informe con los siguientes resultados en cifras redondeadas y en pesos mexicanos al \fechaValores:

\subsubsection{Valor razonable de la Marca Congeladora Morelia (Inciso F.\ref{f1}):}

\begin{figure}[H]
\centering
\includegraphics[width=16cm]{../0.imagenes/wwm_7}
\end{figure}

\subsubsection{Valor Razonable de la Cláusula de No Competencia (Inciso F.\ref{f2}):}

\begin{figure}[H]
\centering
\includegraphics[width=16cm]{../0.imagenes/wwm_8}
\end{figure}

