\subsection{DESARROLLO Método de Precio Comparable No Controlado (PC)}

Se utilizó el Método PC debido a que fue posible comparar las tasas de interés pactadas por TDN y sus partes relacionadas, puesto que existe información pública disponible referente a tasas de intereses similares a las pactadas por TDN entre terceros independientes en circunstancias comparables.\\

Por lo anterior, no fue necesario descartar otras metodologías, lo anterior toda vez que al aplicar el Método PC en primera instancia, ya no es necesario analizar la aplicación de cualquier otro método, toda vez que la LISR da prioridad a la utilización del método en comento.

\subsubsection{Búsqueda de operaciones comparables}

Como primer paso, se identificó  la Tasa de interés interbancaria de equilibrio (TIIE): que es una tasa representativa de las operaciones de crédito entre bancos. La TIIE es calculada diariamente (para plazos 28, 91 y 182 días) por el Banco de México con base en cotizaciones presentadas por las instituciones bancarias mediante un mecanismo diseñado para reflejar las condiciones del mercado de dinero en moneda nacional. La TIIE se utiliza como referencia para diversos instrumentos y productos financieros.

\begin{figure}[H]
\centering
\includegraphics[width=.2\textwidth]{../0.imagenes/tiie}
\end{figure}

Posteriormente se consultó en la CNBV información sobre tasas de interés referentes a los intereses promedio otorgados por la banca comercial a empresas y a la cartera de crédito otorgado por entidades financieras a empresas en el mercado mexicano dependiendo la industria donde se desempeñan.\\ 

Para poder determinar un spread, consideramos para nuestro análisis tasas pasivas que se encuentran en el mercado, como lo son los Certificados Bursátiles y el Costo de Captación a Plazo. Como se mencionó anteriormente, para nuestro análisis usamos las tasas vigentes por el periodo del financiamiento, a continuación se presentan los valores de dichos instrumentos financieros: \\

\begin{figure}[H]
\centering
\includegraphics[width=.8\textwidth]{../0.imagenes/tiie_2}
\end{figure}

Inicialmente se identificaron los promedios de las tasas pactadas por las instituciones en el mercado con las cuales se pudo obtener un rango Intercuartil para poder analizar la tasa de interés pactada por TDN y sus partes relacionadas.\\

Una vez obtenidos dichos indicadores, se obtuvo el Spread correspondiente como resultado de disminuir las tasas pasivas antes mencionadas de las tasas activas. \\

Adicionalmente, a estos resultados se agregó la Tasa de Interés Interbancaria de Equilibrio (TIIE), la cual es determinada por el Banco de México con base en cotizaciones presentadas por las instituciones de crédito, teniendo como fecha de inicio la publicación en el Diario Oficial de la Federación. \\

A continuación, se presenta la información obtenida del procedimiento anterior:\\

\subsubsection{Intereses devengados a favor}

\begin{figure}[H]
\centering
\includegraphics[width=\textwidth]{../0.imagenes/tiie_3}
\end{figure}

\begin{figure}[H]
\centering
\includegraphics[width=.2\textwidth]{../0.imagenes/tiie_4}
\end{figure}

\subsubsection{Conclusiones}

Durante 2022 TDN recibió ingresos por concepto de intereses devengados a favor con sus partes relacionadas residentes en México, derivado de diversos financiamientos denominados en pesos. El método seleccionado para evaluar si dicha operación se encuentra a valores de mercado fue el Método PC.\\

Como se puede observar, considerando el rango de intereses obtenidos, se puede concluir que las tasas de interés pactadas por TDN con sus partes relacionadas, se encuentran dentro de los rangos Intercuartiles de mercado, por lo que TDN en la operación de Intereses devengados a favor, cumple con el principio de plena competencia, así como con lo establecido en los artículos 76 fracción XII, 179 y 180 de la LISR vigentes durante 2022.
