\subsection{DESARROLLO DEL AVALÚO MEDIANTE EL ENFOQUE DE ACTIVOS}

\subsubsection{Análisis de mercado del Valor de Reposición Neto Marcario.}

Se realizó una muestra con base en la información obtenida de diversos despachos de abogados que cuentan con el servicio de ``Registro de marca''. A su vez, se obtuvo información de otros costos indirectos atribuibles al activo intangible sujeto de análisis; respecto del cual se inserta a continuación la siguiente relación (cifras en pesos mexicanos MXN):

\begin{figure}
\centering
\includegraphics[width=.8\textwidth]{../0.imagenes/vrn_1}
\end{centering}

\begin{figure}
\centering
\includegraphics[width=.8\textwidth]{../0.imagenes/vrn_2}
\end{centering}

El valuador llevó a cabo un análisis de estadística descriptiva, utilizando una muestra que se presenta en el punto K1.1. Al respecto se hizo el análisis cuantitativo de la información de mercado y se decidió utilizar el modelo conocido como ``Rango Intercuartil'' sugerido por la OCDE como un método de dispersión adecuado para evaluar operaciones bajo el principio de plena competencia (Arm’s length), el cual permite ubicar un rango de mercado entre el percentil 25 y 75; aplicando la mediana como medida centrar para determinar el valor de reposición nuevo del Registro de una Marca, con un valor de \$90,017.00 MXN.

\begin{figure}
\centering
\includegraphics[width=.8\textwidth]{../0.imagenes/vrn_3}
\end{centering}

 De la muestra analizada conforme al rango intercuartil y de las gráficas observadas en la figura 3, se aprecia que los valores obtenidos del mercado son homogéneos y no presentan diferencia significativa entre ellos, por lo que dicha muestra es apta y representativa para el estudio.
 
 \subsubsection{Estimación del valor de reposición del diseño gráfico e imagen corporativa de la marca como costo indirecto.}
 
 El valuador llevó a cabo una investigación independiente con diversos proveedores de diseño respecto del costo de reposición del diseño (\textit{Branding}) de la marca, habiendo obtenido un presupuesto especial para la marca sujeta de estudio por parte de una firma de diseñadores especializada en este tipo de servicios, la cual fue la única que cumplió con los criterios técnicos adecuados y el desglose del servicio por cada etapa del proyecto de diseño; razón por la cual se estimaron los costos indirectos que se aprecian continuación:
 
 
 \begin{figure}
\centering
\includegraphics[width=.8\textwidth]{../0.imagenes/vrn_4}
\end{centering}
 
 Para la determinación del indicador de valor por enfoque de Costos de la Marca \textcolor{principal}{\marca} el valuador consideró el valor de Reposición Neto del Diseño Gráfico con las partidas necesarias para poder realizar el servicio y que se muestran en la figura número 4, dando un resultado de Honorarios por diseño \$10,218.00 MXN.
 
 \subsubsection{Estimación del Enfoque de Costos:}
 
  \begin{figure}
\centering
\includegraphics[width=.8\textwidth]{../0.imagenes/vrn_5}
\end{centering}
 