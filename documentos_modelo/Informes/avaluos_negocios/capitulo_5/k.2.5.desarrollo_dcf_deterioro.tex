\subsection{Estimación del VALOR DE USO de la UGE.  Mediante el Modelo de Flujos de Efectivo Descontados.}

\textcolor{principal}{\textbf{Flujos de efectivo descontados (Discounted Cash FLow). DCF Valuation Method.}} Para la valuación de una empresa es indispensable el conocimiento de los siguientes cuatro conceptos: i) Flujos de efectivo (Cash Flows \$). ii) Tasa de crecimiento de ingresos (Growth Rate \%). iii) Tasa de descuento. (Disc. Rate \%). iv) Valor terminal. (Terminal Value \$).\\

El modelo de Flujos de Efectivo descontados, es un método dinámico de valuación que toma en consideración el valor del dinero en el tiempo y permite evaluar el efecto concreto de las variables en los rendimientos y comportamientos futuros de la empresa. Se basa en medir la capacidad de generar riqueza futura, por lo que se proyecta el Flujo de Efectivo el cual se actualiza mediante su tasa de descuento. \\

Su formulación se aprecia a continuación:

\begin{figure}[H]
\centering
\includegraphics[width=10cm]{../0.imagenes/dcf_1}
\end{figure}

\subsubsection{Estimación del Flujo de Efectivo Histórico del Negocio (Periodo 2013-2021).}

El perito valuador aplicó la metodología de análisis de flujo de efectivo a la firma (Free Cash Flow to Firm o FCFF) conocida como ``EBIT pormenorizado sin depreciación''; misma que se le atribuye al profesor Aswath Damodaran:

\begin{figure}[H]
\centering
\includegraphics[width=\textwidth]{../0.imagenes/dcf_2}
\end{figure}

\newpage

\subsubsection{Proyección del Flujo de Efectivo del Negocio (Forecast 2022-2027).}

La proyección del Flujo de efectivo del negocio se llevó a cabo conforme a las siguientes premisas (Ceteris paribus): i) Tasa de Crecimiento anual: 5.48\% (CAGR 2021-2027F), ii) Margen bruto: 17.3\%, Margen operativo, 12.39\%. Tasa Fiscal: 30\% (MTR). Los resultados de la proyección se aprecian a continuación:

\begin{figure}[H]
\centering
\includegraphics[width=\textwidth]{../0.imagenes/dcf_3}
\end{figure}

\subsubsection{Estimación del Valor Presente de los Flujos de Efectivo y del Valor Terminal del Negocio:}

Para la estimación del valor terminal del negocio, se tomó en cuenta el pronóstico de largo plazo de crecimiento  para la economía mexicana obtenido de Banxico\footnote{Fuente: \url{https://www.banxico.org.mx/publicaciones-y-prensa/encuestas-sobre-las-expectativas-de-los-especialis/encuestas-expectativas-del-se.html}}, a la cual se le incorporó el efecto de la inflación de largo plazo de los pronósticos de Banxico, mediante la ecuación Fisher. \\

Los flujos de efectivo del negocio se descontaron a valor presente, mediante la tasa WACC. Se muestran los resultados:\\


\begin{figure}[H]
\centering
\includegraphics[width=.9\textwidth]{../0.imagenes/dcf_4}
\end{figure}

\subsubsection{Cifras conclusivas del VALOR DE USO de la UGE.}

Se muestran a continuación los resultados del modelo DCF conforme al enfoque de ingresos, mismo que se considera el modelo idóneo para determinar el valor de uso de cualquier negocio en marcha, en este caso la UGE de NAVALMEX:\\


\begin{figure}[H]
\centering
\includegraphics[width=8cm]{../0.imagenes/dcf_5}\vspace{10pt}

\textbullet \textcolor{principal}{\textbf{Precio Neto de Venta de la UGE (Compañías): \$630’714,755.00 MXN}}
\end{figure}