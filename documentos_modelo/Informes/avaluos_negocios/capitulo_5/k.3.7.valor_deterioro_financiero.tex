\subsection{ANÁLISIS DE DETERIORO NAVALMEX}

\subsubsection{Estimación del Valor de Recuperación de la UGE:}

Se considera como el valor de recuperación de la UGE al monto mayor entre el  precio neto de venta y su valor de uso.\\

\begin{figure}[H]
\centering
\includegraphics[width=12cm]{../0.imagenes/deterioro_1}
\end{figure}

Ante la presencia de alguno de los indicios de deterioro del valor razonable de una firma, como es el caso de la anotación al Buró de Crédito por parte de ``Atradius'' según los hechos comentados por el solicitante y que se mencionan en el APÉNDICE 2 de este estudio, es posible cuantificar la pérdida por deterioro.

\subsubsection{Estimación del Deterioro Financiero de NAVALMEX:}

Una vez determinado el Valor de Recuperación de la Unidad Generadora de Efectivo, si dicho valor es menor al Valor Neto en Libros de la UGE, se debe considerar que \textbf{\underline{existe deterioro.}} \\

Se muestran a continuación los resultados del Análisis de Deterioro de la empresa NAVALMEX, cuyo indicio de deterioro consistió según el solicitante en la anotación ilegal en el buró de crédito por parte de ATRADIUS mencionada en el APÉNDICE 2 , la cual le ha ocasionado diversos problemas operativos y financieros de 2017 a la fecha y ha repercutido negativamente en sus cifras financieras, según se aprecia:

\begin{figure}[H]
\centering
\includegraphics[width=10cm]{../0.imagenes/deterioro_2}\\[10pt]

\textbullet \textcolor{principal}{\textbf{Valor Razonable del Deterioro Financiero de la UGE: }}\\[10pt]

\textcolor{principal}{\textbf{\$487’280,547.00 MXN}}\\[10pt]

(Cuatrocientos ochenta y siete millones, doscientos ochenta mil, \\
quinientos cuarenta y siete pesos 00/100 M.N.)


\end{figure}


