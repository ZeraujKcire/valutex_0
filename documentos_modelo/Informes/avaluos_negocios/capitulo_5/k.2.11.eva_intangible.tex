\begin{rightcolumn}


\subsubsection*{MÉTODO DE VALOR AGREGADO ESTIMADO (EVA).}

Se llevó a cabo la valuación del activo intangible bajo la metodología EVA (\textit{Estimated Value Added}), en la cual se analizó la proyección del capital invertido del negocio, misma que se contrastó respecto del flujo de operación neto del SKU para evaluar la rentabilidad respecto del costo promedio ponderado de capital (Figura 25):

\begin{figure}[H]
\centering
\includegraphics[width=11cm]{../0.imagenes/eva_01}
\end{figure}

\textcolor{principal}{Estimación del valor terminal del EVA.} Conforme al marco teórico del modelo, resulta adecuado capitalizar un flujo terminal de EVA con posterioridad al horizonte explícito de proyección, mediante una perpetuidad creciente, según se aprecia:

\begin{figure}[H]
\centering
\includegraphics[width=11cm]{../0.imagenes/eva_02}
\end{figure}

\subsubsection*{INDICADOR DE VALOR POR MÉTODO EVA.}

Dicho método permitió reflejar en la valuación el componente de valor agregado por enfoque de ingresos a valor presente,  con un resultado en cifras redondas al 31/12/2022 de \$28'923,994.00 MXN, según se muestra:

\begin{figure}[H]
\centering
\includegraphics[width=11cm]{../0.imagenes/eva_03}
\end{figure}

Por lo anterior, el valuador le asignó un 50.00\% de importancia en la ponderación del valor conclusivo del activo intangible.


\end{rightcolumn}

\begin{leftcolumn}

\begin{figure}[H]
\centering
\caption{Diagrama del EVA}\vspace{5pt}
\includegraphics[width=6cm]{../0.imagenes/eva_11}
\end{figure}

\begin{figure}[H]
\centering
\caption{Comprobación del método EVA}\vspace{5pt}
\includegraphics[width=6cm]{../0.imagenes/eva_12}
\end{figure}

\begin{figure}[H]
\centering
\caption{Indicadores del Macroentorno }
\includegraphics[width=6cm]{../0.imagenes/eva_13}
\end{figure}

\begin{figure}[H]
\centering
\includegraphics[width=6cm]{../0.imagenes/logo_1}
\end{figure}



\end{leftcolumn}