\subsection{APLICACIÓN DEL ENFOQUE FÍSICO PARA CONSTRUCCIONES}

\begin{enumerate}[a)]
\item \textcolor{principal}{\textbf{Estimación del valor de mercado del terreno}}.  Según quedo explicado en el \autoref{sec:k} de este capítulo, se obtuvo un\ valor razonable ponderado del terreno conforme al área permitida de construcción (92,385 $m^2$), por un monto de \textcolor{principal}{\textbf{\$25,802,982.89  USD.}}
\item Estimación del valor razonable de las Construcciones:

\begin{enumerate}[B.1.]
\item \underline{Estimación del VRN por habitación mediante enfoque de mercado.-} Una vez habiendo estimado el valor razonable unitario por habitación (llave) que corresponde al negocio turístico inmobiliario, el valuador llevó a cabo la estimación del valor de reposición nuevo de cada habitación, para lo cual se realizaron diversas entrevistas con constructores de la zona conforme a un análisis de comparabilidad basado en: i) activos, ii) funciones y iii) riesgos; buscando obtener un muestreo de comparables internos conforme al principio de plena competencia (\textit{Arm’s length}). Del anterior análisis de mercado, se obtuvo una estimación que oscila entre los \$711,250 usd/cuarto (Percentil 25) hasta los \$719,463 usd (Percentil 75), con una media aritmética de \$715,275 usd/$m^2$\footnote{La cual representa el 54.20\% del valor de mercado de cada habitación}  según los resultados que se aprecian a continuación:

\begin{figure}[H]
\centering
\includegraphics[width=14cm]{../0.imagenes/enfoque_fisico_1}

\end{figure}

\begin{figure}[H]
\centering
\includegraphics[width=9cm]{../0.imagenes/enfoque_fisico_2}

\end{figure}

\begin{figure}[H]
\centering
\includegraphics[width=14cm]{../0.imagenes/enfoque_fisico_3}

\end{figure}

\item  \underline{Aplicación de Ajustes, Depreciación y Deméritos por Reinversión a las construcciones para la obtención} \\\underline{del VNR (Valor Neto de Reposición)}.\\

 El valuador llevó a cabo la estimación de tres componentes indispensables para poder determinar el valor de reposición del negocio sujeto de estudio en este capítulo: 

\begin{enumerate}[i)]
\item \textcolor{principal}{\textbf{Factor de ajuste por temas locales y zona}}. Se determinó un ajuste del 15\% a la cotización del VRN por cuestiones locales, atendiendo a la dispersión de la muestra y a situaciones particulares de la economía local de San Miguel Allende que encarecen, o en su caso, abaratan  algunas cotizaciones.  Por lo anterior se le aplicó una cobertura de riesgo adicional al pronóstico como una tasa de error (\textit{Error Rate}), buscando ajustar en el VRN la negociación que este tipo de constructores pueden hacer a sus clientes, como por ejemplo, las políticas de pronto pago, descuentos por volumen, entre otros.

\item \textcolor{principal}{\textbf{Tasa de depreciación económica.}} Para esta estimación, se impactó un demérito lineal del 15\% anual, considerando  que la vida útil de este tipo de remodelaciones con materiales de lujo y cuyos componentes son sujetos a influencias de moda y diseño, tienden a mostrar vidas útiles más reducidas; por lo que se eligió dicha tasa, la cual es muy similar al tratamiento fiscal de la amortización de los gastos diferidos y erogaciones preoperativas (art. 33 LISR) y tambi\'en al tratamiento de las tasas de depreciación de construcciones mencionadas en el art. 34 de la LISR.

\item \textcolor{principal}{\textbf{Tasa de requerimientos adicionales de Capex.}} En el segmento de lujo en el que opera la presente empresa turística, existen necesidades constantes de reinversión a mejoras. Asumiendo que la depreciación de las construcciones también funciona como una medida mínima de reinversión para que un activo continúe siendo productivo; se estimó un indicador anual del 5\% anual.

\item \textcolor{principal}{\textbf{Estimación de la tasa de reinversión para evitar la obsolescencia funcional de Instalaciones y equipamiento (FF\&E)}}.  Se determinó una tasa de reinversión adicional del 20\% al VRN para poder conservar la percepción de que las instalaciones del Hotel sean consideradas como nuevas y de lujo.  

\end{enumerate}
\end{enumerate}
\end{enumerate}

\subsection{APLICACIÓN DEL ENFQUE DE COSTOS. Conclusión de valor.}
Se muestra a continuación la estimación del Valor de Reposición Neto de las Construcciones de ``ROSEWOOD''; conforme al propósito y uso del avalúo mencionado:

\begin{figure}[H]
\centering
\includegraphics[width=14cm]{../0.imagenes/enfoque_fisico_4}

Conforme a la aplicación del Enfoque de Físico o de Costos, \\
al \fechaValoresCorto{} por un monto de:\\
\textbf{\$44,030,183.18 USD.}\\
(Cuarenta y cuatro millones, treinta mil, ciento treinta y ocho d\'olares 00/100 Moneda de los Estados Unidos de Am\'erica)
\end{figure}

\textcolor{principal}{\textbf{RECONCILIACIÓN DE VALORES.} Aplicación del enfoque residual dinámico para la estimación del valor intangible del operador turístico.}\\

 El valuador aplicó la metodología residual para determinar el valor razonable del capital intangible atribuible  a la intervención del operador turístico en el negocio. Dicha participación en el valor razonable se infiere financieramente como la diferencia resultante de restar al valor de la firma el valor de mercado del terreno y de las construcciones, según se muestra:
 
 \begin{figure}[H]
\centering
\includegraphics[width=12cm]{../0.imagenes/enfoque_fisico_reconciliacion}

\end{figure}



