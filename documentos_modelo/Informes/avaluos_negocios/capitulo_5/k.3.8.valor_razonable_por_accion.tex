\subsection{Valor razonable unitario por acción.}

A petición expresa del solicitante, el perito valuador procedió a realizar el cálculo del \textcolor{principal}{valor razonable} que corresponde a cada título accionario. Por lo anterior, se contabilizó primero el número de acciones disponibles (según la información proporcionada por el solicitante), para después dividir el valor razonable total del \textit{Capital Accionario} entre el número de acciones; obteniéndose así un \textcolor{principal}{valor unitario por acción de \$\valorUnitarioAccion{} pesos por acción}, según se aprecia:

\begin{figure}[H]
\centering
\includegraphics[width=.7\textwidth]{../0.imagenes/valor_por_accion}\\

\textbf{\textcolor{principal}{Conclusión del dictamen:}}\\

\textbf{ \$\valorUnitarioAccion{} MXN}}\\[5pt]
(\textcolor{principal}{\valorUnitarioAccionLetra{} 00/100 M.N.})
\end{figure}
